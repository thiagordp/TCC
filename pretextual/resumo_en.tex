\textAbstract {In the last two decades the concept of the Internet of Things has become popular, proposing the introduction of technology and the interconnection of different objects such as lamps, refrigerators, cars, etc. through a network, often the Internet. Thus, the number of interconnected devices tends to grow and high amounts of data will be produced. It is then necessary to use tools capable of processing and extracting relevant information from this data. Among the existing methods are the recommendation systems, which, from the analysis of the records, are able to trace the users profile and provide suggestions. In recent years, these systems have been incorporated in a variety of contexts as streaming platforms and, gradually, applications will introduce these ideas into the Internet of Things. Among the environments in which both concepts can be integrated are smart houses, where the use of technology opens the way to new forms of interaction with the home. Common devices such as refrigerators add technologies and new features, however, the existing models only provide digital interactions, however do not pay attention to the tastes and habits of users. Taking this into account, the present work proposes a new refrigerator model capable of monitoring interactions and tracing preferences profile in products. This is possible thanks to the use of RFID tags, which will have the role of identifying the various products contained in the refrigerator and the application of recommendation algorithms. In the end, the equipment will be able to offer new products to the user according to their profile, as well as promotions of related items and suggestions of recipes that include products contained in the refrigerator}
\keywords {Internet of Things. Recommender systems. Smart fridge. User preferences}

