\textAbstract {
In the last two decades the concept of the Internet of Things has become popular, proposing the introduction of technology and the interconnection of different objects such as lamps, refrigerators, cars, etc. through a network, often the Internet. Thus, the number of interconnected devices tends to grow producing a large amount of data. It is then necessary to use tools capable of processing and extracting relevant information from such data. Among the existing areas are the Recommender Systems based on analyzes of user interactions being able to trace user profiles and provide suggestions. In recent years, these systems have been incorporated into a variety of contexts such as streaming platforms and, gradually, other applications will introduce such ideas into the Internet of Things context. Among the environments in which both concepts can be integrated are smart houses, in which the use of technology creates new forms of interaction with the home. Common devices present in the houses such as refrigerators add, thus, technologies and new features, however, existing models only provide digital interactions, however do not address the tastes and habits of users. Taking this into account the present work proposes a model of refrigerator capable of monitoring interactions and tracing preferences profile by products. This is possible by using RFID tags which have the role of identifying the various products contained in the refrigerator and the application of recommendation algorithms. In the end, an evaluation was made, based on a created scenario, where the equipment was able to monitor the available products and offer recommendations to users from their profiles as well as the replacement of products considered essential. In addition, it was possible to recommend recipes both from the contents of the refrigerator as well as from the user profile. From the above, it is concluded that the integration of the Internet of Things and Recommendation Systems concepts can contribute to help users in their daily tasks facilitating, in some way, decision making process.}


%minha proposta



% fim



\keywords {Internet of Things. Recommender systems. Smart fridge. User preferences.}

% Ao final, o equipamento será capaz de oferecer novos produtos ao usuário de acordo com o seu perfil, além de promoções de itens relacionados e sugestões de receitas que englobem produtos contidos na geladeira.

% V1:
%In the last two decades the concept of the Internet of Things has become popular, proposing the introduction of technology and the interconnection of different objects such as lamps, refrigerators, cars, etc. through a network, often the Internet. Thus, the number of interconnected devices tends to grow and high amounts of data will be produced. It is then necessary to use tools capable of processing and extracting relevant information from this data. Among the existing methods are the recommendation systems, which, from the analysis of the records, are able to trace the users profile and provide suggestions. In recent years, these systems have been incorporated in a variety of contexts as streaming platforms and, gradually, applications will introduce these ideas into the Internet of Things. Among the environments in which both concepts can be integrated are smart houses, where the use of technology opens the way to new forms of interaction with the home. Common devices such as refrigerators add technologies and new features, however, the existing models only provide digital interactions, however do not pay attention to the tastes and habits of users. Taking this into account, the present work proposes a new refrigerator model capable of monitoring interactions and tracing preferences profile in products. This is possible thanks to the use of RFID tags, which will have the role of identifying the various products contained in the refrigerator and the application of recommendation algorithms. At the end, the equipment was able to monitor the available products and offer new products to the user, based on their profile, as well as the replacement of products considered essential. In addition, it was possible to recommend recipes from the contents of the refrigerator as well as from the user profile.