





\textoResumo {
%Introdução e contextualização
Nas últimas duas décadas conceito da Internet das Coisas vem se popularizando, propondo a introdução da tecnologia e da interconexão de objetos distintos, como lâmpadas, geladeiras, carros, etc., através de uma rede, frequentemente, a Internet. Assim, o número de aparelhos interligados tende a crescer e altas quantidades de dados serão produzidas. É, então, necessário o uso de ferramentas capazes de processar e extrair informações relevantes desses dados. Entre os métodos existentes estão os sistemas de recomendação, aos quais, a partir de análises dos registros, são capazes de traçar o perfil de usuários e fornecer sugestões. Nos últimos anos, esses sistemas vêm sendo incorporados em diversos contextos como plataformas de streaming e, gradativamente, aplicações introduzirão essas ideias no âmbito da Internet das Coisas. Entre os ambientes, nos quais é possível integrar ambos os conceitos, estão as casas inteligentes ou smart homes, onde o uso da tecnologia abre caminho para novas formas de interação com o lar. Dispositivos comuns como geladeiras agregam tecnologias e novas funcionalidades, no entanto, os modelos existentes apenas proporcionam interações digitais, contudo não atentam aos gostos e hábitos dos usuários. Levando isso em conta, o presente trabalho propõe um novo modelo de geladeira capaz de monitorar as interações e traçar perfil de preferências em produtos. Isso é possível graças ao uso de etiquetas RFID, que terão o papel de identificar os diversos produtos contidos na geladeira e da aplicação de algoritmos de recomendação. Ao final, o equipamento será capaz de oferecer novos produtos ao usuário de acordo com o seu perfil, além de promoções de itens relacionados e sugestões de receitas que englobem produtos contidos na geladeira.}

\palavrasChave {Internet das Coisas. Sistemas de Recomendação. Geladeira Inteligente. Preferências do usuário.}

% No final vai ter 4 PARTES: 

% Introdução / cont
% materiais e métodos
% resultados alcançados
% conclusões

% TCC 1 
% Introdução / cont
% materiais e métodos
% resultados esperados


% Nas últimas duas décadas o paradigma da Internet das Coisas vem emergindo, propondo a introdução da tecnologia e da inter-conexão de objetos distintos como lâmpadas, geladeiras, carros, etc., através de uma rede, sendo esta, muitas vezes, a Internet. Além disso, com o grande número de aparelhos interligados, se produzirá altas quantidades de dados provenientes das interações com o ambiente e com os usuários. Assim, torna-se necessário o uso de ferramentas que processem e extraiam informações relevantes desses dados. Entre os métodos existentes estão os sistemas de recomendação, aos quais, a partir da análise dos registros das interações, são capazes de traçar o perfil de usuários. Nos últimos anos, esses sistemas vêm sendo aplicados em diversos contextos como plataformas de \textit{streaming}.  O presente trabalho propõe um novo modelo de interação entre geladeiras e usuários, onde o equipamento a partir dos algoritmos citados é capaz de monitorar as interação com o usuário e a traçar o seu perfil de preferências em produtos. Isso é possível graças ao uso etiquetas RFID, aos quais operam por radio frequência, que terão o papel de identificar os diversos produtos contidos na geladeira e as interações do usuário com tais produtos. A partir disso é possível traçar um perfil do usuário a partir de algoritmos de recomendação. Ao final, a geladeira será capaz de oferecer novos produtos ao usuário de acordo com o seu perfil além de promoções. É possível que o usuário selecione um conjunto de produtos essenciais, ou seja, que jamais podem faltar e a partir disso realizar compras automáticas. A partir do perfil, será possível também fazer sugestões de receitas que englobem os produtos contidos na geladeira.