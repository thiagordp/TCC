\agradecimento{
% Deus e família
Agradeço, primeiramente, a Deus pelo dom da vida e pela oportunidade de ter conhecido pessoas tão importantes ao longo dos últimos anos. Além disso, tenho de manifestar a minha profunda gratidão aos meus pais e meu irmão, que ao longo destes anos, foram muito compreensivos e pacientes comigo e, constantemente, me incentivaram a estudar.

%Alan
Gostaria também de agradecer ao meu conterrâneo, Alan Kunz Cechinel, que, desde o começo do curso, se mostrou um grande amigo, um companheiro de noites de estudo e uma grande inspiração.

% Professores
Devo também mostrar minha gratidão a três professores em especial na Universidade aos quais me proporcionaram aprendizado profissional e pessoal muito importante. O primeiro deles, o Prof. Dr. Alexandre Leopoldo Gonçalves, que ao longo dos últimos dois anos, me orientou com muita dedicação, paciência e profissionalismo e me deu lições sobre a vida acadêmica, profissional e pessoal que levarei por toda a minha vida. Além disso, a Prof. Dra. Analúcia Schiaffino que me concedeu o meu primeiro projeto na universidade, em que pude aprender muito e que mudou a minha visão de Universidade. Por fim, o Prof. Dr. Anderson Luiz Fernando Perez, que me permitiu quase morar no Laboratório de Automação de Robótica Móvel (LARM) e que passou, a cada conversa no laboratório, uma nova lição sobre a vida.

% EJEC
Não poderia esquecer da Empresa Júnior de Engenharia de Computação (EJEC) e a todos os seus membros que, ao longo dos últimos dois anos, me ensinaram o que é ser um empreendedor e colocar o empreendedorismo em prática. Poderia dizer com convicção que foi a experiência mais impactante dentro da universidade. Nada disso teria acontecido sem o convite do atual presidente e fundador, Kaio Anselmo.

% Amigos e colegas
Agradeço a todos os colegas e professores que conheci e que fiz grandes amizades.

% Universidade
Por fim, gostaria de agradecer à Universidade Federal de Santa Catarina pela oportunidade de ter vivenciado tudo isso.
}



