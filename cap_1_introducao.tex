\chapter{Introdução}

%TODO: Introdução: Contextualização dos dias atuais... (Era uma vez)
% Apresentar o tema e o que significa

%TODO: Escrita sobre Internet das Coisas

	%TODO: Tecnologias que proporcionam a IoT

%TODO: Escrita sobre Objetos inteligentes

%TODO: Escrita s\part{title}obre Sistemas de Recomendação



% Contextualização de todo o trabalho

% Apresentar o cenário.

Nas últimas décadas, presencia-se os acelerados avanços em ciência e tecnologia impulsionados por empresas dos mais diversos ramos e pelas constantes pesquisas nas universidades. 

%TODO: Aprimorar

Um dos avanços mais significativos é a Internet, com um grande impacto no desenvolvimento da economia global e na sociedade atual. Em duas décadas tem decorrido um grande crescimento na disponibilidade do acesso à rede. Em setembro de 2016, o número de usuário da rede mundial de computadores era de, aproximadamente, 3,75 bilhões, cerca de metade da população mundial \cite{Stats2017}.  Outro grande avanço se tem nos celulares, aos quais evoluíram tanto nos últimos anos que passaram de simples e grandes telefones sem fio à dispositivos menores, no entanto, com acesso a Internet, recursos avançados de áudio e vídeo, poder de processamento equiparável ao de computadores de mesa e/ou notebooks.

Em meio ao domínio da Internet convencional, um novo paradigma surgiu no meio acadêmico e aos poucos ganha terreno nas grandes empresas: a sua proposta é levar a tecnologia a objetos do dia a dia, como condicionadores de ar, lâmpadas, fogões etc., e, assim, criar novas formas de interação além de funcionalidades inéditas, seguindo o exemplo dos \textit{smartphones}. 

Mencionada pela primeira vez, por Kevin Ashton, em 1999 \cite{Ashton2009}, a Internet das Coisas ou IoT (em inglês, \textit{Internet of Things}) está cada vez mais próxima da realidade. Abrigará um imenso ecossistema de dispositivos com capacidade de processamento, sensoriamento, conexão com demais dispositivos, entre outros avanços. Estima-se que, em 2020, cerca de 24 bilhões de dispositivos IoT estejam conectados, implicando em cerca de quatro dispositivos por pessoa \cite{Meola2016}. 

Para entender a demanda de tantos dispositivos serão necessários novas formas de interconexão desses dispositivos, bem como formas de fornecimento de energia

% TODO: Falar sobre smart grids e fog

Os dispostivos com funcionalidades expandidas, os chamados \textit{smart objects}, poderão, a partir da IoT, operar em conjunto para formar os chamados \textit{smart environment}, ambientes nos quais a integração dos dispositivos agrega novas funcionalidades e formas de interação para aquele ambiente. Um desses ambientes pode ser a cozinha inteligente ou \textit{smart kitchen}. Uma cozinha inteligente é capaz de prover ao usuário novas maneiras de interagir com os utensílios na preparação de alimentos, escolha de produtos entre outros. A partir disso, surgem diversas oportunidades em termos de criação de produtos.

Uma das maneiras de ampliar as funcionalidades são os sistemas de recomendação, capazes de entender os gostos e preferências do usuário e, a partir disso, recomendar novos itens que ele talvez não conheça e possa a vir se interessar.

\section{Problemática}


\section{Objetivos}
Esta seção apresenta o objetivo geral e os objetivos específicos do trabalho.

\subsection{Geral}

\textit{Rascunho}

Desenvolver uma geladeira capaz de monitorar os produtos contidos nela e, a partir disso, fazer compras automaticamente quando estes estiverem em falta.
Além de fazer recomendações de receitas com base nos padrões de consumo dos produtos presentes na geladeiras.

% TODO: Detalhar especificamente o que dá pra fazer ou ser mais geral?
% TODO: Propor um sistema diferente para geladeiras ou uma geladeira com recursos novos?

\subsection{Específicos}
\begin{enumerate}
	\item Levantar o estado da arte com relação a Internet das Coisas e Sistemas de Recomendação
	\item Propor um sistema de monitoramento de produtos 
	\item Propor um projeto de leitura e monitoramento dos produtos contidos na geladeira.
\end{enumerate}

\section{Justificativa}


\section{Organização do trabalho}

O capítulo 2 é sobre Internet das Coisas

O capítulo 3 é sobre Sistemas de Recomendação

O capítulo 4 é sobre o Sistema Proposto

O capítulo 5 é sobre a Avaliação do Sistema Proposto
	- Descrever um cenário
	- Avaliar e discutir o cenário a partir do sistema proposto
	
O capítulo 6 é sobre Considerações Finais




