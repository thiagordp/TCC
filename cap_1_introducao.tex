\chapter{Introdução}

%TODO: Introdução: Contextualização dos dias atuais... (Era uma vez)
% Apresentar o tema e o que significa

%TODO: Escrita sobre Internet das Coisas

	%TODO: Tecnologias que proporcionam a IoT

%TODO: Escrita sobre Objetos inteligentes

%TODO: Escrita sobre Sistemas de Recomendação



% Contextualização de todo o trabalho

% Apresentar o cenário.

Nas últimas décadas, o mundo presencia os acelerados avanços em ciência e tecnologia impulsionados por empresas dos mais diversos ramos além das constantes pesquisas nas universidades. Estima-se que XXXXXXXXX %TODO: Estimativa sobre o avanço da internet

Um dos avanços mais dignos de menção é a Internet. Ao longo das últimas décadas tem decorrido um grande crescimento na "disponibilidade" do acesso. Outro grande avanço se tem nos celulares, aos quais evoluíram tanto nos últimos anos que passaram de simples telefones sem fio a dispositivos com acesso a Internet, recursos avançados de áudio e vídeo, poder de processamento equiparável ao de computadores de mesa e/ou notebooks.

Nos últimos anos, um novo paradigma surgiu no meio acadêmico e aos poucos ganha terreno nas grandes empresas: a sua proposta é de seguir o exemplo dos smartphones e levar a tecnologia a objetos do dia a dia e, assim, criar formas de interação e funcionalidades inéditas. Mencionado pela primeira vez, por XXXXX %TODO: Colocar nome do cara que inventou a IoT
, em XXXXXX, %TODO: ano de criação do termo IoT
a Internet das Coisas ou IoT (em inglês, \textit{Internet of Things}) está cada vez mais próxima da realidade. Abrigará um imenso ecossistema de dispositivos com capacidade de processamento, conexão com outros dispositivos, entre outros avanços. Estima-se que, em 2020, XXXXXXXX %TODO: colocar a estimativa sobre 2020 e a IoT.

Para entender a demanda de tantos dispositivos serão necessários novas formas de interconexão desses dispositivos, bem como formas de fornecimento de energia

% TODO: Falar sobre smart grids e fog

Os dispostivos com funcionalidades expandidas, os chamados \textit{smart objects}, poderão, a partir da IoT, operar em conjunto para formar os chamados \textit{smart environment}, ambientes nos quais a integração dos dispositivos agrega novas funcionalidades e formas de interação para aquele ambiente. Um desses ambientes pode ser a cozinha inteligente ou \textit{smart kitchen}. Uma cozinha inteligente é capaz de prover ao usuário novas maneiras de interagir com os utensílios na preparação de alimentos, escolha de produtos entre outros. A partir disso, surgem diversas oportunidades em termos de criação de produtos.

Uma das maneiras de ampliar as funcionalidades são os sistemas de recomendação, capazes de entender os gostos e preferências do usuário e, a partir disso, recomendar novos itens que ele talvez não conheça e possa a vir se interessar.



\section{Problemática}


\section{Objetivos}

Esta seção apresenta o objetivo geral e os objetivos específicos do trabalho.

\subsection{Geral}

\textit{Rascunho}

Projetar e implementar uma geladeira capaz de monitorar os produtos contidos nela e, a partir disso, fazer compras automaticamente quando estes estiverem em falta.
Além de fazer recomendações de receitas com base nos padrões de consumo dos produtos presentes na geladeiras.

% TODO: Detalhar especificamente o que dá pra fazer ou ser mais geral?
% TODO: Propor um sistema diferente para geladeiras ou uma geladeira com recursos novos?

\subsection{Específicos}
\begin{enumerate}
	\item Levantar o estado da arte com relação a Internet das Coisas e Sistemas de Recomendação
	\item Propor um sistema de monitoramento de produtos 
	\item Propor um projeto de leitura e monitoramento dos produtos contidos na geladeira.
\end{enumerate}

\section{Justificativa}


\section{Organização do trabalho}

O capítulo 1 é sobre

O capítulo 2 é sobre

O capítulo 3 é sobre