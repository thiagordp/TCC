%%%%%%%%%%%%%%%%%%%%%%%%%%%%%%%%%%%%%%%%%%%%%%%%%%%%%%%%%%%%%%%%%%%%%%%
% Universidade Federal de Santa Catarina             
% Biblioteca Universitária                     
%----------------------------------------------------------------------
% Exemplo de utilização da documentclass ufscThesis
%----------------------------------------------------------------------                                                           
% (c)2013 Roberto Simoni (roberto.emc@gmail.com)
%         Carlos R Rocha (cticarlo@gmail.com)
%         Rafael M Casali (rafaelmcasali@yahoo.com.br)
%%%%%%%%%%%%%%%%%%%%%%%%%%%%%%%%%%%%%%%%%%%%%%%%%%%%%%%%%%%%%%%%%%%%%%%
\documentclass{ufscThesis} % Definicao do documentclass ufscThesis	

%----------------------------------------------------------------------
% Pacotes usados especificamente neste documento
\usepackage{graphicx} % Possibilita o uso de figuras e gráficos
\usepackage{color}    % Possibilita o uso de cores no documento
%\usepackage{pdfpages} % Possibilita a inclusão da ficha catalográfica
\usepackage{listings} % Possibilita colocar códigos

\usepackage{caption} % colocar textos em figuras
\usepackage{subfigure}

\usepackage{xifthen} % deixa fazer uns if-then-else mutcho loko

\usepackage[brazil]{babel}  % usados para arrumar os caracteres em português
\usepackage[utf8]{inputenx} % 
\usepackage[T1]{fontenc}    %

\usepackage{bibentry} % permite usar o comando nobibliography

\usepackage{minted} % Sintaxe colorida

\usepackage{xparse} % Permite definir umas funções com mais recursos.

%----------------------------------------------------------------------
% comandos de customização dos pacotes
\renewcommand\listingscaption{Algoritmo}
% \renewcommand{\lstlistlistingname}{Lista de \lstlistingname s}% List of Listings -> List of Algorithms

\graphicspath{{figuras/}}

%----------------------------------------------------------------------
% Comandos criados pelo usuário
\newcommand{\afazer}[1]{{\color{red}{#1}}} % Para destacar uma parte a ser trabalhada
\newcommand{\ABNTbibliographyname}{REFERÊNCIAS} % Necessário para abnTeX 0.8.2

\newcommand{\figura}[5][Extraido de:]{
\begin{figure}[h!tb]
	\centering
	\caption{#3.}
	\includegraphics[width=#4]{#2.png}
	\ifthenelse{\isempty{#5}}{}{%
		\\ #1 \citeonline{#5}.
	}	
	\label{fig:#2}
\end{figure}
}

\NewDocumentEnvironment{tabela}{mm}
{
	\begin{table}[htb]
		\centering
}
{
		\caption{#2}
		\label{tab:#1}
	\end{table}
}

\newcommand{\reffig}[1]{Figura \ref{fig:#1}}
\newcommand{\refalg}[1]{Algoritmo \ref{alg:#1}}
\newcommand{\reftab}[1]{Tabela \ref{tab:#1}}
\newcommand{\refcap}[1]{\footnote{vide tópico \ref{cap:#1}.}}
\newcommand{\reftop}[1]{Seção \ref{cap:#1}}

\newcommand{\criarSigla}[3][]{%
	\ifthenelse{\isempty{#1}}{%
		#2 - #3\sigla{#3}{#2}%
	}{%
		\emph{#2} - #3\sigla{#3}{\emph{#2}}\footnote{Traduzido como: #1.}%
	}%
}

%----------------------------------------------------------------------


%----------------------------------------------------------------------
% Identificadores do trabalho
% Usados para preencher os elementos pré-textuais
\instituicao[a]{Universidade Federal de Santa Catarina} % Opcional
\departamento[a]{} %TODO departamento de engenharia da computação?
\curso[a]{Universidade Federal de Santa Catarina}
\documento[o]{Trabalho\ de\ Conclusão\ de\ Curso} % [o] para dissertação [a] para tese
\titulo{OpenAutOS}
\subtitulo{Um Sistema Operacional Veícular} % Opcional
\autor{Bruno Fontana Canella}
\grau{Bacharel em Engenharia de Computação}
\local{Araranguá} % Opcional (Florianópolis é o padrão)
\data{04}{dezembro}{2016}
\coordenador[Coordenador\\Universidade Federal de Santa Catarina]{Prof. Dr. Anderson Luiz Fernandes Perez}
\orientador[Orientador\\Universidade Federal de Santa Catarina]{Prof. Dr. Anderson Luiz Fernandes Perez}
%\coorientador[Coorientador\\Universidade ...]{Prof. Dr.}

\numerodemembrosnabanca{3} % Isso decide se haverá uma folha adicional
\orientadornabanca{sim} % Se faz parte da banca definir como sim
\coorientadornabanca{não} % Se faz parte da banca definir como sim
\bancaMembroA{Primeiro membro\\Universidade ...} %Nome do presidente da banca
\bancaMembroB{Segundo membro\\Universidade ...}      % Nome do membro da Banca
\bancaMembroC{Terceiro membro\\Universidade ...}     % Nome do membro da Banca
\bancaMembroD{Quarto membro\\Universidade ...}       % Nome do membro da Banca
%\bancaMembroE{Quinto membro\\Universidade ...}       % Nome do membro da Banca
%\bancaMembroF{Sexto membro\\Universidade ...}        % Nome do membro da Banca
%\bancaMembroG{Sétimo membro\\Universidade ...}       % Nome do membro da Banca

\dedicatoria{Este trabalho é dedicado aos meus colegas de classe e aos meus queridos pais.}

\agradecimento{Eu queria mandar um beijo pro meu pai, pra minha mãe e especialmente pra você Xuxa.}

\epigrafe{Escrever um TCC é que nem fazer estrada. Depois que tá feita a base é só passar asfalto.}
{(Anderson, 2016)}

\textoResumo {Afim de padronizar o desenvolvimento de sistemas e dispositivos para veículos automotivos, foram criados padrões abertos, regidos pelas principais montadoras de veículos, as quais os batizaram como AUTOSAR. Dentre estes padrões, foi estabelecido um para governar o desenvolvimento de sistemas operacionais, capazes de gerenciar os recursos eletrônicos de um veículo. A maioria das soluções existentes hoje no mercado, e que respeitam este padrão, são proprietárias. Com o intuito de criar uma alternativa de qualidade comercial, este trabalho visa desenvolver, documentar e validar um sistema operacional automotivo de código aberto, nas normas estabelecidas pelo padrão AUTOSAR.}
\palavrasChave {Automação Veicular, Sistema Operacional Embarcado, AUTOSAR, OpenAUTOS.}

\textAbstract {botar no google translate, e é isso ai.}
\keywords {Operating Systems, Vehicle Automation, AUTOSAR, Open Source.}

%----------------------------------------------------------------------
% Início do documento                                
\begin{document}
%--------------------------------------------------------
% Elementos pré-textuais
\capa  
%\folhaderosto[comficha] % Se nao quiser imprimir a ficha, é só não usar o parâmetro

\begin{titlepage}
	\vfill
	\begin{center}
		%{\large \ABNTautordata} \\[5cm]
		\ABNTautordata \\[5cm]
		
		\tituloformat{\ABNTtitulodata}: \\
		\tituloformat{\subtitulodata} \\[1cm]
		
		\hspace{.45\textwidth} % posicionando a minipage
		\begin{minipage}{.5\textwidth}
			\begin{espacosimples}
				
				Trabalho de Conclusão de Curso submetido à \ABNTinstituicaodata,
				como parte dos requisitos necessários para a obtenção do Grau de
				Bacharel em Engenharia de Computação.
				
				Orientador: Prof. Anderson Luiz Fernandes Perez, Dr.
			\end{espacosimples}
		\end{minipage}
		\vfill \localformat \ABNTlocaldata, \mesdata\ de \anodata.
	\end{center}
	\vspace{1cm}
\end{titlepage}

%\folhaaprovacao
%\paginadedicatoria
%\paginaagradecimento
%\paginaepigrafe
\paginaresumo
%\paginaabstract
%\pretextuais % Substitui todos os elementos pre-textuais acima
\listadefiguras % as listas dependem da necessidade do usuário
\listadetabelas 
\listadeabreviaturas
%\listadesimbolos
\sumario

%--------------------------------------------------------
% Simbolos e Abreviaturas
\abreviatura{RAM}{\emph{Random Access Memory}}
\abreviatura{ROM}{\emph{Read-Only Memory}}
%--------------------------------------------------------
% Elementos textuais

\chapter{Introdução}

Desde seu surgimento, popularização e evolução até os dias atuais, os veículos automotivos aumentaram em muito a sua complexidade, a ponto de que apenas o conhecimento mecânico do veículo não é mais suficiente. A quantidade de componentes eletrônicos presentes nos veículos automotivos aumentou consideravelmente passando, inclusive, a substituir sistemas puramente mecânicos. Dentre os fatores que alavancaram estas mudanças destacam-se o barateamento, miniaturização e popularização dos componentes eletrônicos, bem como a adequação da industria automobilística as novas leis de trânsito, que passaram a ditar a emissão máxima de poluentes e a exigir mecanismos de segurança, como o ABS e o \emph{Airbag}.

Com o propósito de padronizar e, assim, facilitar o desenvolvimento e intercambialidade de auto-peças por terceiros, as principais montadoras e fabricantes de veículos entraram em um consenso, estipulando um padrão de normas de desenvolvimento para veículos automotivos, chamada de AUTomotive Open System ARchitecture ou AUTOSAR, a qual se encontra em sua quarta revisão.

Para gerenciar os diversos módulos eletrônicos agora presentes em um veículo, bem como garantir a interoperabilidade entre eles, foram criadas normas referente ao desenvolvimento de sistemas operacionais. Estas normas visam o estabelecimento de padrões para o funcionamento, comunicação e especificações do sistema, sem sacrificar a liberdade criativa de desenvolvimento do sistema, como a seleção de hardwares e implementação de algoritmos.

\section{Justificativa e Motivação}

A maioria das soluções em sistemas operacionais automotivos são exclusivamente comerciais e de código fechado. Embora existam soluções de código aberto para sistemas embarcados, não existe, na atualidade, um \criarSigla{Sistema Operacional}{SO} de código aberto homologado nos padrões do AUTOSAR. O projeto que mais chega próximo deste cenário é o Trampoline, que se encontra em fase de homologação da norma \cite{Trampoline:HOME}.

Visando a criação de um SO embarcado, para uso em veículos populares, e que mantivesse um padrão de código aberto, surgiu a idealização do OpenAUTOS. Através do desenvolvimento do OpenAUTOS, deseja-se alcançar um SO nacional que seja referência na área, utilizando componentes e tecnologias com alta disponibilidade e de fácil acesso, além de agregar contribuições com a própria comunidade acadêmica.

\section{OBJETIVOS}

Esta seção apresenta o objetivo geral e os objetivos específicos deste trabalho.

\subsection{Geral}

Desenvolver um sistema operacional embarcado de código aberto que atenda as normas estabelecidas pelo padrão AUTOSAR.

\subsection{Específicos}
\begin{enumerate}
	\item Levantar o estado da arte com respeito a algoritmos para sistemas operacionais embarcados;
	\item Estudar padrões de sistemas automotivos;
	\item Levantar os requisitos para implementação de um SO de acordo com a norma AUTOSAR;
	\item Estabelecer um projeto de código aberto em um repositório online;
	\item Documentar o código do projeto;
	\item Criar um modelo físico que utilize o SO desenvolvido;
	\item Validar o OpenAUTOS em um veículo automotivo real.
\end{enumerate}

\section{Organização do trabalho}

Este trabalho está dividido em 4 capítulos, contando com a introdução.

O \textbf{Capítulo 2} apresentará os domínios eletrônicos de funcionamento em veículos, seus meios de comunicação, unidades de controle e engenharia de software automotivo.

O \textbf{Capítulo 3} abordará os principais conceitos sobre sistemas operacionais. Ao final do capítulo serão relatados alguns estudos de caso a respeito de sistemas operacionais embarcados com foco para a automação veicular.

O \textbf{Capítulo 4} apresentará a proposta do SO OpenAUTOS, destacando as escolhas tanto do projeto do software bem como a arquitetura de hardware adotada.

\include{cap_2_automacao_veicular}

\include{cap_3_sistemas_operacionais_embarcados}

%%\include{cap_3_sistemas_operacionais_embarcados_old}

\include{cap_4_openautos}

%--------------------------------------------------------
%Referências

\bibliographystyle{ufscThesis/ufsc-alf}
\bibliography{referencias}

%-------------------------\textsl{}-------------------------------
% Elementos pós-textuais
%\apendice
%\chapter{Exemplificando um Apêndice}
%Texto do Apêndice aqui. 

%\anexo
%\chapter{Exemplificando um Anexo}
%Texto do anexo aqui.
\end{document}
