%%%%%%%%%%%%%%%%%%%%%%%%%%%%%%%%%%%%%%%%%%%%%%%%%%%%%%%%%%%%%%%%%%%%%%%
% Universidade Federal de Santa Catarina             
% Biblioteca Universitária                     
%----------------------------------------------------------------------
% Exemplo de utilização da documentclass ufscThesis
%----------------------------------------------------------------------                                                           
% (c)2013 Roberto Simoni (roberto.emc@gmail.com)
%         Carlos R Rocha (cticarlo@gmail.com)
%         Rafael M Casali (rafaelmcasali@yahoo.com.br)
%%%%%%%%%%%%%%%%%%%%%%%%%%%%%%%%%%%%%%%%%%%%%%%%%%%%%%%%%%%%%%%%%%%%%%%
\documentclass{ufsctex} % Definicao do documentclass ufscThesis	


%----------------------------------------------------------------------
% Pacotes usados especificamente neste documento

\usepackage{graphicx, subcaption} % Possibilita o uso de figuras e gráficos
\usepackage{color}    %  Possibilita o uso de cores no documento
\usepackage{pdfpages} % Possibilita a inclusão da ficha catalográfica
\usepackage{listings} % Possibilita colocar códigos
\usepackage{lipsum}
\usepackage{caption} % colocar textos em figuras
\captionsetup{justification=raggedright,singlelinecheck=false, labelsep=endash,position=above, skip=0pt}%, textfont=footnotesize}
\usepackage{subfigure}
\usepackage{xifthen} % deixa fazer uns if-then-else mutcho loko
\usepackage[brazil]{babel}  % usados para arrumar os caracteres em português
\usepackage[utf8]{inputenx} % fa
%\usepackage[utf8]{inputenc}
\usepackage[T1]{fontenc}    %
\usepackage{mathtools}
\usepackage{graphicx}
\usepackage{lmodern}
\usepackage{calc}
\usepackage{enumitem}
\usepackage{bibentry} % permite usar o comando nobibliography
\usepackage{minted} % Sintaxe colorida
\usepackage{booktabs}
\usepackage{xparse} % Permite definir umas funções com mais recursos.
\usepackage{adjustbox}
\usepackage{array}
\usepackage{mdframed}
\usepackage{float}
\newcolumntype{C}[1]{>{\centering\let\newline\\\arraybackslash\hspace{0pt}}m{#1}}
%\usepackage[square,numbers]{natbib}
%----------------------------------------------------------------------
% comandos de customização dos pacotes
%\renewcommand{\thelisting}{\arabic{listing}}
%\SetupFloatingEnvironment{listing}{name=Algoritmo,listname={Lista de Algoritmos},within=none}

%\captionsetup[listing]{position=above,skip=0pt}
%\captionsetup[figure]{position=above,skip=0pt}
%\captionsetup[table]{position=above,skip=0pt}


\graphicspath{{figuras/}}


% Numeração das tabelas e figuras de acordo com o capítulofa
\usepackage{amsmath}
%\numberwithin{table}{chapter}
%\numberwithin{figure}{chapter}
%\numberwithin{subfigure}{}

%\renewcommand{\thesubfigure}{\thefigure(\alph{subfigure})}
%----------------------------------------
% Comandos criados pelo usuário
\newcommand{\afazer}[1]{{\color{red}{#1}}} % Para destacar uma parte a ser trabalhada
\newcommand{\ABNTbibliographyname}{REFERÊNCIAS} % Necessário para abnTeX 0.8.2

\newcommand{\figura}[5][Fonte:]{
	\begin{figure}[h!tb]
		\caption{#3.}
		\includegraphics[width=#4]{#2.png}
		\ifthenelse{\isempty{#5}}{}{%
			\\ \footnotesize{#1 \citeonline{#5}.}
		}	
		\label{fig:#2}
	\end{figure}
}

\NewDocumentEnvironment{tabela}{mm}
{
	\begin{table}[htb]
	
	{
	    %TODO: Rever
	}
		\caption{#2}
		\label{tab:#1}
	\end{table}
}

\usepackage{longtable, ltcaption}
\usepackage{trivfloat}
\usepackage{floatrow}
\usepackage{chngcntr}
%\begin{quadro}[h!tp]
%    \centering
%    \caption{Teste}
%    \begin{tabular}{|c|c|} \hline
%        Coluna 1 &amp; Coluna 2 \\ \hline
%        Conteúdo 1 &amp; Conteúdo 2 \\ \hline
%    \end{tabular}
%\end{quadro}





\captionsetup[quadro]{position=above,skip=0pt}


\newcommand{\reffig}[1]{Figura \ref{fig:#1}}
\newcommand{\refalg}[1]{Algoritmo \ref{alg:#1}}
\newcommand{\reftab}[1]{Tabela \ref{tab:#1}}
\newcommand{\refcap}[1]{\footnote{vide tópico \ref{cap:#1}.}}
\newcommand{\reftop}[1]{Seção \ref{cap:#1}}

\newcommand{\criaSigla}[2][]{
    #1 (#2) \sigla{#2}{#1}
}

\newcommand{\criarSigla}[3][]{%
	\ifthenelse{\isempty{#1}}{%
		#2 (#3)\sigla{#3}{#2}%
	}{%
		\emph{#2} (#3)\sigla{#3}{\emph{#2}}\footnote{Traduzido como: #1.}%
	}%
}


\trivfloat{quadro}
\floatstyle{plaintop} % Forçar posição da legenda para o topo
\restylefloat{quadro} % Forçar posição da legenda para o topo
\renewcommand{\listquadroname}{Lista de Quadros} % Forçar texto na Lista de Quadros



%\renewcommand{\listofquadrosname}{Lista de Quadros}

%\DeclareFloatingEnvironment[name=Equação,listname={Lista de Equações}]{equacao} % \listof...
%\renewcommand{\theequacao}{\arabic{equacao}}
%\captionsetup[equacao]{position=above,skip=0pt}

%----------------------------------------------------------------------

%----------------------------------------------------------------------
% Identificadores do trabalho
% Usados para preencher os elementos pré-textuais
\instituicao[a]{Universidade Federal de Santa Catarina} % Opcional
\departamento[a]{Departamento de Computação} %TODO departamento de engenharia da computação?
\curso[a]{Universidade Federal de Santa Catarina}
\documento[o]{Trabalho\ de\ Conclusão\ de\ Curso} % [o] para dissertação [a] para tese

\titulo{Um modelo de geladeira inteligente que leva em conta as preferências e hábitos de seus usuários}

% V1: Um modelo de geladeira aprimorado que facilite a vida dos usuários e que leva em conta suas preferências

%\subtitulo{Subtítulo} % Opcional

\setcounter{tocdepth}{2} 
\autor{Thiago Raulino Dal Pont}
\grau{Bacharel em Engenharia de Computação}
\local{Araranguá} % Opcional (Florianópolis é o padrão)
\data{05}{dezembro}{2017}
\coordenador[Coordenadora]{Prof. Dr. Eliane Pozzebon} %TODO Rever o Coordenador do curso
\orientador[Orientador]{Prof. Dr. Alexandre Leopoldo Gonçalves}
%\coorientador[Coorientador\\Universidade ...]{Prof. Dr.}

\numerodemembrosnabanca{3} % Isso decide se haverá uma folha adicional
\orientadornabanca{sim} % Se faz parte da banca definir como sim
\coorientadornabanca{não} % Se faz parte da banca definir como sim
\bancaMembroA{Primeiro membro\\Universidade ...} %Nome do presidente da banca
\bancaMembroB{Segundo membro\\Universidade ...}      % Nome do membro da Banca
\bancaMembroC{Terceiro membro\\Universidade ...}     % Nome do membro da Banca
\bancaMembroD{Quarto membro\\Universidade ...}       % Nome do membro da Banca
%\bancaMembroE{Quinto membro\\Universidade ...}       % Nome do membro da Banca
%\bancaMembroF{Sexto membro\\Universidade ...}        % Nome do membro da Banca
%\bancaMembroG{Sétimo membro\\Universidade ...}       % Nome do membro da Banca

\dedicatoria{Este trabalho é dedicado aos meus colegas de classe e aos meus queridos pais.}

\agradecimento{A todos qυе direta оυ indiretamente fizeram parte dа minha formação, о mеυ muito obrigado.}
\epigrafe{\textit{``Se A é o sucesso, então A é igual a X mais Y mais Z. O trabalho é X; Y é o lazer; e Z é manter a boca fechada.''}\\ \\Albert Einstein}


\textoResumo {
%Introdução e contextualização
Nas últimas duas décadas o conceito de Internet das Coisas vem se popularizando, propondo a introdução da tecnologia e da interconexão de objetos distintos, como lâmpadas, geladeiras, carros, etc., através de uma rede, frequentemente, a Internet. Assim, o número de aparelhos interligados tende a crescer produzindo uma grande quantidade de dados. Torna-se, então, necessário o uso de ferramentas capazes de processar e extrair informações relevantes desses dados. Entre os métodos existentes estão os sistemas de recomendação, aos quais, a partir de análises da interações de usuários, são capazes de traçar perfis de usuários e fornecer sugestões. Nos últimos anos, esses sistemas vêm sendo incorporados em diversos contextos como plataformas de \textit{streaming} e, gradativamente, outras aplicações introduzirão tais ideias no âmbito da Internet das Coisas. Entre os ambientes, nos quais é possível integrar ambos os conceitos, estão as casas inteligentes ou \textit{smart homes}, onde o uso da tecnologia abre caminho para novas formas de interação com o lar. Dispositivos comuns presentes em casas, como geladeiras, agregam, assim, novas tecnologias e funcionalidades, no entanto, os modelos existentes apenas proporcionam interações digitais, contudo não atentam aos gostos e hábitos dos usuários. Levando isso em conta, o presente trabalho propõe um modelo de geladeira capaz de monitorar as interações e traçar perfil de preferências por produtos. Isso é possível devido ao uso de etiquetas RFID, que terão o papel de identificar os diversos produtos contidos na geladeira e da aplicação de algoritmos de recomendação. Ao final, fez-se uma avaliação, a partir de um cenário criado, onde o equipamento foi capaz de realizar o monitoramento dos produtos disponíveis e de oferecer recomendações aos usuários a partir o seus perfis, além da reposição de produtos considerados essenciais. Além disso, foi possível recomendar receitas tanto a partir do conteúdo da geladeira, bem como, do perfil do usuário. A partir do exposto, conclui-se que a integração dos conceitos de Internet das Coisas e Sistemas de Recomendação pode contribuir para auxiliar usuários em suas tarefas diárias facilitando, de algum modo, o processo de tomada de decisão.}

% Ao final, o equipamento será capaz de oferecer novos produtos ao usuário de acordo com o seu perfil, além de promoções de itens relacionados e sugestões de receitas que englobem produtos contidos na geladeira.

% Ao final, o equipamento foi capaz de oferecer novos produtos ao usuário, a partir o seu perfil, além da reposição de  produtos considerados essenciais. Além disso, foi possível recomendar receitas tanto a partir do conteúdo da geladeira, bem como, do perfil de preferências do usuario.

\palavrasChave {Internet das Coisas. Sistemas de Recomendação. Geladeira Inteligente. Preferências do usuário.}

% No final vai ter 4 PARTES: 

% Introdução / cont
% materiais e métodos
% resultados alcançados
% conclusões

% TCC 1 
% Introdução / cont
% materiais e métodos
% resultados esperados


% Nas últimas duas décadas o paradigma da Internet das Coisas vem emergindo, propondo a introdução da tecnologia e da inter-conexão de objetos distintos como lâmpadas, geladeiras, carros, etc., através de uma rede, sendo esta, muitas vezes, a Internet. Além disso, com o grande número de aparelhos interligados, se produzirá altas quantidades de dados provenientes das interações com o ambiente e com os usuários. Assim, torna-se necessário o uso de ferramentas que processem e extraiam informações relevantes desses dados. Entre os métodos existentes estão os sistemas de recomendação, aos quais, a partir da análise dos registros das interações, são capazes de traçar o perfil de usuários. Nos últimos anos, esses sistemas vêm sendo aplicados em diversos contextos como plataformas de \textit{streaming}.  O presente trabalho propõe um novo modelo de interação entre geladeiras e usuários, onde o equipamento a partir dos algoritmos citados é capaz de monitorar as interação com o usuário e a traçar o seu perfil de preferências em produtos. Isso é possível graças ao uso etiquetas RFID, aos quais operam por radio frequência, que terão o papel de identificar os diversos produtos contidos na geladeira e as interações do usuário com tais produtos. A partir disso é possível traçar um perfil do usuário a partir de algoritmos de recomendação. Ao final, a geladeira será capaz de oferecer novos produtos ao usuário de acordo com o seu perfil além de promoções. É possível que o usuário selecione um conjunto de produtos essenciais, ou seja, que jamais podem faltar e a partir disso realizar compras automáticas. A partir do perfil, será possível também fazer sugestões de receitas que englobem os produtos contidos na geladeira.
\textAbstract {
In the last two decades the concept of the Internet of Things has become popular, proposing the introduction of technology and the interconnection of different objects such as lamps, refrigerators, cars, etc. through a network, often the Internet. Thus, the number of interconnected devices tends to grow producing a large amount of data. It is then necessary to use tools capable of processing and extracting relevant information from such data. Among the existing areas are the Recommender Systems based on analyzes of user interactions being able to trace user profiles and provide suggestions. In recent years, these systems have been incorporated into a variety of contexts such as streaming platforms and, gradually, other applications will introduce such ideas into the Internet of Things context. Among the environments in which both concepts can be integrated are smart houses, in which the use of technology creates new forms of interaction with the home. Common devices present in the houses such as refrigerators add, thus, technologies and new features, however, existing models only provide digital interactions, however do not address the tastes and habits of users. Taking this into account the present work proposes a model of refrigerator capable of monitoring interactions and tracing preferences profile by products. This is possible by using RFID tags which have the role of identifying the various products contained in the refrigerator and the application of recommendation algorithms. In the end, an evaluation was made, based on a created scenario, where the equipment was able to monitor the available products and offer recommendations to users from their profiles as well as the replacement of products considered essential. In addition, it was possible to recommend recipes both from the contents of the refrigerator as well as from the user profile. From the above, it is concluded that the integration of the Internet of Things and Recommendation Systems concepts can contribute to help users in their daily tasks facilitating, in some way, decision making process.}


%minha proposta



% fim



\keywords {Internet of Things. Recommender systems. Smart fridge. User preferences.}

% Ao final, o equipamento será capaz de oferecer novos produtos ao usuário de acordo com o seu perfil, além de promoções de itens relacionados e sugestões de receitas que englobem produtos contidos na geladeira.

% V1:
%In the last two decades the concept of the Internet of Things has become popular, proposing the introduction of technology and the interconnection of different objects such as lamps, refrigerators, cars, etc. through a network, often the Internet. Thus, the number of interconnected devices tends to grow and high amounts of data will be produced. It is then necessary to use tools capable of processing and extracting relevant information from this data. Among the existing methods are the recommendation systems, which, from the analysis of the records, are able to trace the users profile and provide suggestions. In recent years, these systems have been incorporated in a variety of contexts as streaming platforms and, gradually, applications will introduce these ideas into the Internet of Things. Among the environments in which both concepts can be integrated are smart houses, where the use of technology opens the way to new forms of interaction with the home. Common devices such as refrigerators add technologies and new features, however, the existing models only provide digital interactions, however do not pay attention to the tastes and habits of users. Taking this into account, the present work proposes a new refrigerator model capable of monitoring interactions and tracing preferences profile in products. This is possible thanks to the use of RFID tags, which will have the role of identifying the various products contained in the refrigerator and the application of recommendation algorithms. At the end, the equipment was able to monitor the available products and offer new products to the user, based on their profile, as well as the replacement of products considered essential. In addition, it was possible to recommend recipes from the contents of the refrigerator as well as from the user profile.


%



%----------------------------------------------------------------------
% Início do documento                                
\begin{document}

    \counterwithout{quadro}{chapter}
    \counterwithout{equation}{chapter} % OK

	%--------------------------------------------------------
	% Elementos pré-textuais
	\capa  
	%\folhaderosto[comficha] % Se nao quiser imprimir a ficha, é só não usar o parâmetro
	
	\begin{titlepage}
	\vfill
	\begin{center}
		%{\large \ABNTautordata} \\[5cm]
		\ABNTautordata \\[5cm]
		
		\tituloformat{\ABNTtitulodata} 
		\tituloformat{\subtitulodata} \\[1cm]
		
%		\vspace{1cm}
		
		\hspace{.45\textwidth} % posicionando a minipage
		\begin{minipage}{.5\textwidth}
			\begin{espacosimples}
				
				\textnormal{Trabalho de Conclusão de Curso submetido à \ABNTinstituicaodata,
				como parte dos requisitos necessários para a obtenção do Grau de
				Bacharel em Engenharia de Computação.}
				
				\textnormal{Orientador: Prof. Alexandre Leopoldo Gonçalves, Dr.}
			\end{espacosimples}
		\end{minipage}
		
		\vfill 
		\localformat \ABNTlocaldata\\
		\anodata
	
	\end{center}
	\vspace{1cm}
\end{titlepage}



	
%	\folhaaprovacao
%	\paginadedicatoria
%	\paginaagradecimento
%	\paginaepigrafe
	\paginaresumo
	\paginaabstract
	%\pretextuais % Substitui todos os elementos pre-textuais acima
	\listadefiguras % as listas dependem da necessidade do usuário
	\listadetabelas 

	\listadeabreviaturas
	
	%%%%%%%%%%%%%%%%%%%%%%%%%%%%%%%%%%%%%%%%%
    \sigla{6LoWPAN}{\textit{IPv6 Over Wireless Personal Area Network}}
    \sigla{AES128}{\textit{Advanced Encryption Standard de 128 bits}}
    \sigla{AP}{\textit{Access Point}}
    \sigla{BLE}{\textit{Bluetooth Low Energy}}
    \sigla{BR/EDR}{\textit{Basic Rate/Enhanced Data Rate}}
    \sigla{BSS}{\textit{Basic Service Set}}
    \sigla{EPC}{\textit{Electronic Product Code}}
    \sigla{FC}{Filtragem Colaborativa}
    \sigla{HTTP}{\textit{Hypertext Transfer Protocol}}
    \sigla{IDE}{\textit{Integrated Development Environment}}
    \sigla{IDF}{\textit{Inverse Document Frequency}}
    \sigla{IEEE}{\textit{Institute of Electrical and Electronics Engineers}}
    \sigla{IETF}{\textit{Internet Engineering Task Force}}
    \sigla{IPv6}{\textit{Internet Protocol Version 6}}
    \sigla{ISM}{\textit{Industrial, Scientific and Medical}}
    \sigla{JAX-WS}{\textit{Java API for XML Web Services}}
    \sigla{JEE}{\textit{Java Enterprise Edition}}
    \sigla{JSON}{\textit{JavaScript Object Notation}}
    \sigla{kNN}{\textit{k Nearest Neighbors}}
    \sigla{MLP}{\textit{Multilayer Perceptron}}
    \sigla{NDEF}{\textit{NFC Data Exchange Format}}
    \sigla{NFC}{\textit{Near Field Communication}}
    \sigla{OSI}{\textit{Open System Interconnection}}
    \sigla{PDF}{\textit{Portable Document Format}}
    \sigla{REST}{\textit{Representational State Transfer}}
    \sigla{RFID}{\textit{Radio-Frequency IDentification}}
    \sigla{RMSE}{\textit{Root Mean Square Error}}
    \sigla{RNA}{Rede Neural Artificial}
    \sigla{SPI}{\textit{Serial Peripheral Interface}}
    \sigla{SQL}{\textit{Search Query Language}}
    \sigla{SR}{Sistema de Recomendação}
    \sigla{SVD}{\textit{Singular Value Decomposition}}
    \sigla{TDM}{\textit{Time-Division Multiplexing}}
    \sigla{TF-IDF}{\textit{Term Frequency - Inverse Document Frequency}}
    \sigla{TF}{\textit{Term Frequency}}
    \sigla{URI}{\textit{Uniform Resource Identifier}}
    \sigla{URL}{\textit{Uniform Resource Locator}}
    \sigla{WPAN}{\textit{Wireless Personal Area Network}}
	%%%%%%%%%%%%%%%%%%%%%%%%%%%%%%%%%%%%%%%%%
	
	%\listadesimbolos
%	\addcontentsline{toc}{chapter}{\listquadroname}
    
  
    \listofquadros % Adiciona lista de quadros
	\sumario
	
	%--------------------------------------------------------
	%########################################################
	% Elementos textuais
	\chapter{Introdução}

Nas últimas décadas, presencia-se os acelerados avanços em ciência e tecnologia impulsionados por empresas dos mais diversos ramos e pelas constantes pesquisas nas universidades. 

Um dos avanços mais significativos é a Internet, com um grande impacto no desenvolvimento da economia global e na sociedade atual. Em duas décadas tem decorrido um grande crescimento na disponibilidade do acesso à rede. Em setembro de 2016, o número de usuários da rede mundial de computadores era de, aproximadamente, 3,75 bilhões, cerca de metade da população mundial e por volta de 92 vezes maior em relação ao ano 2000 \cite{MiniwattsMarketingGroup2016}.  Outro grande avanço se tem nos celulares, aos quais evoluíram tanto nos últimos anos que passaram de simples e grandes telefones sem fio à dispositivos menores, no entanto, com acesso a Internet, recursos avançados de áudio e vídeo e poder de processamento equiparável ao de computadores de mesa (\textit{desktops}) e/ou notebooks.

Em meio ao domínio da Internet, um novo paradigma surgiu no meio acadêmico e aos poucos ganha terreno nas grandes empresas. A sua proposta é levar a tecnologia a objetos do dia a dia, como condicionadores de ar, lâmpadas, fogões etc., e, assim, criar novas formas de interação além de funcionalidades inéditas, seguindo o exemplo dos \textit{smartphones}. 

Mencionada pela primeira vez, por Kevin Ashton, em 1999 \cite{Finep2015}, a Internet das Coisas ou IoT (em inglês, \textit{Internet of Things}) está cada vez mais próxima da realidade. Abrigará um variado ecossistema de dispositivos com capacidade de processamento, sensoriamento, conexão com demais dispositivos e, em muito casos, com a Internet, entre outros avanços. Estima-se que, em 2020, cerca de 24 bilhões de dispositivos IoT estejam conectados, implicando em cerca de quatro dispositivos por pessoa \cite{Meola2016}.   
Contudo, com o crescimento do número de itens computacionais, a quantidade de dados gerado por eles também cresce, mas de maneira exponencial \cite{Chiang2016}. A partir disso, o fluxo de dados na rede de internet se intensifica a ponto de comprometer o desempenho desta. Isso ocorre devido ao modelo atual de funcionamento da rede, ou seja, centralizado. Portanto, uma nova arquitetura se faz necessária para incorporar os dispositivos IoT à Internet tradicional. % já citado
Nesse contexto, a \textit{fog computing} ou computação de neblima, surge como potencial solução, com base na proposta de uma nova forma de organização de rede para complementar a atual. Isso é possível com base na aproximação de algumas funcionalidades centralizadas em servidores aos dispositivos que as utilizam \cite{Chiang2016}. Para tanto, um dispositivo de rede seria responsável por uma \textit{nuvem local}. Assim, os dispositivos IoT se comunicariam com esse equipamento e obteriam funcionalidades necessárias de maneira mais eficiente. A nuvem local se comunicaria diretamente com a nuvem convencional, transferindo apenas informações necessárias \cite{Syed2016}.

Os objetos inteligentes, ou \textit{smart objects}, com funcionalidades expandidas como comunicação, sensoriamento, processamento e atuação sobre o ambiente, promovem a interação entre o mundo físico (analógico) e o mundo digital \cite{Stojkoska2017}. Isso ocorre graças a sensores capazes de capturar grandezas como temperatura e luminosidade e, a partir disso, permite que aplicações tenham conhecimento do contexto do ambiente. Baseando-se nesses conceitos, algumas companhias vêm colocando no mercado novos produtos com características citadas. Como exemplo, é citável o Amazon Echo\textsuperscript{\textregistered}\footnote{https://www.amazon.com/Amazon-Echo-Bluetooth-Speaker-with-WiFi-Alexa/dp/B00X4WHP5E}, um dispositivo que opera com o serviço de assistente pessoal Alexa, e interage com pessoas em uma casa a partir de comando de voz. Outro produto destacável, é a \textit{smart lock} da empresa Nuki\textsuperscript{\textregistered}\footnote{https://nuki.io/en/shop/nuki-smart-lock/}, pelo qual é possível abrir e fechar a porta apenas com um toque no aplicativo móvel pelo \textit{smartphone} ou através de um \textit{smart watch}. Por outro lado, áreas como esportes também recebem atenção. O CARV\textsuperscript{\textregistered}\footnote{https://www.kickstarter.com/projects/333155164/carv-the-worlds-first-wearable-that-helps-you-ski}, um dispositivo vestível ou \textit{wearable}, ao qual propõe um calçado para praticantes de ski capaz de analisar em tempo real o modo de esquiar e fornecer informações detalhadas sobre.

Os \textit{smart objects} poderão, a partir da IoT, operar em conjunto e comporem os chamados \textit{smart environments}, ambientes nos quais a integração dos dispositivos agrega novas funcionalidades e formas de interação para aquele ambiente \cite{Asano2016}. Entre os ambientes inteligentes emergentes estão as \textit{smart grids}, às quais propõem a atualização do sistema elétrico atual a partir do uso da tecnologia. Uma das principais mudanças será o direcionamento do fluxo de energia e informações em dois sentidos. Como consequência, será possível consumir e fornecer energia para o sistema elétrico, bem como trocar informações sobre o estado da rede de eletricidade, o consumo entre outros avanços. Tudo isso será viável em virtude da capacidade de sensoriamento, troca de informações, controle e de tecnologia da informação e comunicação \cite{Cecilia2016}.  

Além das \textit{smart grids}, outro ambiente em expansão é a \textit{smart home}. Através dela, os moradores de uma casa podem interagir com um ambiente inteligente capaz de responder ao seus comportamentos e prover diversas funcionalidades \cite{DeSilva2012}. Isso se deve à presença de dispositivos dotados com tecnologias de sensoriamento, controle e comunicação. Além disso, a integração desse ambiente com a \textit{smart grid} pode propiciar um aumento na eficiência do consumo da casa, a partir do gerenciamento dos dispositivos conectados implicando, desse modo, menos gastos com eletricidade no final do mês (citar). %TODO: Citar

Os ambientes inteligentes podem abrigar outros menores. No caso das \textit{smart homes}, é possível subdividí-las em ambientes como a cozinha inteligente ou \textit{smart kitchen}. Nesse ambiente, usuário tem à disposição novas maneiras de interagir com os utensílios na preparação de alimentos, eletrodomésticos entre outros. A partir disso, surgem diversas oportunidades em termos de criação de produtos, como geladeiras, fogões, cafeteiras conectadas. Em relação às geladeiras inteligentes ou \textit{smart fridges}, por exemplo, viu-se um avanço nos últimos anos. Desde os anos 2000 vem-se pensando em como conectar refrigeradores à Internet, com a LG\textsuperscript{\textregistered}\footnote{http://www.lg.com} entre as primeiras companhias a implementar o conceito de dispositivos conectados à Internet. Em pesquisas recentes, propõe-se adicionar outros recursos como monitoramento dos produtos no interior e seus respectivos prazos de validades entre outros \cite{Hachani2016}. 

As interações das pessoas com os ambientes citados gerará uma grande quantidade de dados. Um aproveitamento eficiente desses dados pode ampliar as aplicações da IoT. Uma das diversas formas para colocar essa ideia em prática são os sistemas de recomendação. Com base nas preferências indicadas pelo usuário ou no seu comportamento, esses sistemas buscam selecionar e fornecer informações relevantes \cite{Filho2008}. Sistemas de recomendação são divididos em duas classes: filtragem colaborativa em que as recomendações são feitas com base na similaridade das preferências de usuários com os demais, isto é, em gostos de outros usuários em produtos, serviços etc., que o usuário não conhece, mas tem alta probabilidade de interesse. Outra categoria é a baseada em conteúdo, na qual os conteúdos apresentados ao usuário são baseados nas suas próprias preferências, ou seja, em itens semelhantes aos de interesse. Por fim, existe a abordagem mista, em que ambas as categorias citadas são mescladas, aproveitamento, desse modo, as melhores características de cada uma \cite{Thomas2016}.
Além disso, as aplicações de sistemas de recomendação são adotadas nos mais diversos campos, entre eles, plataformas de streaming de filmes e séries, sites de vendas online entre outros. Ainda assim, os sistemas de recomendação vêm recebendo novas propostas de uso como sistemas aptos a propor pontos de carga para condutores de carros elétricos \cite{Ferreira2011} e notícias personalizadas \cite{Yeung2010}. 

% TODO: Escrever sobre HCI

\section{Problemática}
% Explicação:
% A problemática deve enfatizar os desafios identificados na literatura, ou seja, durante a leitura dos artigos alguns desafios foram identificados nas áreas de pesquisa do trabalho (Internet das Coisas, Smart Things ou correlatos, Sistemas de Recomendação) e devem ser apresentados. A partir da dissertação sobre tais desafios se finaliza a seção com uma pergunta de pesquisa, geralmente iniciando pela palavra "Como".


Com o avanço da Internet das Coisas, a rede de internet necessitará de algumas adaptações para suportar o grande número de dispositivos conectados e alto volume de dados que serão transmitidos a todo momento. Além disso, como cada dispositivo estará vinculado à rede, ele estará, portanto, ao alcance de ciber criminosos. No entanto, a segurança em IoT não evoluiu suficientemente para garantir preservação desses dispositivos.

Em relação as geladeiras inteligentes, um dos problemas no monitoramento e registro de itens em uma geladeira é o modo com o qual é realizado. Muitas propostas fazem uso de dispositivos que operam por ondas eletromagnéticas. No entanto, os alimentos que contém água além de estruturas metálicas podem interferir no desempenho das leituras a depender da tecnologia utilizada \cite{Periyasamy2015} \cite{Qing2007}. Por outro lado, os métodos que utilizam processamento de imagem têm êxito em monitoramento de alimentos naturais como verduras e frutas \cite{Shweta2017}, mas não há propostas que englobe processamento de imagem para produtos embalados, como leite, enlatados entre outros na literatura.

Apesar do grande número de estudos a cerca da IoT, as tecnologias e aplicações propostas não têm dado devida atenção aos aspectos de usabilidade e experiência do usuário, focando mais no ponto de vista técnico \cite{Koreshoff2013}. É necessário portanto, incluir no projeto de aplicações o conceito de interface humano e tecnologia.

% Concluir com a pergunta de pesquisa
Desse modo, tem-se como pergunta de pesquisa: ``Como melhorar o modelo de geladeira comum que capture a interação com os usuários permitindo a estes experiências que facilitem o seu dia a dia na cozinha?''.

% V1: Como projetar uma geladeira inteligente com experiência única aos seus usuários e lhes proporcionar uma melhor qualidade de vida?
% V2: Como projetar uma geladeira inteligente que capture a interação com os usuários permitindo a estes experiências que facilitem o seu dia a dia na cozinha.
% V3: Como melhorar o modelo de geladeira atual que capture a interação com os usuários permitindo a estes experiências que facilitem o seu dia a dia na cozinha?.

\section{Objetivos}
Esta seção apresenta o objetivo geral e os objetivos específicos do trabalho.

\subsection{Geral}

% V1: Desenvolver uma geladeira capaz de monitorar os produtos contidos nela e prover recomendações de receitas com base nos padrões de consumo dos produtos.

% Desenvolver uma geladeira que facilite o dia a dia dos usuários a partir da análise das interações entre os mesmos.

Desenvolver um modelo de geladeira inteligente que facilite o dia a dia dos usuários a partir da análise das interações entre os mesmos.
    
% V4 Melhorar o modelo de geladeira inteligente para que facilite o dia a dia dos usuários a partir da análise das interações entre os mesmos.

\subsection{Específicos}


\begin{itemize} \parskip -1pt
	%\item Levantamento do estado da arte com relação a Internet das Coisas e Sistemas de Recomendação
	\item Elaborar e desenvolver um projeto de leitura e monitoramento dos produtos contidos na geladeira.
	\item Implementar um sistema de análise das interações e recomendação de serviços.
	\item Elaborar um cenário que permita a avaliação do modelo proposto.
	\item Avaliar e discutir os resultados obtidos a partir do sistema proposto.

%V1
% 	\item Levantar o estado da arte com relação a Internet das Coisas e Sistemas de Recomendação
% 	\item Propor um projeto de leitura e monitoramento dos produtos contidos na geladeira.
% 	\item Propor um sistema de análise das interações e recomendação de serviços.
% 	\item Elaborar um cenário que permita a avaliação do sistema proposto.
% 	\item Avaliar e discutir os resultados obtidos a partir do sistema proposto.

\end{itemize}

\section{Justificativa e Motivação}

A Internet tem evoluído nas últimas décadas e impactado na economia mundial e no dia a dia das pessoas. Nos seus primeiros anos de existência, tinha como principal função o uso militar e acadêmico, como foco em troca de informações \cite{Leiner2012}. Contudo, anos mais tarde foi aberta para uso da população em geral permitindo, desse modo, que pessoas comuns tivessem acesso a rede. Hoje, cerca de metade da população mundial usam a Internet frequentemente \cite{MiniwattsMarketingGroup2016}. 

No cenário atual, um novo grupo está sendo conectado na Internet: "as coisas". Cria-se um novo paradigma, a Internet das Coisas, onde a rede não será mais utilizada apenas da maneira tradicional, como em um computador de mesa ou \textit{smartphone} entre outros, mas por dispositivos que possuem acesso a redes e capacidades como sensoriamento, atuação e comunicação com outros dispositivos. Muitos dos dispositivos serão versões conectadas dos objetos presentes no dia a dia, como televisão, fogão, geladeira, lâmpada e porta. Todos esses equipamentos, operando em conjunto com a rede, criarão um ecossistema de objetos com funcionalidades inéditas. 

Com a IoT estima-se que até 2020, cerca de 24 bilhões de dispositivos estejam conectados \cite{Meola2016}, garantindo espaço para inovação em produtos e serviços. Além disso, espera-se que a Internet das Coisas se torne um grande atrativo para o mercado. Estima-se que em 2025 sejam gerados em torno de 13 trilhões de dólares (citar).

A sociedade se beneficia com o desenvolvimento da Internet das Coisas. As soluções geradas considerando esse conceito trarão novas formas de interação entre as pessoas e os objetos que as cercam no cotidiano. Ambientes como casa, indústria e sala de aula terão a disposição novas formas de interação com esses a partir da tecnologia. 

Tratando-se de uma casa inteligente, chamada também de \textit{smart home}, os moradores tem a disposição conforto e comodidade em virtude dos objetos conectados presentes nela, entre eles a geladeira. Presente em grande parte dos lares, o refrigerador tem um papel fundamental na vida dos moradores. Os alimentos contidos nela devem ser bem conservados para o consumo. No entanto, produtos são esquecidos no seu interior e, por vezes, passam do prazo de validade. Além disso, seria muito cômodo aos usuários, dessa geladeira, estando em um supermercado e soubessem quais itens estão faltando ou vencidos, evitando assim compras em demasia ou modéstia. Apesar da facilidade na visita ao supermercado com uma lista em tempo real dos produtos necessários, seria ainda mais cômodo se o refrigerador, automaticamente realiza-se compras de itens essenciais como leite ou carne e o supermercado entregasse as compras em casa. Ainda que tais funcionalidades não sejam comuns, existem propostas de geladeiras inteligentes que as implementam. Contudo, não há abordagens em que se leve em conta os interesses do usuários como preferências em certos alimentos, horários em que o consumo é mais comum entre outros. Por isso, acredite-se que entender o comportamento do usuário com IoT pode melhorar o dia a dia deste.

Portanto, este trabalho trará como contribuição a melhora do modelo atual de geladeira inteligente que além das funcionalidades citadas, leve em conta os interesses e padrões de consumo de seus usuários.


% V1: ...como contribuição a proposição e implementação de uma geladeira inteligente que além das funcionalidades citadas, leve em conta os interesses e padrões de consumo de seus usuários. 



\section{Procedimentos metodológicos}

% O que é metodologia
O método é a ordem que se deve impor aos diferentes procedimentos necessários para atingir um certo objetivo \cite{Cervo2007}. Por meio destes procedimentos, a pesquisa caracteriza-se como uma atividade voltada para a investigação de problemas teóricos ou práticos \cite{Matias-Pereira2012}.

% Tipo de pesquisa
Este trabalho pode ser caracterizado, quanto à sua finalidade, como uma pesquisa aplicada, visto que, conforme \citeonline{Matias-Pereira2012}, ``os conhecimentos adquiridos são utilizados para aplicação prática e voltados para a solução de problemas concretos da vida moderna''. Quanto ao objeto, o projeto é descrito como uma pesquisa bibliográfica, já que é necessária para o levantamento do estado da arte do tema, fundamentação teórica e definição da contribuição do trabalho \cite{Matias-Pereira2012}. De acordo com a modalidade, a pesquisa se identifica como uma pesquisa tecnológica, onde será criado um artefato tecnológico, sendo este um protótipo de geladeira capaz de reconhecer as interações do usuário, realizar compras automáticas além de recomendações de outros produtos e receitas com base nas preferências do usuário. 

A metodologia de desenvolvimento deste trabalho é dividida em 8 etapas, das quais, a ordem cronológica é apresentado na \reffig{cap1_metodologia-etapas}.

\begin{figure}[hb]
    
    \caption{Fluxo das Etapas do Trabalho}
    \includegraphics[width=\textwidth]{cap1_metodologia-etapas}
    \label{fig:cap1_metodologia-etapas}
    
    Fonte: Autor
\end{figure}

A seguir, a sequência de etapas demonstradas anteriormente são especificadas em detalhes.

\begin{itemize}\parskip -1pt
    \item Etapa 1: Análise e definição do escopo do trabalho.
	\item Etapa 2: Levantamento bibliográfico a cerca de Internet das Coisas e Sistemas de Recomendação;
	\item Etapa 3: Elaborar e desenvolver um projeto de leitura e monitoramento dos produtos contidos na geladeira.
	\item Etapa 4: Implementar um sistema de análise das interações e recomendação de serviços.
	\item Etapa 5: Desenvolvimento de um protótipo funcional que integre as Etapas 3 e 4.
	\item Etapa 6: Criação de um cenário de testes para avaliar o protótipo.
	\item Etapa 7: Avaliação e discussão dos resultados obtidos no cenário proposto.
	\item Etapa 8: Escrita do Trabalho de Conclusão de Curso.
\end{itemize}



% Colocar uma figura do fluxo
% Contextualizar para caracterizar
% O que vai usar 
% Teste
% Dizer como vai avaliar
% TODO: Descrição das etapas

\section{Organização do trabalho} % Não é necessário, no TCC 1

Este trabalho é divido em seis capítulos. O \textbf{Capítulo 1} apresentada uma introdução do estado da arte das áreas envolvidas bem como a problemática do trabalho e os objetivos gerais e específicos.

O \textbf{Capítulo 2} trata da Internet das Coisas, no qual é feita uma revisão da arquitetura para organização dos diversos componentes, das tecnologias existentes que possibilitam o desenvolvimento de novos dispositivos e, por fim, uma descrição sobre alguns dos ambientes nos quais a Internet das Coisas será incorporada nos próximos anos.

O \textbf{Capítulo 3} é sobre Sistemas de Recomendação

O \textbf{Capítulo 4} é sobre o Sistema Proposto

O \textbf{Capítulo 5} é sobre o cronograma para a displina de TCC II.

% O capítulo 5 é sobre a Avaliação do Sistema Proposto
% %	- Descrever um cenário
% %	- Avaliar e discutir o cenário a partir do sistema proposto
	
% O capítulo 6 é sobre Considerações Finais

% TODO: criar o arquivo para o organograma
	\chapter{Internet das Coisas}
\label{cap:internet_of_things}
%%%%%%%%   PARTE 1   %%%%%%%%
%\section{Conceito}

% TODO: INTRODUÇÃO

% Contextualização

% REVER
A tecnologia, com o passar dos anos, está cada vez mais presente nas indústrias, lares, comércios etc., ao mesmo tempo tornando-se indispensável para todas essas entidades. No entanto, nos últimos anos um novo paradigma está emergindo: a Internet das Coisas. A partir dela, a Internet vai deixar de existir como é vista hoje tornando, assim, onipresente (citar).

% O que é 

O conceito de Internet das Coisas (IoT) está relacionado à interconexão de objetos distintos através de uma rede, sendo esta, muitas vezes, a Internet. Desse modo, elementos do mundo real, que antes funcionavam de maneira independente ao meio aos quais estavam inseridos, são capazes de interagir com outros objetos à sua volta e, assim, trocar informações que possam ser relevantes permitindo a agregação de novas funcionalidades.  Além disso, a IoT abre espaço para interação 
entre o mundo físico e o digital a partir de dispositivos capazes de capturar dados físicos no meio em que estão tais como, temperatura, distância etc., representá-los digitalmente e trasmití-los para outros dispositivos.

	
O termo ``Internet das Coisas'' foi citado pela primeira vez por Kevin Ashton, diretor executivo da AutoIDCentre do MIT, em 1999 enquanto realizava uma apresentação para promover a ideia do uso de Identificadores de Radio Frequência (RFID) na etiquetagem de produtos. O uso da tecnologia beneficiaria a logística da cadeia de produção \cite{Finep2015}. Apesar de o termo IoT ter sido usado apenas em 1999, aplicações práticas da ideia já existiam anos antes. Um exemplo disso, é a torradeira que podia ser ligada e desligada via internet criada em 1990 \cite{Suresh2014}.

% TODO: Projeções de grandes companhias (número de dispositivos, expansão)

A Internet das Coisas está em grande expansão. Estima-se que em 2020 cerca de 24 bilhões de dispositivos IoT estejam conectados, implicando em cerca de quatro dispositivos por pessoa. Para tanto, em torno de 6 trilhões de dólares serão investidos em desenvolvimento de tecnologias de hardware e software, como aplicações, segurança e dispositivos de hardware. Apesar da grande quantia investida, o setor é visto como promissor. Estima-se será gerado em torno de 13 trilhões de dólares em 2025 \cite{Meola2016}.



\section{Arquitetura}
% Escrever sobre as diversas arquiteturas existentes, características de cada uma, as melhores aplicações de cada uma etc.

Segundo \citeonline{Al-Fuqaha2015}, para que seja possível a Internet das Coisas, em meio ao grande número de objetos, são necessários seis elementos básicos: identificação de cada dispositivo na rede, sensoriamento sobre o ambiente, comunicação entre os dispositivos e a Internet, computação, serviços e semântica. As arquiteturas para Internet das Coisas, devem levar em conta esses pontos.

Ao longo dos últimos anos, alguns modelos de arquiteturas foram propostos no âmbito da Internet das Coisas. \citeonline{Al-Fuqaha2015} em seu trabalho, mostra algumas das arquiteturas mais comuns para IoT, entre elas, as que estão mostradas na Figura \ref{fig:cap2_arquiteturas}.
\begin{figure}[htb]
	\caption{Arquiteturas para IoT}
	\subfigure[3 camadas]{\label{fig:cap2_arq_3layer}\includegraphics[width=0.25\textwidth]{cap2_arq_3layers_readaptation.png}}
	\subfigure[5 camadas]{\label{fig:cap2_arq_5layer}\includegraphics[width=0.25\textwidth]{cap2_arq_5layers_readaptation.png}}
	
	\footnotesize{Fonte: Adaptado de \citeonline{Al-Fuqaha2015}}
	\label{fig:cap2_arquiteturas}
\end{figure}

A arquitetura em três camadas pode ser definida como a base para dispositivos relacionados à IoT e envolve a  percepção, a rede e a aplicação. A primeira camada compreende os objetos inteligentes dotados de sensoriamento e atuação sobre o ambiente, já a segunda se refere a infraestrutura de comunicação responsável por conectar os dispositivos entre si e com a Internet e, por fim, a camada de aplicação provê serviços, processamento e tomada de decisão.  

% Teoricamente
Neste trabalho, será feito uso da arquitetura de cinco camadas, sendo que a primeira camada é responsável por comportar os objetos dotados de sensoriamento e/ou atuação, pelos quais, interagem diretamente com o ambiente. Já a segunda camada, é responsável por transmitir de forma segura os dados provenientes da camada anterior. A camada de gerenciamento de serviços atua como intermediária entre requisitores de serviços e provedores, além de processar os dados da camada inferior e entregar devidos serviços de acordo com o necessário. O quarto nível interage diretamente com os usuários a partir do fornecimento de serviços como exibição de informações de sensoriamento, além do controle sobre atuadores. Já a última camada é responsável por gerenciar todas as atividades e serviços da IoT, além de possibilitar a tomada de decisão e análise de \textit{big data} a partir dos dados provenientes da camada de aplicação \cite{Al-Fuqaha2015}.

%As demais arquiteturas seguem o mesmo raciocínio da arquitetura de três camadas, ou seja, um conjunto de camadas de sensoriamento, outro de infraestrutura de rede e um último de aplicação, variando em aspectos internos de cada camada de acordo com o objetivo final \cite{Ray2016}.

% Falar sobre a inexistência de um padrão global para a IoT, mas mostrar as instituições que estão tentando tornar a IoT padronizada em realidade.

\section{Tecnologias}



% Escrever sobre as tecnologias
% Introduzir comos as tecnologias influem e são necessárias na IoT.
% Contextualizar um pouco a evolução da tecnologia
% =========================================================

Para compreender melhor a funcionamento e a evolução da Internet das Coisas, é importante ter conhecimento e entendimento das tecnologias que dão base a ela. As principais tecnologias necessárias estão imersas nas camadas das arquiteturas expostas na seção anterior, ou seja, sensoriamento e atuação, redes e aplicação. A seguir, uma breve introdução em algumas das tecnologias é exposta.

\subsection{Bluetooth}

% O que é
% MELHORAR e referenciar
O Bluetooth é uma especificação de rede  \criarSigla[Rede Sem Fio de Área Pessoal]{Wireless Personal Area Network}{WPAN}, ou seja, rede sem-fio pessoal, sendo descrito e especificado pelo padrão definido pela  \criarSigla[Instituto de Engenheiros Eletricistas e Eletrônicos]{Institute of Electrical and Electronics Engineers}{IEEE}, o IEEE 802.15.1. O Bluetooth foi criado na década de 90 com o objetivo de unir tecnologias distintas, tais como computadores, celulares entre outros a partir de uma padronização de comunicação sem fio entre os dispositivos \cite{Kardach2008}. 
%\begin{figure}[H]
%	\centering
%	\caption[ABC]{Interconexão entre diversas classes de dispositivos}
%	\label{key}
%	\includegraphics[width=0.7\textwidth]{img/bluetooth-ecosystem}
%    \newline Fonte: Repositório de imagens Pixabay %(link: 
%    %https://pixabay.com/en/bluetooth-connectivity-wireless-1690677/)
%\end{figure}
Uma das principais características dessa tecnologia \textit{wireless} é o curto alcance de transmissão variando de centímetros até alguns metros \cite{Huang2007}. 

%A tecnologia vem sendo usada ao longo dos últimos anos em diversas aplicações como transferência de arquivos entre dispositivos, transmissão de áudio entre smartphones e fones sem fio, dispositivos capazes de determinar contexto, como os beacons, entre outros.

% Topologias
No IEEE 802.15.1 há suporte para criação de redes \textit{ad-hoc}, aos quais, é desnecessário uma infraestrutura de rede para conexão dos dispositivos. A partir disso é possível criar redes chamadas \textit{picorredes}, nas quais os dispositivos são organizados em até oito associados, sendo um deles um mestre, ao qual coordena as operações, e os demais escravos \cite{BluetoothSIG2017}.

% Características
A tecnologia Bluetooth opera na \criarSigla[Industrial Científico e Médico]{Industrial Scientific and Medical}{ISM} de 2.4 GHz de uso livre em modo \criarSigla[Multiplexação por Divisão de Tempo]{Time-Division Multiplexing}{TDM} com um delta de $625\mu$s, proporcionando uma taxa de transmissão máxima em torno de 2 Mb/s, podendo variar de acordo com o dispositivo e a categoria de tecnologia de Bluetooth utilizada \cite{BluetoothSIG2017}.

% Forma de conexão.

\subsubsection{Categorias}

Segundo \citeauthor{BluetoothSIG2017}, o Bluetooth pode ser categorizado em:

\begin{enumerate}[label=(\Alph*)]
    
    \item {\criarSigla[Taxa Básica / Taxa de Dados Aprimorada]{Basic Rate/Enhanced Data Rate}{BR/EDR}}
    %2.0 a 2.1
    
    % REVER
    Esta é a subdivisão mais popularizada do Bluetooth presente nas versões $2.0$ e $2.1$, onde as principais características são alta velocidade de transmissão alta, baixo alcance e necessidade de conexão através de pareamento, onde os dispositivos devem confirmar a conexão. A partir disso, há um transmissão contínua de dados. Uma desvantagem é o consumo de energia considerável para o funcionamento do Bluetooth, já que há uma conexão contínua e uma taxa de transmissão que mantêm o dispositivo ativo por um longo período ininterrupto.
    % taxa de dados
    A taxa de transmissão gira em torno de 2Mb/s.
     
    
    \item \criarSigla[Bluetooth de Baixo Consumo Energético]{Bluetooth Low Energy}{BLE}
    
    % 4.0, 4.1, 4.2
    
    % O que é
    O BLE é a mais recente categoria do Bluetooth incorporada na versão 4.0, em 2011, além de ser a menos comum \cite{LinkLabs2015}.
    % Foco
    BLE está centrado no baixo consumo de energia para permitir que certos dispositivos não precisem recarregar ou trocar suas fontes de energia, geralemente baterias, por longos períodos, que podem chegar a anos. 
    % Pareamento
    Para uma conexão para transmissão de dados, ao contrário do BR/EDR, não é necessário um pareamento para realizá-la, além disso esta tem curta duração, na ordem de milissegundos.
    %Taxa e alcance
    Ademais, a taxa de dados é baixa e o alcance alto. A baixa taxa de dados decorre do modo de funcionamento dos dispositivos BLE, aos quais, enviam dados em rajadas, ou seja, de tempos em tempos dados são transmitidos em forma de \textit{broadcast} e os dispositivos que estiverem conectados receberão esses dados. Nos intervalos de tempo em que o dispositivo não transmite, ele ``dorme'', isto é, entra em modo de consumo mínimo a fim de poupar energia.
    
    %Aplicação
    A aplicação prática dessas características está na IoT através de \textit{beacons} e \textit{wearables}, aos quais incorporam o BLE. Os beacons foram introduzidos pela \textit{Apple} em conjunto com o iOS 7, com o nome de \textit{iBeacon}, que permitia aos aplicativos possuíssem senso de localização \cite{Apple2014}. Com esses dispositivos é possível aprimorar a experiência do usuário em estabelecimentos como museus, supermercados, shoppings, estádios, através da identificação de contexto, na qual, com base na detecção de um beacon e da aproximação ou afastamento deste, uma aplicação móvel em um smartphone de um usuário pode exibir conteúdos, indicar promoções entre outros relacionados aquele dispositivo BLE.
    %
    %\begin{figure}[H]
    %	\centering
    %	\caption[ABC]{{\noindent Alguns exemplares de beacons}}
    %	\label{fig:beacons}
    %	\includegraphics[width=0.5\textwidth]{img/beacons.jpg}
    %	\newline Fonte: Shine Solutions %(link: 
    %%https://shinesolutions.com/2014/02/17/the-beacon-experiments-low-energy-bluetooth-devices-in-action/
    %\end{figure}
    
    \item{Dual-mode}
    
    Esta categoria se refere a dispositivos, como \textit{smartphones} que precisam se conectar tanto com dispositivos BR/EDR, como fones de ouvido, e BLE, como \textit{beacons} \cite{BluetoothSIG2017a}.

\end{enumerate}


\subsubsection{Bluetooth 5.0}

A versão 5.0 do Bluetooth foi lançada em dezembro de 2016 e trás consigo aprimoramentos em desempenho e segurança, garantindo duas vezes mais velocidade, quatro vezes mais alcance, oito vezes mais taxa de dados e, por fim, maior coexistência \cite{BluetoothSIG2017b}. 

Com a nova versão, veio a flexibilidade para construção de soluções baseadas em necessidade. Parâmetros como alcance, velocidade e segurança podem ser regulados para diversos objetivos a depender das aplicações \cite{BluetoothSIG2017b}.

Algumas atualizações contribuem para a redução de interferência com outras tecnologias sem fio, dessa forma, proporciona melhor coexistência entre dispositivos Bluetooth e de outras tecnologias, dentro do cenário emergente da IoT \cite{BluetoothSIG2017b}.


% ==================================================================================================
\subsection{RFID}

O protocolo de \criarSigla[Identificação por Rádio Frequência]{Radio-Frequency IDentification}{RFID} é uma tecnologia de identificação automática, entre diversas outras como código de barras, cartão inteligente e procedimentos biométricos, no entanto se distingue pelo modo de funcionamento, ou seja, por ondas eletromagnéticas. Além disso, o RFID se destaca em relação às outras tecnologias no que se refere às influências externas no seu funcionamento, como sujeira, posição de leitura. Desse modo, não é necessário nem limpar ou reposicionar o dispositivo RFID para efetuar a leitura \cite{Finkenzeller2010}. 

% LEITOR E TRANSPONDER
No RFID, os dados são transmitidos através de ondas de rádio entre dois dispositivos: \textit{transponder} ou \textit{tag} e \textit{leitor}. O transponder é localizado no objeto identificado, um produto, equipamento etc., e nele são mantidos os dados de identificação. Já o leitor é responsável pela leitura e escrita dos dados presentes no transponder \cite{Finkenzeller2010}.
% FUNCIONAMENTO
Para a transmissão dos dados entre os dois dispositivos o leitor emite ondas de rádio na tag. Ao receber o estímulo, a tag responde com os dados contidos nela. Além disso, existem tags que utilizam a energia do campo eletromagnético gerado pelo leitor para seu funcionamento, sendos estas chamadas de \textit{passivas}. Existem, também, aquelas que possuem uma fonte própria de energia e por isso são denominadas \textit{ativas} \cite{Finkenzeller2010}.

%Um exemplo de tag ativa é mostrada na Figura Y.

%\begin{figure}[H]
%	\centering
%	\caption[ABC]{{\noindent Exemplo tag RFID ativa}}
%	\label{fig:active-rfid}
%	\includegraphics[width=0.3\textwidth]{img/active-tag.jpg}
%	\newline Fonte: Vnsky %(link: 
%	%http://www.vnsky.com/parts/2673180/ACTIVE-TAG-REFERENCE-DESIGN-KIT.html
%\end{figure}

%Uma tag passiva é mostrada na Figura Y.

%\begin{figure}[H]
%	\centering
%	\caption[ABC]{{\noindent Exemplo tag RFID passiva}}
%	\label{fig:passive-rfid}
%	\includegraphics[width=0.3\textwidth]{img/passive.png}
%	\newline Fonte: Eletrosome %(link: 
%	%https://electrosome.com/rfid-radio-frequency-identification/
%\end{figure}

% TIPOS (alimentação)

% ATIVO
% PASSIVO

% FREQ. DE OPERAÇÃO
Uma das características mais importantes dos dispositivos RFID é a frequência de operação.
%já que ela influi no distância máxima de operação. Tal fator é determinado pelo leitor. 
Os dispositivos são classificados, de acordo com esse parâmetro, em três grupos:

\begin{itemize} \parskip -1pt
	\item \textbf{LF (Baixa Frequência):} Entre 30kHz à 300kHz
	\item \textbf{HF (Alta Frequência):} Entre 3MHz à 30MHz
	\item \textbf{UHF (Ultra Alta Frequência)}: Entre 300MHz a 3GHz.
\end{itemize}

É possível distinguir pelo alcance:

\begin{itemize} \parskip -1pt
	\item \textbf{{Long-range} ou longo alcance:} maior que um metro
	\item \textbf{{Remote-coupling} ou ligação remota:} até um metro
	\item \textbf{{Close-coupling} ou ligação próxima:} até um centímetro
\end{itemize}

%TODO: Escrever sobre os fatores que influenciam no funcionamento, alcance etc.

% ==================================================================================================
\subsection{NFC}
% O_QUE É
O \criarSigla[Comunicação por Campo de Proximidade]{Near Field Communication}{NFC} é um sistema de comunicação sem fio derivado do RFID. Ele permite transações simples e seguras entre dois dispositivos a partir da curta distância de operação, em torno de 4cm, e do funcionamento baseado em aproximação dos objetos em questão \cite{NFCForum2016}. 
Assim, é possível realizar leituras de tags e obter conteúdos de acordo com a aplicação, transferir dados entre smartphones entre outras funcionalidades.
% COMPATIBILIDADE
Outra vantagem do NFC é a compatibilidade com a infraestrutura de cartões sem contato existentes permitindo usar um único dispositivo em tecnologias diferentes. Desse modo, é possível interagir com tags RFID, por exemplo.


% FUNCIONAMENTO
Como o RFID, o NFC funciona através de ondas eletromagnéticas, mas com uma taxa de transmissão máxima de 424 kbps \cite{NFCForum2016}. Além disso, pode operar em dois modos de comunicação: ativo e passivo  \cite{Igoe2014}. Assim como no RFID, é possível que os dispositivos NFC que contenham os dados usem a energia do leitor para transmitir seus dados, no modo passivo, ou usem uma fonte própria para tal procedimento, no modo ativo.


% MODOS DE OPERAÇÃO
Outra característica importante no NFC são os modos de operação. De acordo com \citeonline{NFCForum2016} existem três modos:

\begin{itemize} \parskip -1pt
	\item \textbf{Leitor/Escritor de tag}: Tem por objetivo ligar o mundo físico ao digital através 
	de aplicações que leem e/ou escrevem em tags para obter dados e, assim, fornecer conteúdo ao 
	usuário relacionado à tag lida. Um exemplo é um smartphone ao ler uma tag NFC de um cartaz na 
	rua.
	\item \textbf{Peer to Peer}: Visa conectar dispositivos por aproximação física e permite transferência de dados. Um exemplo é o Android Beam\textsuperscript{\textregistered}\footnote{https://www.android.com/intl/pt-BR\_br/} que permite troca de arquivos entre smartphones com o 
	sistema operacional móvel da Google.
	\item \textbf{Emulação de cartão}: Conecta o dispositivo do usuário em uma infraestrutura 
	possibilitando a simulação de um cartão, além da realização de transações financeiras e 
	identificação no sistema de transporte a partir da aproximação do dispositivo a um leitor 
	específico.
\end{itemize}

% CATEGORIAS
Há quatros tipos de tags definidas \cite{NFCForum2016a}, sendo que todos operam no modo Leitor/Escritor descrito anteriorente : 

\begin{itemize} \parskip -1pt
	\item \textbf{Tipo 1}: 96 bytes de memória disponível e expansível para 2kiB. Usuário pode 
	configurá-la para somente leitura.
	\item \textbf{Tipo 2}: 48 bytes de memória disponível e expansível para 2kiB. Usuário pode 
	configurá-la para somente leitura.
	\item \textbf{Tipo 3}: Baseado no padrão industrial japonês e conhecido como FeliCa. Pode ser 
	configuradas para leitura/escrita ou somente leitura na fabricação. A memória disponível varia, 
	mas com um limite teórico de 1MiB.
	\item \textbf{Tipo 4}: A memória disponível varia estando acima de 35 kiB por serviço. É 
	possível ser configurada para leitura/escrita ou somente leitura.
\end{itemize}

O NFC possui um padrão com o qual dispositivos devem estar formatados, o \criarSigla[Formato de Troca de Dados por NFC]{NFC Data Exchange Format}{NDEF} (\textit{NFC Data Exchange Format}) um formato comum de comunicação \cite{Igoe2014}. Desse 
modo, os dados armazenados em tags devem estar gravados nesse formato. A partir do NDEF é possível armazenar e trocar documentos binários como Mime, que incluem imagens, arquivos \criarSigla[Formato de Documento Portável]{Portable Document Format}{PDF} entre outros, 
\criarSigla[Localização Uniforme de Recursos]{Uniform Resource Locator}{URL}, texto simples entre outros.


% ==================================================================================================
\subsection{Zigbee}

% WHAT IS IT
O Zigbee é um protocolo padrão de comunicação de baixa potência para redes sem-fio \textit{mesh}, ao qual permite a diversos dispositivo trabalharem em conjunto \cite{Faludi2011}. Além disso, é descrito como um conjunto de camadas implementadas sobre o IEEE 802.15.4 \cite{Faludi2011}, ao qual especifica a camada física (PHY) e o controle de acesso ao meio (MAC) para redes sem-fio de baixa potência \cite{IEEE2011}.

As camadas do Zigbee, de acordo com \citeonline{Faludi2011}, fazem:

\begin{itemize} \parskip -4pt
	\item \textbf{Roteamento:} Tabelas de roteamento que definem como um nó envia dados até um 
	destino.
	\item \textbf{Rede Adhoc:} Criação automática de rede.
	\item \textbf{\textit{Self healing mesh}:} Descobe se nós se perderam da rede e a 
	reconfigura para garantir uma rota para os dispositivos conectados ao nó faltantes.
\end{itemize}

O Zigbee opera na faixa não licenciada ISM, de 2,4GHz, o que permite sua expansão global e, assim, ser capaz de operar em qualquer local do mundo. Além disso, especifica que os nós das redes criadas possam assumir papeis específicos. Cada nó deve assumir uma das categorias a seguir \cite{Faludi2011}:

\begin{itemize} \parskip -4pt
	\item \textbf{Coordenador}: Responsável por criar a rede, distribuir endereços, manter a rede segura e em funcionamento entre outras funções. Por fim, cada rede tem um e apenas um coordenador.
	\item \textbf{Roteador}: Tem capacidade de unir redes existentes, enviar e receber informações e rotear 	informações, atuando como um intermediário entre dispositivos que, por estarem muito distantes entre si, não podem se comunicar diretamente. É permitido às redes terem múltiplos roteadores, podendo também não possuírem nenhum, no entanto no caso de existirem, cada roteador deve estar conectado a um coordenador ou outro roteador.
	\item \textbf{Dispositivo final}: É um tipo de nó capaz de se unir a redes e de enviar e receber 
	informações da rede. Além disso, podem se desligar de tempos em tempos para poupar energia. 
	Caso mensagens para um dispositivo final desligado sejam detectadas, o nó responsável por ele, 
	podendo ser um coordenador ou roteador, armazena as mensagens até que o nó desperte.
\end{itemize}

Há diversas topologias suportadas, nas quais, englobam os três tipos de nós e suas possíveis 
maneiras de organização \cite{Faludi2011}:

\begin{itemize} \parskip -4pt
	\item \textbf{Par a par}: Uma rede formada apenas por dois nós, sendo um deles, obrigatoriamente, um 
	coordenador e nó restante podendo ser um roteador ou dispositivo final.
	\item \textbf{Estrela}: Nessa topologia, o coordenador se situa no centro da rede e os demais nós, 
	roteadores ou dispositivos finais, conectados apenas a ele, formando uma rede no formato de 
	estrela.
	\item \textbf{Mesh}: Os dispositivos finais circundam os demais nós roteadores e coordenador. O 
	coordenador e roteadores atuam como intermediários, roteando mensagens para dispositivos 
	finais, outros roteadores ou para o coordenador. Apesar da nova função do coordenador, este 
	permanece no controle e gerenciamento da rede.
	\item \textbf{Cluster tree}: Nessa topologia, cada roteador é responsável por um conjunto de 
	dispositivos finais. As mensagens vindas desses dispositivos devem ser encaminhadas 
	primeiramente para seu roteador responsável para então ser encaminhada ao destino na rede.
\end{itemize}

% TODO: Colocar ilustração das topologias

O Zigbee define três maneiras de identificação de nodos, que podem utilizadas em uma aplicação para 
diferenciar os nós.

\begin{itemize} \parskip -4pt
	\item \textbf{64 bits}: Único e permanente para cada rádio fabricado.
	\item \textbf{16 bits}: Dinamicamente configurado pelo coordenador ao entrar em uma rede. É único apenas 
	dentro do contexto da rede.
	\item \textbf{Node Id}: Pequena cadeia de texto. Não é possível garantir sua unicidade em nenhum 
	contexto, apesar disso, é mais amigável aos olhos humanos.
\end{itemize}

%\subsubsection{Criação de uma rede}

%Cada rede de sensores deve possuir um identificador chamado endereço PAN (Personal Area Network). 
%Além disso, cada nó deve ter o mesmo PAN configurado e o mesmo canal de comunicação, que é escolhido de acordo com a disponibilidade pelo coordenador. 

%Para que uma mensagem chegue a um destino, é necessário que o nó emissor tenha conhecimento do endereço do nó destinatário do pacote.

%\subsubsection{XBee}

%O XBee é um dispositivo fabricado pela Digi. Existem cerca de 
%30 combinações de hardware, protocolos de \textit{firmware}, potência de transmissão e antenas.

%Apesar das diversas combinações, há duas versões básicas do XBee: Série 1 e Série 2.

%Os nós XBee Série 1 proveem comunicações ponto a ponto, bem como uma implementação proprietária de 
%rede \textit{mesh}. Já os nós Série 2 permitem diversas derivações de padrões de redes mesh Zigbee.

% IMAGEM XBEE S1 e XBEE S2
% ==================================================================================================

\subsection{Wi-Fi}
% O que é
Wi-Fi, é uma das diversas classes de \criarSigla[Rede Local Sem Fio]{Wireless Local Area Network}{WLAN} normatizado pelo padrão IEEE 802.11, no qual foca nas camadas física e de enlace do modelo \criarSigla[Conexão de Sistema Aberta]{Open System Interconnection}{OSI} \cite{Gast2005}. 
Além disso, há padrões específicos para o Wi-fi, como o 802.11a, 802.11b e 802.11g,  além da possibilidade de unir alguns de padrões para formar outros híbridos, como o 802.11a/g e 802.11a/b/g \cite{Kurose2012}. 

Apesar da distinção, os padrões citados compartilham diversas características, como o protocolo de acesso ao meio, estrutura de quadros da camada de enlace, habilidade de reduzir a taxa de transmissão a fim de alcançar distâncias maiores. A principal diferença é vinculada à camada física.

Os padrões a, b e g são regulamentados de acordo com a Tabela \ref{tab:ieee80211abg-phy}, sendo que pode variar em diversos países.

\begin{table}[hbt]
    \caption{Resumo dos protocolos 802.11}
    \label{tab:ieee80211abg-phy}
    \centering
    \begin{tabular}{@{}ccc@{}}
        \toprule
        \textbf{Padrão} & \textbf{Faixa de Frequências} & \textbf{Taxa de dados} \\ \midrule
        802.11b         & 2,4 - 2,485 GHz               & até 11 Mbps            \\
        802.11a         & 5,1 - 5,8 GHz                 & até 54 Mbps            \\
        802.11g         & 2,4 - 2,485 GHz               & até 54 Mbps            \\ \bottomrule 
        \end{tabular}
        
    Adaptado de: \citeonline{Kurose2012}
\end{table}



Além dos três protocolos citados há outros mais recentes ou que estão em fase de criação. O protocolo 802.11n, por exemplo, criado em 2012, faz uso de múltiplas antenas e, além disso, permite atingir uma taxa de transmissão de centenas de megabits por segundo. \cite{Kurose2012}.

A arquitetura básica do IEEE 802.11, exposta na \reffig{cap2_arquitetura-ieee80211},  é formada por \criarSigla[Conjunto Básicos de Serviço]{Basic Service Set}{BSS} onde cada um é composto de um Ponto de Acesso (AP), um dispositivo para unir os conjuntos, sendo esse um roteador ou \textit{switch}, responsável por ligar cada BSS à Internet e, por fim, os dispositivos que desejam se conectar a rede.

\figura{cap2_arquitetura-ieee80211}{Arquitetura básica para o IEEE802.11}{0.6\textwidth}{Kurose2012}


% Taxonomia
    % Adhoc
    % 
O IEEE 802.11 suporta dois tipos de interconexão de dispositivos: \textit{ad-hoc} e ponto de acesso com dispositivo. No primeiro caso, é possível interconectar dispositivos, como notebooks, sem a necessidade de uma infraestrutura de rede, no entanto com a impossibilidade de conexão com a Internet, apenas com os dispositivos na rede. Assim, é possível efetuar transferência de arquivos de maneira rápida e sem cabos. Além de redes ad-hoc, suporta conexões entre um ponto de acesso e um dispositivo para conexão com a Internet. Em diversos casos, o AP e o roteador estarão incorporados em um mesmo dispositivo \cite{Kurose2012}.

% Mobilidade

%O protocolo permite a mobilidade de um dispositivo entre pontos de acessos e ainda assim manter a conexão com estes sem a necessidade de se obter um novo endereço IP. Isso será possível se ambos os pontos de acessos fizerem parte de uma mesma subrede.

% Tipos de escaneamento numa mesma subrede
Existem algumas funcionalidades mais avançadas no IEEE 802.11. Uma delas se refere à possibilidade de se adaptar a taxa de dados a partir da escolha da técnica de modulação da camada física de acordo com as características do canal. Há também a possibilidade de reduzir o consumo de energia a partir fazendo com que o nó em modo \textit{sleep} por determinados períodos de tempo o que pode gerar uma economia de até 99\% \cite{Kurose2012}.

No contexto da Internet das Coisas, o Wi-fi é fundamental na inserção de novos dispositivos conectados à Internet. O seu uso, consideravelmente difundido, facilitará o alcance de novos dispositivos à rede e propiciará o crescimento destes sem a necessidade da expansão da infraestrutura a cada novo objeto conectado \cite{Suresh2014}.

\subsection{Outros}
% Z-wave
Entre as tecnologias utilizadas em Smart Homes está o Z-Wave, um protocolo sem fio focado em automação residencial e comercial de pequeno porte, criado pelo ZenSys e hoje representado pela Z-Wave Alliance  \cite{Gomez2010}. O protocolo foi desenvolvido especificamente para controle, monitoramento e verificação de estado. Em relação a aspectos técnicos, o Z-Wave opera na faixa de frequências em torno de 1GHz, o que evita interferências com outras tecnologias como Bluetooth e Wi-Fi que operam em 2,4GHz, em geral. Entre as principais vantagens do Z-Wave está a interoperabilidade que este proporciona entre os diversos produtos desenvolvidos com a tecnologia, além da segurança obtida a partir do uso de criptografia AES128 \cite{Z-WaveAlliance2015}.
\sigla{AES}{Advanced Encryption Standard}

% 6LowPan
Por outro lado, há a tendência de dispositivos IoT conectados à Internet usarem o protocolo de endereçamento IPv6 para serem identificados na rede. No entanto, algumas das aplicações terão limitações como fonte de energia e capacidade de transferências de dados limitadas. Como proposta de solução, tem-se o 6LowPan, definido no RFC 6282 pela Internet Engineering Task Force (IETF) focado em dispositivos com restrições de consumo de energia \cite{Olsson2014}. A principal característica desse protocolo é a redução da transmissão de dados a partir da compressão dos cabeçalhos do IPv6. Com isso, o 6LowPan é capaz de reduzir a sobrecarga de pacote para dois bytes \cite{Al-Fuqaha2015}. As redes 6LoWPAN são conectadas às redes IPv6 a partir de roteadores de borda, capazes de trocar dados entre os dispositivos dentro da rede 6LoWPAN e a Internet e  entre os dispositivos da rede além de ser responsáveis por manter a rede em funcionamento. Outra característica importante nesse padrão é que este torna possível implementar o IPv6 em redes IEEE 802.15.4 \cite{Olsson2014}.

\section{Ambientes inteligentes}

A Internet das Coisas trará tecnologia para ambientes e, assim, aprimorará as funcionalidades existentes além de trazer novas, proporcionando eficiência e qualidade. A seguir, alguns ambientes em destaque são expostos e como a Internet das Coisas atua nestes casos. 

\subsection{Smart grid}

Na rede elétrica tradicional, a inteligência é concentrada nas unidades geradoras de energia e parcialmente nos distribuidores e o fluxo de energia é unidirecional, seguindo apenas um caminho entre a geração e o consumo. No entanto, com o aumento do uso de fontes próprias geração de energia como placas fotovoltaicas em residências, além do crescimento do consumo tem-se a necessidade de adaptação do modelo de rede elétrica existente. As \textit{smart grids} surgiram, então, como novo modelo de geração e distribuição de energia elétrica, onde o fluxo de energia passa a ser bidirecional e o uso de tecnologias de medições de consumo permitem prever demandas, otimizar a distribuição e aprimorar a eficiência e a confiabilidade do sistema elétrico \cite{Cecilia2016}. Portanto, a smart grid tenderá a ser um avanço do modelo atual com o uso de tecnologias de comunicação, sistema eletrônicos de potência avançados e de medição o que possibilitará monitoramento em tempo real, permitindo o fluxo de informações em ambos os sentidos entre consumidores e unidades geradores além da garantia de otimização do fluxo de energia.

%Empresas como a \textit{Texas Instruments} (TI) têm investido em soluções para smart grid que proporcionam segurança, eficiência e inteligência. A TI oferece soluções para monitoramento da rede através de medidores de eletricidade, gás e calor, além de tecnologias para comunicação como o \textit{Power Line Communications} entre outras \cite{TexasInstruments2017}.

\subsection{Smart home}

% TODO: Conceitos
O conceito de \textit{smart home} ou casa inteligente propõe um novo modelo para um ambiente domiciliar no qual a implementação e o uso da tecnologia abrem espaço para novas formas de interação com o lar, além de proporcionar mais comodidade e um melhor gerenciamento dos equipamentos presentes. Isso será possível graças ao uso de sensores e atuadores no ambiente, nos eletrodomésticos e utensílios. E, para interconectar todos os dispositivos, fará-se o uso das tecnologias de rede existentes, como o Zigbee e a Internet.
Além disso, a interconexão dos dispositivos em uma casa inteligente proporcionará funcionalidades inéditas de interação de acordo com as ações do morador, como entrar e sair de um cômodo da casa. Nesse caso, seria possível implementar um sistema que apagasse e acendesse a lâmpada conforme os sensores de presença indicarem a ausência do indivíduo e a hora do dia \cite{DeSilva2012}.

Entre as apostas para as casas inteligentes está o aumento da eficiência do consumo energético. O uso da tecnologia por meio de medidores de energia, tomadas e aparelhos inteligentes permitirá o monitoramento e controle do consumo dos dispositivos da casa. Com base nisso, é viável a otimização do consumo de cada equipamento controlando-o para ativá-lo somente quando necessário e, assim, evitar desperdícios, além de previsão da demanda de energia para cada momento do dia \cite{Stojkoska2017}.

As casas inteligentes necessitam que os dispositivos sejam organizados hierarquicamente para garantir o bom desempenho dos componentes do sistema. \citeonline{Stojkoska2017} propõe um modelo, mostrado na \reffig{cap2_smart-home_framework-stojkoska} para organização de casa inteligente que integra \textit{smart grid} e envolve cinco componentes principais: a casa inteligente em si, a nuvem, unidade geradora de energia, aplicações de terceiros e interfaces de usuário. Nesse contexto, a casa inteligente contém redes de sensores sem fio que adquirem dados do ambiente e enviam esses dados para um \textit{home hub}, um ponto central capaz de se conectar à uma rede externa. Já a unidade geradora de energia é responsável por, além da geração e fornecimento de energia, trocar informações sobre custo da energia, consumo atual e futuro da casa entre outros. A nuvem é responsável por armazenar todos os dados provenientes de sensores e outros dispositivos da casa e por comportar uma infraestrutura de processamento. A partir dos dados existentes na nuvem é possível que as aplicações de terceiros entreguem soluções web ou aplicações mobile para os usuários. Para obter acesso às soluções, o último componente é necessário, a interface de usuário. Nesse ponto, o usuário tem em mãos a capacidade de monitorar em tempo real os gastos de energia dos equipamentos de sua casa, bem como controlá-los de acordo com a sua necessidade e desejo.

\figura{cap2_smart-home_framework-stojkoska}{Modelo para casas inteligentes}{\textwidth}{Stojkoska2017}


%(escanoladores, reguladores, balanceadores de carga).
%(stoj...; mostrar a figura do cara)

% Segundo \cite{DeSilva2012}, as casas inteligentes podem ser divididas em três categorias principais. 
% A primeira delas atenta-se em detectar e reconhecer ações dos moradores e seu estado de saúde e, baseando-se nelas, prover serviços com foco no bem-estar e na saúde dos residentes. Nesse categoria estão incluídos cuidado de idosos, da saúde e de crianças. 
% A segunda categoria se refere à captura e multimídia de eventos ocorridos no dia a dia.
% A terceira tem como foco na segurança. Uma sistema de alerta pode ser ativado, no momento em que uma invasão, sequestro ou desastre natural ocorrer, a partir captura de imagens, sensoriamento entre outros.


Uma casa inteligente permite que os moradores tenham maior independência no seu dia a dia, especialmente em caso de pessoas idosas, com dificuldade de locomoção, além daquelas com deficiências físicas e visuais.
No seu artigo, \citeonline{DeSilva2012} propõe um sistema que usa imagens de uma casa, capaz de detectar o dia a dia de uma pessoa idosa. É possível também identificar possíveis quedas e avisar o responsável ou ao atendimento médico. Apesar, da aplicação citada possuir um enfoque no morador, não há muitas propostas para casa inteligente com o foco nos usuários, no entanto, existem muitas que focam nos aspectos técnicos, como dispositivos, arquiteturas entre outros \cite{Wilson2015}.

% TODO: Tecnologias
    % Tecnologias de 
    % Eclipse SmartHome

Em relação às tecnologias para desenvolvimento de aplicações e dispositivos para casas inteligentes, tem-se diversos sistemas sendo implementados. Entre eles está o projeto \textit{Eclipse SmartHome} \textsuperscript{\texttrademark}
\footnote{http://www.eclipse.org/smarthome}, ao qual disponibiliza aos desenvolvedores um \textit{framework} para desenvolvimento de soluções casas inteligentes e ambientes assistidos destinadas a usuários finais. 
    
% TODO: Desafios

A empresa \textit{Amazon} \textsuperscript{\textregistered}
\footnote{https://www.amazon.com/} oferece o \textit{Amazon Echo}, um dispositivo que dispõe de diversas funcionalidades multimídia, como reprodução de músicas através de controle por voz, além de oferecer informações como previsão do tempo, notícias, tráfego entre outros através do \textit{Alexa Voice Service} (Serviço de Voz Alexa). É capaz de controlar a luz, tomadas e termostatos além de ser compatível com produtos de empresas, como SAMSUNG\textsuperscript{\textregistered}
\footnote{http://www.samsung.com/br/}, Philips{\textregistered}
\footnote{http://www.philips.com.br/} entre outras, com foco em \textit{smart homes} \cite{amazon2017}.

%\subsubsection{Smart Kitchen}

Seguindo a lógica das casas inteligentes, as \textit{Smart Kitchens} ou cozinhas inteligentes promovem o aprimoramento dos dispositivos da cozinha com a inserção da tecnologia. A partir disso, utensílios como panelas, talheres entre outros poderão fazer uso de tecnologia para entregar novas funcionalidades \cite{Staender2012}. Por exemplo, no caso das panelas, é possível colocar sensores de temperatura e câmera para determinar a temperatura atual e o estado atual do alimento que está sendo cozido. A partir disso, o sistema computacional presente na panela, processará os dados e fará uma comunicação com o fogão para ajustar a intensidade do fogo caso ainda não esteja pronto, ou simplesmente, desligar o fogo, caso o esteja. 

\subsection{Outros}
%\subsection{Smart factory}
% Quarta revolução industrial,

% Apoio a logística

%\subsection{Smart City}

Entre os ambientes inteligentes em expansão está a cidade inteligente ou \textit{smart city}. Apesar de não ter uma definição conceitual clara \cite{Cocchia2014}, no contexto da Internet das Coisas, a ideia principal por trás desse ambiente é trazer qualidade de vida aos cidadãos, crescimento sustentável e melhor uso de recursos públicos, aos quais são possíveis graças ao uso da IoT com foco no ambiente urbano. A Internet das Coisas, nesse contexto, permite uma melhor gerenciamento, otimização dos serviços públicos como transporte, iluminação, vigilância e manutenção de áreas públicas entre outros  \cite{Zanella2014}. 

%
Já no ambiente industrial, a Internet das Coisas em conjunto com sistemas interconectados promoverá a Indústria 4.0 também chamada quarta revolução industrial . Além disso, os sistemas cyber-físicos, aos quais podem ser definidos como sistemas que integram processos físicos, computacionais, de comunicação e de rede, integrados com a Indústria 4.0, poderão ser definidos como a \textit{smart factory} ou fábrica inteligente \cite{Lee2015}. Por outro lado, a Indústria 4.0 é vista como possível solução para problemas atuais em indústrias como poluição, consumo de combustíveis fósseis entre outros. Para tanto o uso de tecnologias emergente para implementar IoT e serviços onde processos de engenharia e de negócios estarão integrados possibitando produção com qualidade e baixo custo e que seja flexível, eficiente e sustentável \cite{Hussain2016}.


% \subsection{Wearables}

% \textit{Wearables} são dispositivos computacionais vestíveis, ou seja, são itens que uma pessoa pode usar no dia a dia como, roupas, relógios, óculos, sapatos entre outros e, ainda obter novas funcionalidades, graças à presença da tecnologia nesses dispositivos. Isso é possível devido a sensores aos quais medem sinais vitais do corpo humano e dados do ambiente, a depender da aplicação, além de pequenos equipamentos de hardware responsáveis por ler os dados dos sensores e comandos do usuário e assim tomar as ações necessárias de acordo com a situação.

% Os \textit{wearables} têm aplicações nas mais diversas áreas, partindo das áreas da saúde e esportes até lazer e trabalho. 

% Algumas empresas têm investido na área de dispositivos vestíveis. Um exemplo é a Microsoft que vem desenvolvendo o \textit{Holo Lens}, um dispositivo que se assemelha a um óculos, com foco em realidade aumentada. A tecnologia pode ser usada na concepção e design de produtos, educação, astronomia entre outros.

% Outra tecnologia em ascensão é o \textit{smart watch}, relógio digital conectado à internet, dotado de aplicativos entre outras funções que vão além de mostrar as horas.

% Na área de esportes e saúde, existem as \textit{smart bands}, pulseiras capazes de monitorar sinais, como batimentos além das calorias eliminadas durante um exercício, distância percorrida entre outros.



%\subsection{Turismo}
% Exemplo de turismo do artigo
%Uma aplicação proposta consiste em um sistema de informação turístico com intuito de expandir a experiência dos visitantes nos diversos pontos turísticos no Japão. Isso é possível graças ao uso de beacons, sensores sem fio descritos na seção anterior, e realidade aumentada, viabilizada por uma aplicação móvel \cite{SHIBATA2016}. 


\section{Desafios}

% Contextualizar
Apesar dos avanços constantes, a área de Internet das Coisas terá de superar alguns desafios para que possa se expandir sem prejudicar o desempenho das aplicações e a experiência dos usuários. Esses desafios, segundo \citeonline{Hussain2016}, são a heterogeneidade dos dispositivos, a interoperabilidade, a escalabilidade, segurança, privacidade e Qualidade de Serviço (QoS).

% Heteregeineidade
O primeiro dos desafios, a heterogeneidade, se refere as diferenças de hardware e software dos dispositivos bem como seu propósito, como objetivos, plataforma de hardware, modos de interação, entre outros. %Um solução simples à primeira vista, é fazer com que se use os mesmos padrões de hardware e software, no entanto não há uma única tecnologia de comunicação, por exemplo, que supra as necessidades de todas aplicações existentes.

% Interoparabilidade
A Interoperabilidade diz respeito à capacidade de dispositivos, que usam diferentes tecnologias, poder interagir. Nesses casos é necessário um dispositivo que atue como mediador ao qual tem acesso a ambas tecnologias para que a troca de dados ocorra.

% Escabilidade
Outro grande desafio é a escalabilidade. As aplicações de IoT devem suportar o crescimento do número de dispositivos conectados, usuários, aplicações entre outros sem qualquer comprometimento da Qualidade de Serviço (QoS). Esse aumento em dispositivos e afins, deve ser refletida nos recursos que sustentam a IoT. 
% Em meio ao grande número de dispositivos, comunicação e interligação são desafios da IoT.

% Segurança e privacidade
A segurança e privacidade, da mesma forma, requerem atenção. A restrição de recursos torna difícil proteger a informação, ainda assim é necessário garantir transações seguras e o não comprometimento dos dados do usuários. No entanto, os métodos tradicionais não podem ser utilizados, pois diferentes padrões estão envolvidos. Por outro lado, IoT implica em conexão global, ou seja, implica que qualquer indivíduo pode acessar externamente. 
Portanto, há a necessidade de novos mecanismo que garanta a segurança e privacidade e é importante que os desenvolvedores garantam esses requisitos em suas aplicações. 


% Quality of Service
A Qualidade dos Serviços em IoT vem sendo estudado com frequência. Entre os principais desafios estão disponibilidade, confiabilidade, mobilidade, desempenho, escalabilidade entre outros. No entanto, nem toda a aplicação exige que os pontos citados sejam atendidos integralmente \cite{Hussain2016}. 
	\chapter{Sistemas de recomendação}
\label{cap:sistemas_de_recomendacao}

% Introdução

% Falar sobre as mudança das coisas ligadas por "links" para por dados.

A web tem proporcionado diversas formas de interação, seja entre usuários ou entre sistemas e como resultado, tem-se uma grande quantidade de dados gerada. Os usuários que lidam com esses dados brutos certamente terão dificuldades em assimilar alguma informação útil e de seu interesse. Se faz necessário, portanto, um modo de processar tal montante de dados e extrair informações úteis. 
% Contextualizar com a Amazon, Netflix entre outros

% Os Sistemas de Recomendação surgem como potencial solução para esse problema. Um Sistema de Recomendação (SR) a partir de dados coletados sobre seus usuários na forma de preferências em certos conjuntos de itens, ou seja, produtos, eventos, ações entre outros e outras fontes de informação, provê aos usuários previsões e recomendações de itens de acordo c.

Sistema de Recomendação (SR) surge então como possível solução, com base na análise do perfil de cada usuário e, a partir deste, fornecer recomendações de itens que possam lhe interessar. Os itens recomendados pode ser filmes, livros, receitas, páginas web entre outros \cite{Bobadilla_2013}.

\section{Histórico}

A web primordial ou Web 1.0 era estática, na qual a única perspectiva de interação era o consumo de conteúdo, ou seja, apenas leitura deste se fazia possível. Era, portanto, amplamente utilizada por organizações na divulgação de seus produtos e serviços \cite{Aghaei2012}. 

% Falar sobre a web 2.0, e interação do usuário.
A Web 2.0, por outro lado, agregou dinamicidade à Internet, desde a viabilização da interação entre usuários e possibilidade de inserção de conteúdo na rede por parte desses, a partir de blogs e redes sociais, por exemplo \cite{Nath2014}.

% Falar da web 3.0
Com a Web 3.0, se deu a transformação da Web de Documentos, presente nas versões anteriores, para a Web de Dados, onde os diversos conteúdos deixam de se relacionar por links e passam a ser por dados e, assim, podem ser utilizados não apenas por usuários, mas por máquinas.
%, às quais, usam a rede para acessar recursos externos entre outros.
 
A web de documentos consiste em objetos, como websites, e as ligações entre eles, isto é, os \textit{links}, com foco no consumo de conteúdo por pessoas. A web de dados, entretanto, torna legível o conteúdo da web para as máquinas, priorizando estas em detrimento dos humanos a partir da representação e organização das dados formato organizado como o \textit{Resource Description Framework} (RDF) \cite{Aghaei2012}. Acrescentou-se, dessa forma, a possibilidade da comunicação entre dispositivos, além do aprimoramento do gerenciamento dos dados.

Já se discute a cerca da Web 4.0, apesar de, conceitualmente, não bem definida. Propõe-se que, no futuro, haveria uma simbiose entre os computadores e as pessoas e, como consequência, seria possível a construção de interfaces mais poderosas, entre elas, as controladas pela mente \cite{Aghaei2012}.

A ideia de fazer uso de todo o volume de dados gerados por muitos usuários, com a Web 2.0, para auxiliar na procura por conteúdos mais úteis e interessantes, já existia desde a década de 1990 \cite{Jannach2010}.

%    PARC Tapestry system
O primeiro sistema ao qual continha a ideia de recomendação de conteúdo, foi o PARC Tapestry System, ao qual introduziu o conceito de filtragem colaborativa. Ademais, era um sistema experimental de e-mail, ao qual objetivava categorizar o grande volume de mensagens eletrônicas recebidas em categorias de acordo com o interesse do usuário \cite{Goldberg1992}.
%    GroupLens - notícias
Alguns anos mais tarde, o \textit{GoodNews} foi desenvolvimento com o foco em notícias, onde cada artigo era avaliado de acordo com a média de avaliações dos usuários e os melhores eram recomendados. Dessa forma, o sistema não levava em conta gostos individuais e eliminava, assim, a necessidade de armazenamento de dados de usuários.
% Ringo system - filtragem colaborativa para músicas
Outro sistema desenvolvido, o Ringo, provia recomendações aos seus usuários a cerca de músicas. Inicialmente o utilizador fornecia avaliações  de aproximadamente 125 artistas e, de acordo com as respostas era feito uma avaliação do perfil.  A aplicação, então, passava a recomendar novos artistas e álbuns que o usuário poderia gostar \cite{Resnick1994}.

% Comercialização (...)
% Pesquisa (quais as pesquisar mais recentes
% Hoje (no que é usado, com qual finalidade);
% Definir item

Caracterizar
Falar sobre os objetivos: previsão de avaliações, recomendação de itens.

Falar sobre avaliações explícitas e implícitas

% Introduzir as abordagens e o que as diferencia

\section{Filtragem Colaborativa (FC)}
    
    Em diversas ocasiões do cotidiano as pessoas requisitam opiniões de outras a cerca de certos produtos ou serviços, seja filmes, restaurantes, equipamentos eletrônicos entre outros. A opinião do indivíduo, então, influencia na escolha do outro e o ajuda a tomar uma decisão sobre o problema. Por outro lado, um amigo recomenda a outro que assista um filme de ação em cartaz nos cinemas, sabendo que ainda não assistiu-o e que gosta de filmes do gênero. Esse indivíduo que recebeu a sugestão certamente levará em conta o conselho, assistirá o filme, provavelmente gostará dele e recomendará a um outro amigo que também não viu o filme.    
    Com base nesse contexto de recomendações entre indivíduos, tem-se o conceito de Filtragem Colaborativa (FC), no qual, a partir de um perfil de preferências de um certo indivíduo, obtido através de seu histórico, em conjunto com as opiniões de outros usuários semelhantes a ele, prevê quais os itens que tem a maior possibilidade de gostar ou de se interessar \cite{Jannach2010}.

    % Explicação do conceito: qual a ideia, objetivos, vantagens e desvantagens.
        % Características
        % Não precisa das características específicas dos itens.

    % Não precisa das características específicas dos itens.
    % processamento custoso
        % Escalabilidade comprometida
    % maior precisão

    %"Se usuários compartilham interesses, eles terão gostos parecidos no futuro".
    %Pegar recomendações apenas dos melhores match's.
    
    Os sistemas baseados em filtragem colaborativa se utilizam da matriz de avaliações de itens pelos usuários, como mostrado na Tabela \ref{tab:matriz-av-item-user}. Como saída, gera-se previsões de avaliações que usuários aplicariam para itens por eles não avaliados ou um conjunto de melhores itens para recomendação. Portanto, a partir do conjunto de dados da matriz é possível fazer uso de algoritmos que levem em conta as avaliações de todo o conjunto de usuários, para então definir qual é a avaliação plausível para um determinado item ou quais itens recomendar \cite{Bobadilla_2013}.
    
    \begin{table}[htb]
        
        \caption{Matriz de avaliações de itens por usuários}
        \label{tab:matriz-av-item-user}
        \begin{tabular}{@{}lcccccc@{}}
        \toprule
                  & Item 1 & Item 2 & Item 3 & Item 4 & Item 5 & Item 6 \\ \midrule
        Jack      & 5      & 4      & 2      & ?      & 5      & 1      \\
        Will      & 3      & 5      & 3      & 5      & 1      & 1      \\
        Elizabeth & 5      & 3      & 3      & 4      & 2      & 4      \\
        Hector    & 3      & 5      & 4      & 5      & 4      & 4      \\ \bottomrule
        \end{tabular}
        
        \footnotesize{Fonte: Autores}
    \end{table}
    
    Segundo \citeonline{Ricci2010}, os métodos de filtragem colaborativa podem ser agrupados em duas classes: os baseados em memória ou em vizinhança e os métodos baseados em modelo. A principal diferença entre as abordagens está no modo de uso da matriz de avaliações na geração de recomendações. 
    
    % Não precisa das características específicas dos itens.
    
    % Ideia
    % para cada fórmula, um exemplo
    % ex: quando mostrar o pearson, calcular a similaridade entre dois usuários.
        
    \subsection{Baseado em memória}
        % processamento custoso
        % Escalabilidade comprometida
        % maior precisão
        
        % Usam diretamente os dados da matriz
        Métodos de recomendação baseados em memória operam diretamente sobre a base de dados de avaliações de itens pelos usuários para geração de recomendações. Além disso, as recomendações estarão sempre atualizadas devido ao uso das mais recentes avaliações recebidas dos usuários \cite{Bobadilla_2013}. 
        
        % Rever
        %No entanto, para grandes bases de dados onde existem milhares de itens a serem avaliados e milhares de usuários, os métodos baseados em memória são lentos em termos de processamento, dificultando o seu uso em recomendações geradas em tempo real \cite{Mustafa2017}.
        
        Em geral, os sistemas produzem recomendações com base no conceito de vizinho mais próximos, onde o objetivo é encontrar semelhanças entre usuários ou entre itens com suporte nas diversas avaliações adquiridas pela base de dados \cite{Mustafa2017}. Por exemplo, se um usuário tem preferência em certos filmes de ficção científica e existe um outro que também tem algum gosto pelo gênero, ambos poderão ser classificados como vizinhos próximos. Em casos como esse, o grau de semelhança é obtido a partir de cálculos de similaridade.
        
        \subsubsection{Medidas de Similaridade} \label{sssec:similaridade}
        
        % Definição
        A similaridade entre usuários, itens, etc., podem ser obtidas a partir da matriz de avaliações dos usuários em relação ao itens. Os diversos algoritmos operam sobre pares de linhas ou colunas da tabela para encontrar um valor numérico que caracterize o grau de afinidade entre os objetos da comparação (citar). Além disso, uma abordagem geométrica pode ser utilizada para aprimorar a observação do comportamento desses algoritmos \cite{Jones1987}.
        
        % \subsubsection{Correlação de Pearson}
        O grau de semelhança entre dois usuários pode ser mensurado a partir de sua correlação, na qual estima a relação linear entre ambos. Dentre os diversos métodos, a correlação de \textit{Pearson} avalia vetores de mesma dimensão \cite{Ricci2010}. Considerando, um conjunto de produtos $P=\{p_1, p_2, ..., p_n\}$, um conjunto de usuários $U = \{u_1, u_2, ..., u_m\}$ e uma matriz de avaliações $R=\{r_{1,1}, r_{1,2}, ..., r_{n, m}\}$ desses produtos em função dos usuários, com dimensão  $n\times m$, tem-se a Equação \ref{eq:correlacao-pearson} que descreve a correlação de Pearson entre dois usuários $a$ e $b$.
             
        \begin{equation}
             % sim(a, b) =  \frac{\sum\limits_{i=1}^{n}(a_i-\bar{a})(b_i-\bar{b})}{\sqrt{\sum\limits_{k=1}^{n}(%a_k-\bar{a})^2}*\sqrt{\sum\limits_{j=1}^{n}(b_j-\bar{b})^2}} \label{eq:correlacao-pearson}
             sim(a, b) = \frac{\sum_{p\in P}(r_{a, p}-\bar{r_a})(r_{b, p}-\bar{r_b})}{\sqrt{\sum_{p\in P}(r_{a, p}-\bar{r_a})^2}\sqrt{\sum_{p\in P}(r_{b, p}-\bar{r_b})^2}}\label{eq:correlacao-pearson}
        \end{equation}
        
         Observa-se, inicialmente, a subtração de cada posição pelo valor médio das avaliações, o que reduz o efeito negativo, no cálculo, das notas de um determinado usuários que, em sua maioria, são ou altas, ou baixas. Além disso, tem-se um produto interno como numerador e a multiplicação dos comprimentos dos vetores de avaliação de cada usuário como denominador. Como possíveis resultados, a correção de Pearson gera valores entre $-1$ a $1$, onde o primeiro indica correlação negativa perfeita, ou seja, usuários com preferências opostas, e o segundo demonstra correlação positiva perfeita, implicando em gostos equivalentes entre os usuários \cite{Jannach2010}.
        
        %Pearson: É melhor entre outras técnicas para comparar usuários
        
        Como exemplo, considerando a Tabela \ref{tab:matriz-av-item-user}, deseja-se encontrar a similaridade entre os usuários Will e Elizabeth a partir de suas respectivas avaliações para os seis (6) itens e a correlação de Pearson. Fazendo uso da Equação \ref{eq:correlacao-pearson}, tem-se o seguinte cálculo:
        
        \begin{eqnarray}
            sim(W, E) &=& \text{\footnotesize   $\frac{(3-3)(5-3,5)+(5-3)(3-3,5)+...}{\sqrt{(3-3)^2+(5-3)^2+...}\sqrt{(5-3,5)^2+(3-3,5)^2+...}} $} \nonumber \\
            sim(W, E) &=& 0,21 \nonumber
        \end{eqnarray}
    
        Os usuários Will e Elizabeth têm portanto uma similaridade medida pela correlação de Pearson de $0,21$ indicando que ambos têm alguma semelhança entre suas preferências.
    
        %##########    Cosseno    ###################
        Em relação a similaridade de itens, o método do \textit{cosseno} é considerado o padrão. Como base para o seu cálculo, se faz uso das avaliações dadas por todos os usuários a cada item, ou seja, as colunas da matriz de avaliação são utilizadas. No entanto, o método do cosseno não leva em conta o perfil de cada usuário ao considerar apenas a avaliação dada ao item em questão\cite{Jannach2010}.
        
        A Equação \ref{eq:sim-cosseno} define o cálculo de similaridade pelo método do cosseno onde, calcula-se o produto interno entre os vetores de avaliações dos itens, como numerador e a multiplicação dos comprimentos de cada um como denominador.         
        
        %cosseno: É melhor que Pearson para comparar itens.
        %Jones
        \begin{equation} 
            sim(a, b) = \frac{\sum_{u\in U}(r_{u, a})(r_{u, b})}{\sqrt{\sum_{u\in U}(r_{u, a})^2}\sqrt{\sum_{u\in U}(r_{u, b})^2}} \label{eq:sim-cosseno}
        \end{equation}    
    
        Em outras palavras, para cada usuário $u$ pertencente ao conjunto de usuário $U$, é calculada a multiplicação das suas avaliações para cada item e somada aos demais resultados da operação e, por fim, calcula-se o módulo de cada vetor. Um outra interpretação para o cálculo é considerar como sendo o produto interno entre os vetores de avaliação normalizados \cite{Jones1987}. Assim, com a divisão pelo comprimento, os possíveis resultados permanecem entre $0$ e $1$ \cite{Jannach2010}, sendo que geometricamente, esses resultados podem ser avaliados como ângulos. As avaliações de um item podem ser consideradas componentes de um vetor que representa o objeto. Inserindo, então, os vetores num plano será formado um ângulo $\theta$ no qual, no contexto das avaliações de itens é obtido partir do arco cosseno do resultado da similaridade. Quanto mais próximo, esse ângulo estiver de zero grau ($0^o$), maior será a semelhança entre os itens.
        
        Como exemplo, pretende-se encontrar o grau de similaridade entre Item 2 e Item 5 da Tabela \ref{tab:matriz-av-item-user}. Para tanto, considera-se as respectivas colunas de avaliações que usuários forneceram à cada um e a Equação \ref{eq:sim-cosseno}. Por fim, tem-se o seguinte cálculo:
        
        \begin{eqnarray}
            sim(2, 5) &=& \frac{(4\cdot 5)+(5\cdot 1)+(3\cdot 2)+(5 \cdot 4)}{\sqrt{4^2+5^2+3^2+5^2}\sqrt{5^2+1^2+2^2+4^2}} \nonumber \\
            sim(2, 5) &=& 0,87 \nonumber
        \end{eqnarray}
        
        Os itens 2 e 5, portanto tem alta similaridade. Além disso, ao considerar a visão geométrica, têm-se um ângulo de $29,5^o$ formado entre os itens.
    
        %##########    Cosseno ajustado   ###################
        Como dito anteriormente, o método do cosseno não leva em conta o perfil de avaliações do usuário no cálculo da similaridade entre itens, no entanto, um método semelhante chamado \textit{cosseno ajustado}, corrige essa imperfeição. Primeiramente, define-se o cálculo pela Equação \ref{eq:sim-cosseno-ajustado}.
        
        \begin{equation}
             sim(a, b) = \frac{\sum_{u\in U}(r_{u, a}-\overline{r_u})(r_{u, b}-\overline{r_u})}{\sqrt{\sum_{u\in U}(r_{u, a}-\overline{r_u})^2}\sqrt{\sum_{u\in U}(r_{u, b}-\overline{r_u})^2}} \label{eq:sim-cosseno-ajustado}
        \end{equation}
        
        Observa-se, então, que o ajuste se refere à subtração da média das avaliações dadas pelo usuário $u$ a todos os itens. Assim, o efeito negativo causado pela média alta ou baixa de avaliações do usuário é reduzido, deslocando as avaliações para a média do usuário ao invés da média do item. Além disso, o intervalo de valores possíveis com o método do cosseno ajustado, diferentemente do cosseno, é de $-1$ a $1$ \cite{Jannach2010, Ricci2010}. Por outro lado, nota-se a semelhança do método do cosseno ajustado em relação à correlação de Pearson, descrita na Equação \ref{eq:correlacao-pearson}. No entanto, apesar de semelhantes, os contextos aos quais aplica-se cada método é diferente. A correlação de Pearson é utilizada para o cálculo de similaridade entre usuários, já o cosseno ajustado destina-se a encontrar a semelhança entre itens, apesar de ambas fazerem uso da média de avaliações de cada usuário.
        
        %
        \subsubsection{k Vizinhos Mais Próximos (kNN)}
            
            O kNN é um dos principais algoritmos para geração de recomendação e predições de avaliações \cite{Bobadilla_2013}. Além disso, tem como objetivo geral operar como classificador, assim, dado um ponto em um espaço, o kNN encontrará os k pontos mais próximos com base em um conjunto de outros pontos pré-classificados e revelará à qual classe pertence. 
            
            Como exemplo, a \reffig{cap3_knn-no-class} traz dois conjuntos de pontos, azuis mais abaixo na imagem e vermelhos acima. Em meio a esses pontos à um outro, rosa, ao qual deseja-se saber à qual grupo pertence (independentemente de sua cor).             
            
            \begin{figure}[htb]
                \caption{Exemplo de grupos para classificação com o kNN}
                \includegraphics[width=0.55\textwidth]{cap3_knn-no-class}
                \label{fig:cap3_knn-no-class}
                
                {\footnotesize Fonte: Adaptado de \citeonline{Ricci2010}}
            \end{figure}
                        
            A partir de cálculos de similaridade o algoritmo encontrará os $k$ pontos mais próximos, onde tal valor varia de acordo o contexto da aplicação. \citeonline{Duda2000} sugerem um $k$  igual a raiz quadrada do total de pontos $n$, ou seja, $k$ é igual $\sqrt{n}$, no âmbito geral de classificadores, já no contexto de sistemas de recomendação, onde há bases com milhares de usuários, \citeonline{Jannach2010} afirma que $k$ entre $20$ a $50$ é uma boa estimativa. 
            
            Em um aspecto gráfico, $k$ pode ser interpretado como o número de pontos aos quais podem ser inseridos dentro de um círculo centrado no ponto que será classificado, como pode ser visualizado na \reffig{cap3_knn-class}. 
            
            \begin{figure}[htb]
                \caption{Exemplo de classificação com o kNN}
                \includegraphics[width=0.55\textwidth]{cap3_knn-class}
                \label{fig:cap3_knn-class}
                
                {\footnotesize Fonte: Adaptado de \citeonline{Ricci2010}}
            \end{figure}            
            
            % CF baseado em kNN é simples e implementação direta
                 
            O algoritmo de $k$ vizinhos mais próximos opera em três etapas, sendo a primeira, a determinação dos vizinhos mais próximos do usuário $x$ conforme cálculos de similaridade. A seguir, previsões são computadas sobre as avaliações que $x$ daria a itens, aos quais, ainda não conhece, a partir de funções de agregação como, por exemplo, média e soma ponderada de notas de outros usuários ao item. Por fim, com base nas avaliações obtidas, os $m$ melhores itens são recomendados ao usuário \cite{Bobadilla_2013}.
            
            
            % FC baseada em usuário
            O kNN pode ser aplicado nas duas categorias de filtragem colaborativa baseada em memória. A primeira delas é \textit{baseada em usuário}, onde as sugestões são fundamentadas em outros utilizadores com preferências semelhantes. Assim, os itens recomendados não foram comprados pelo usuário ou este não os conhece, no entanto os mais semelhantes o fizeram \cite{Ricci2010}. Contudo, a abordagem tem um custo elevado para processamento da matriz de avaliações e geração de recomendações, onde, a cada recomendação produzida, todos os cálculos são re-executados, ou seja, opera em modo \textit{online}.
            
            % FC baseada em item
            A segunda abordagem é \textit{baseada em item}, onde um item $i$ é avaliado e indicado com base nas notas que o usuário $u$ forneceu para itens similares aquele em questão. Itens são, então, similares se diversos usuários os avaliaram de maneira equivalente \cite{Ricci2010}. Ademais, o desempenho em termos de processamento, comparada à abordagem anterior, é superior, já que parte considerável do processamento pode ser feito \textit{offline} \cite{Jannach2010, Miranda2010}.
            
            %\subsubsection{FC baseada em usuário}  
            % mais custosa (online), toda vez que for recomendar, tem que calcular tudo de novo.
            % encontrar usuários semelhantes com 
            % Baseado em item
                        
            % \subsubsection{FC baseada em item}
            % compara os itens com base nas avaliações dos usuários. 
            % "Quem gosta desse item também costuma gostar deste".
            % Menos custos (offline)
    
    
    \subsection{Baseado em modelo}
    % processamento offline

        Sistemas de Recomendação baseados em modelo  não fazem uso direto da matriz de avaliações para geração de recomendação, contudo ela é utilizada para o aprendizado de um modelo, ao qual fará as recomendações. \cite{Adomavicius2005}. Primeiramente, os dados de avaliações são processados internamente, ou seja, \textit{offline}, antes mesmo de  recomendações serem calculadas. Assim, no momento, em que se recomendar itens, apenas o modelo será necessário \cite{Jannach2010}.
        
        \subsubsection{Fatoração de matriz}
        
        Os modelos de fatoração de matriz levam em conta usuários e itens para explicar as avaliações dadas a partir de vetores de fatores resultantes da inferência dos padrões de notas \cite{Koren2009}. Tais fatores podem ser considerados características do item como, por exemplo, no contexto de livros, o autor, o gênero, no entanto podem não ser interpretáveis, ou seja, não se consegue determinar a qual característica se refere. A partir disso, recomendações serão feitas quando usuários e item forem semelhantes em relação a esses fatores \cite{Jannach2010}. Por outro lado, o uso de avaliações explícitas pode não ser possível, devido à quantidade insuficiente de notas dadas por cada usuário. Apesar disso, o método permite o emprego de informações adicionais para obter as preferências de usuários. Avaliações implícitas obtidas a partir de seu comportamento além de históricos de compra, navegação e padrões de busca são utilizáveis nesse contexto \cite{Koren2009}. 
        
        
        %Entre as diversas técnicas existentes para encontrar os fatores latentes, há o método de Decomposição de Valor Singular (SVD), ao qual, afirma que uma matriz pode ser descomposta em um produto de outras três, como mostra a Equação \ref{eq:svd}, onde a $M$ é a matriz de avaliações $U$ e $V$ são os vetores singulares esquerdo e direito, respectivamente, e $\Sigma$ representa os valores singulares. E, como requisito, todas as matrizes devem ser quadradas \cite{Jannach2010}.
        
        % \begin{equation}
        %     M = U \Sigma V^T \label{eq:svd}
        % \end{equation}
                
        Entre as diversas técnicas existentes para encontrar os fatores latentes, há o método de Decomposição de Valor Singular (SVD). Neste modelo, cada item é ligado com um vetor $q_i$, no qual, os elementos indicam o quanto o item possui os fatores do vetor, cada usuário é associado a um vetor $p_u$, que indica o grau de interesse do usuário nos itens que tem tais fatores altos ou baixos. Nesses casos, os valores que os fatores podem assumir estão entre $-1$ e $1$ .  A Equação \ref{eq:svd} demostra o cálculo para predição de avaliações do usuário $u$ ao item $i$ \cite{Ricci2010}.
        
        \begin{equation}
            \hat{r}_{ui} = \mu +b_u +b_i + q^T_ip_u  \label{eq:svd}
        \end{equation}
        
        Ao efetuar o produto $q^T_ip_u$ exprime-se o interesse do usuário nas propriedades do item. Os demais termos da Equação \ref{eq:svd} indicam a média global de avaliações de todos os itens ($\mu$) e os desvios, em relação a $\mu$, do usuário ($b_u$) e do item ($b_i$) \cite{Ricci2010}.
        
        Por fim, o SVD é capaz de gerar boas recomendações, entretanto é computacionalmente caro e deve ser executado \textit{offline} e pode ser aplicado, apenas, em situações em que as informações de preferências não mudam com o tempo \cite{Bobadilla_2013}.
        
        \subsubsection{Métodos Probabilísticos}
        
        Os métodos probabilísticos procuram inferir a partir do uso de conceitos de estatística e probabilidade, as expectativas de eventos ocorrerem.  No contexto de SR, significa mensurar a possibilidade de um usuário avaliar um determinado produto com a nota determinada. Para tanto, pode se considerar a predição como um problema de classificação , onde se deseja colocar um objeto, entre diversas categorias, naquela que melhor se enquadra \cite{Jannach2010}.
                
        Entre os diversos métodos está o classificador de Bayes ao qual avalia a probabilidade de um evento ocorrer com base em outros eventos, ou seja, dados um conjunto de acontecimentos já decorridos no passado, qual a probabilidade de um determinado evento ocorrer no futuro. 
        Assim, o teorema de Bayes, descrito pela Equação \ref{eq:bayes}, pode ser usado pode ser utilizado para o cálculo da probabilidade para o evento \cite{Aggarwal2016}.
        
        \begin{equation}
            P(A \mid B) = \frac{P(B \mid A) \cdot P(A)}{P(B)} \label{eq:bayes}
        \end{equation}

        Considerando-se os eventos A e B, conforme Equação \ref{eq:bayes}, a expectativa de o evento A ocorrer no futuro sabendo que B transcorreu é determinada a partir da probabilidade do evento A ocorrer por si só ($P(A)$), do evento B, individualmente, ($P(B)$) e a probabilidade do evento B ocorrer em função de A ($P (B \mid A)$).
        
        No contexto de sistemas de recomendação o evento A, presente na Equação \ref{eq:bayes}, é visto como a probabilidade do usuário $u$ dar uma determinada nota $v_s$, dentre as possíveis notas, ao item $i$, tendo como base as avaliações já fornecidas por $u$. \cite{Aggarwal2016}
        
        \begin{equation}
              P(r_{ui} = v_s \mid r_u) = \frac{P(r_{ui}=v_s) \cdot P(r_u \mid r_{ui} = v_s)}{P(r_u)}
        \end{equation}
        
        O método probabilístico com o Teorema de Bayes apresenta algumas vantagens, entre elas, a compensação de pontos de ruídos nos dados de treinamento, não ter \textit{overfitting}, podendo assim aprender com modelos generalizados, além ser capaz de operar com um conjunto de dados pequeno \cite{Jannach2010}.
                
        \subsubsection{Redes Neurais}
        % o que são e onde pode usar
        Redes Neurais Artificais (RNAs) tentam retratar de forma matemática, o comportamento do cérebro biológico, ao qual é formado por células chamadas neurônios e suas diversas interconexões, onde, a cada das ligações, é atribuído um peso. Desse modo, a aprendizagem consiste em alterar, a partir de treinamento, os valores de cada peso a fim de se ter um comportamento específico para a rede. Além disso, a entrada e saída da rede é composta por um conjunto de neurônios que são ligados a parte externa da rede. \cite{Russell2009}. 
        
        A Figura \ref{fig:cap3_rna} mostra uma RNA chamada Percéptron de Múltiplas Camadas (MLP), onde os neurônios são representados por quadrados e elipses, e as conexões entre eles, por setas indicando os pesos a elas associados. A RNA mostrada tem duas entradas, representadas pelos neurônios um (1) e dois (2); dois neurônios, três (3) e quatro (4), na cada intermediária, também chamada oculta e, por fim, mais dois, cinco (5) e seis (6) na camada de saída.
        
        \begin{figure}[htb]        
            \caption{Rede Percéptron de Múltiplas Camadas}
            \includegraphics[width=0.6\textwidth]{cap3_rna}
            \label{fig:cap3_rna}
            
            \footnotesize{Fonte: \citeonline{Russell2009}}
        \end{figure}
        
        Redes neurais são capazes de aprender os padrões em dados de entrada e atuarem como classificadores. Entre as principais vantagens de RNA's de múltiplas camadas entre os demais classificadores é a capacidade de lidar com funções não-lineares, ou seja, a relação entre entradas e saídas é variável. \cite{Aggarwal2016}.
        
        No contexto de Sistemas de recomendação, redes neurais podem ser utilizados para detectar padrões nas avaliações dadas pelos usuários aos diversos itens, e a partir disso, fazer predições de notas à itens que o usuário ainda não avaliou. Assim, considerando uma matriz de avaliações de usuários para itens, como na Tabela \ref{tab:matriz-av-item-user}, a forma com a qual o usuário \textit{Jack} avaliaria o \textit{Item 4} pode ser determinada por RNA \cite{Ricci2010}.
                  
        \subsubsection{Baseado em regras de associação}
        
        Há uma relação entre sistemas baseados em regras e os sistemas de recomendação baseados em regras, onde o primeiro foi proposto para a descoberta das relações entre transações. Assim, busca-se, por exemplo, descobrir qual a relação de compra de produtos entre as diversas transações, ou seja, se a aquisição de um item pode implicar na compra de outro \cite{Aggarwal2016}.  
        
        Recomendações são feitas para um usuário com base nas regras de associação que melhor se encaixam no seu histórico de transações.
        % Quem costuma comprar desse, compra desse outro também.
        Por fim, a partir de informações de relação de compras de produtos é possível aplicar promoções e mudança de layout de estabelecimentos \cite{Jannach2010}.
            
\section{Baseada em conteúdo}

    Sistemas de Recomendação baseados em conteúdo, aos quais têm como origens as pesquisas de filtragem de informação e recuperação de informação \cite{Cazella2010}, tentam recomendar itens ao usuário de acordo com as características de itens que ele gostou e das características do item em específico \cite{Ricci2010}. A recomendação será feita a partir da correspondência do perfil do usuário com as características dos itens. Assim, para esse tipo de sistema necessita-se apenas dos dados referentes aos itens, ou seja, sua descrição, e às preferências de usuário, isto é, seu perfil, não carecendo de uma grande comunidade de usuários para fazer recomendações. \cite{Jannach2010}. 

    \subsection{Descrição do item}
   
    Um item pode ser descrito em termo de seus atributos ou de seu conteúdo. No contexto de SR baseado em conteúdo essa definição pode ser obtida a partir de duas formas, sendo elas a explícita e implícita \cite{Jannach2010}. A forma explícita faz uso de características bens definidas dos itens. No caso de um livro, por exemplo, essas características são autor, gênero, preço, número de páginas entre outros. Em um sistema de recomendação essas informações devem ser inseridas manualmente para que possam ser utilizadas nos algoritmos. Por outro lado, a forma implícita faz uso de algoritmos que extraem informações dos itens. Além disso, essa abordagem é amplamente utilizada no contexto de recomendação de documentos textuais como, por exemplo, artigos científicos \cite{Garcia2013}.
    
    % Entre as principais formas de revelar as melhores palavras para representar o documento está a abordagem booleana, ao qual.
    
    Entre as diversas abordagens representação de itens textuais está o modelo vetor de espaço baseado em palavras-chave. Nesse caso, busca-se representar um item por um conjunto de palavras melhor o descrevem \cite{Jannach2010}.
    
    Um padrão que seguido para SR baseada em conteúdo é usar o conteúdo em si do item para gerar recomendação e não os meta-dados, ou seja, as informações sobre o item. Entre as abordagens que aplicam esse conceito está a TF-IDF, onde os itens são representados como vetores de pesos, sendo cada posição correspondente a um termo com sua respectiva relevância, no contexto de documentos textuais. 
    %Contudo, é possível interpretar as posições como sendo características do item. 
    O TF-IDF é composto de um produto de dois pesos, o primeiro a Frequência do Termo (TF) e a Frequência Inversa do Documento (IDF), onde TF indica a frequência de cada termo no documento. No entanto, documentos maiores terão frequências maiores para as palavras e os menores o contrário, tornando injusta a medição por frequência absoluta. Por isso, é necessária uma normalização \cite{Jannach2010}. 
    
    A Equação \ref{eq:tf} apresenta o cálculo do TF, onde calcula-se a frequência normalizada de um termo $i$ em um item $j$ com base na frequência absoluta do termo no documento dividido pela número de ocorrências da palavra mais frequente entre no documento \cite{Jannach2010}.
     
    \begin{equation}
        \operatorname{TF}(i,j) = \frac{\operatorname{freq}(i,j)}{\operatorname{maxOutros}(i,j)} \label{eq:tf}
    \end{equation}
    
    Já o IDF tem por objetivo reduzir o impacto de palavras demasiadamente frequentes, ou seja, que são comuns em diversos documentos, como preposições e artigos. A Equação \ref{eq:idf} demonstra o cálculo do IDF para o termo $i$ de acordo com o número total de documentos recomendáveis $N$ e $\mathrm{n}(i)$, o número de documentos em que $i$ aparece \cite{Jannach2010}.
    
    \begin{equation}
        \operatorname{IDF}(i) = \log\left(\frac{N}{\operatorname{n}(i)}\right) \label{eq:idf}
    \end{equation}
        
    O cálculo de TF-IDF é, portanto, definido pela Equação \ref{eq:tf-idf}
    
    \begin{equation}
        \operatorname{TF-IDF}(i,j) = \operatorname{TF}(i, j)\cdot \operatorname{IDF}(i) \label{eq:tf-idf}
    \end{equation}

    
    Por fim, técnicas são necessárias para eliminar termos irrelevantes do documento. A primeira se refere a retirada de palavras de parada, como artigos e preposições. Outra técnica, lematização, consiste em substituir palavras semelhantes por sua palavra original, como ``ligado'', ``ligou'' e ``liga'' poderiam ser trocadas por ``ligar''. Outra maneira é reduzir o tamanho do vetor que representa o item, para as $n$ palavras que melhor o representam \cite{Jannach2010}.
    
     
    \subsection{Perfil de usuário}
    
    O perfil do usuário consiste num conjunto de características em que demonstrou-se, no passado, interesse por parte do indivíduo. Além disso, tal perfil pode ser adquirido de duas maneiras: explícita ou implicitamente. Na abordagem explícita, o usuário é diretamente indagado sobre seus gostos e preferências através de um conjunto de questionamentow elaborados pelo sistema e, fundamentando-se nas respostas, o perfil é traçado \cite{Adomavicius2005}.
    
    Na forma implícita, por outro lado, é utilizado o histórico do usuário para extrair suas preferências a partir de algoritmos de aprendizado de máquina ou \textit{machine learning}. Segundo \citeonline{Mitchell1997}, aprendizado de máquina consiste em permitir que um computador aprenda a executar um conjunto de tarefas a partir de um conjunto de dados de experiências prévias.
    Entre as técnicas de aprendizado de máquina aplicados a extração de perfil de usuário estão Árvores de Decisão, Redes Neurais, Feedback de Relevância além da computação evolucionária a partir algoritmos genéticos e, por fim,  métodos probabilísticos \cite{Ricci2010}.
   
    \subsection{Recomendação}
    
    Recomendações são feitas, no contexto de vetores de espaço, a partir da combinação dos vetores que descrevem os itens com o vetor que descreve as preferências de usuário. Assim, os itens com maiores semelhanças com o perfil do usuário são recomendados \cite{Aggarwal2016}. No entanto, é possível também fazer recomendações a partir da comparação das características dos itens que o usuário tem preferência com as apresentadas nos itens aos quais se pretende recomendar. Isso se torna possível graças às medidas de similaridade, conforme visto na Seção \ref{sssec:similaridade}.

    % Falar sobre descrição do item 
    %     Explícita
    %         Características bem definidas (autor, afins)
    %     Implícita
    %         TF-IDF
    %             Documentos e itens
    % Perfil do usuário
    %     explícito
    %         baseado em regras
    %     implícito
            
    %             frequência
    %             binário
                
    %         a partir de histórico
    %         machine learning
            
    %         árvores de decisão
    %         Indução de regras
    %         kNN
    %             estruturado - distancia euclidiana
    %             não-estruturado - cosseno
    %         redes neurais
    %         AG
    %         Redes 
    % Recomendação
    %     Vetores
    %         Multiplicação de vetores: quanto maior melhor
        
\section{Baseada em conhecimento} 
    
Quando se trata de recomendar, nem sempre se terá à disposição uma base de dados com o histórico de interações de usuários. Além disso, mesmo com a existência de tal base,  os dados contidos podem não ser úteis quando trata-se de itens com longa duração como carros, casas, entre outros, onde a ocorrência de compras é muito baixa, em torno de anos \cite{Jannach2010}. Assim, as abordagens descritas anteriormente, ou seja, baseada em filtragem colaborativa e baseada em conteúdo, não são aplicáveis já que não há uma base de dados confiável para extrair informações e traçar um perfil para cada usuário  \cite{Ricci2010}. Por outro lado, há ocasiões em que o usuário esteja disposto a adquirir um produto que possua um conjunto de características específicas. 

Em meio às situações descritas, surge uma abordagem denominada recomendação baseada em conhecimento. Essa abordagem explora outros meios de informações, sendo elas informações sobre o usuário e sobre o item.
Assim, pode ser descrita como uma forma de se obter um conjunto de itens para recomendação aos quais satisfazem um conjunto de restrições definidas pelo usuário, podendo ser características, recursos etc \cite{Jannach2010}.

Segundo \citeonline{Ricci2010} e \citeonline{Aggarwal2016}, há dois tipos específicos de sistemas de recomendação baseados em conhecimento, baseado em restrições e baseado em casos, que diferem de acordo com o modo utilizado para obter itens para indicação.

\subsection{Baseado em restrições}
    É mais rígido onde apenas itens com as características definidas pelas regras são escolhidos. Além disso, a tarefa de levantar um conjunto de itens que satisfaz as necessidades do consumidor é denotada como uma tarefa de recomendação \cite{Ricci2010}. Segundo \citeonline{Schreiber1999}, tarefa define, em termos de pares de entrada e saída, um objetivo de raciocínio. Assim, a tarefa de recomendação busca a partir de um conjunto de requisitos elencar um conjunto de itens que os satisfaz.
    
    A tarefa é executada sobre uma base de conhecimento, ao qual contem regram que relacionam os requisitos do usuário com as características dos itens. Assim, a base é formada por dois conjuntos de variáveis, $V_C$ e $V_{PROD}$, e três conjuntos de restrições, $C_R$, $C_F$, $C_{PROD}$, $C_C$.
    O primeiro conjunto de variáveis, $V_C$, tratá das propriedades do usuário, ou seja, a descrição possível dos requisitos de usuários como peso, tamanho etc. Já o outro conjunto de variáveis, $V_{PROD}$, descreve as propriedades de um dado item pode ter.
    
    Por outro lado, em relação ao conjunto de restrições, o primeiro deles, $C_R$, restringe sistematicamente os possíveis valores das propriedades de usuários, como por exemplo, a propriedade ``tamanho'' deve ter o valor ``pequeno''.
    O segundo conjunto, $C_F$, define condições de filtragem e define a relação entre o usuário e um produto. O terceiro conjunto, $C_{PROD}$ define os possíveis valores que as propriedades que o item pode ter como, por exemplo, a propriedade ``tamanho'' para um item, pode assumir os valores ``pequeno'', ``médio'', ``grande''. Por fim, o último conjunto, $C_C$ se refere às restrições que representam requisitos do usuário.
    
    Com base nos conjuntos de restrições e de variáveis, uma solução para o problema de satisfação de restrições consiste em instanciações das variáveis de tal forma que as restrições especificadas são atendidas. Para tanto, faz-se uso de algoritmos de satisfação de restrições além de consultas conjuntivas de banco de dados. Após um conjunto de itens ser selecionado é feito um ranqueamento indicando os resultados mais relevantes \cite{Ricci2010}.
    
\subsection{Baseado em caso}
    Essa abordagem confere uma tolerância maior ao não cumprimento das regras, já que recomenda itens semelhantes às restrições. Itens são recuperados usando medidas de similaridade que descrevem o quanto as propriedades do item se aproximam com alguns requisitos de usuário \cite{Aggarwal2016}. Caso o usuário considerem inadequados os itens resgatados, este pode modificar os requisitos e uma nova recomendação é feita \cite{Lorenzi2005}.
    Assim, tem-se a distância de similaridade para comparação de requisitos e itens, onde um item $p$ e requisitos $r$ pertencentes a $REQ$ são comparados através da Equação \ref{eq:case-based}, onde $w$ é a importância do requisito $r$ e $\operatorname{sim}(p, r)$ é a distância da característica em relação à expressa pelo usuário \cite{Jannach2010}. 
    
    \begin{equation}
        \operatorname{sim}(p, REQ) = \frac{\sum_{r \in REQ}w_r \cdot \operatorname{sim}(p,r)}{\sum_{r \in REQ}w_r} \label{eq:case-based}
    \end{equation}
    
    Um sistema baseado em caso terá uma base de casos, na qual acondiciona um conjunto de problemas e soluções prévias. Problemas novos são, então, solucionados a partir da adaptação das soluções de problemas similares anteriores \cite{Bridge2005}. 
            
\section{Abordagem Híbrida}

    % Aproveita o melhor de cada abordagem e 
    As abordagens descritas até o presente momento no trabalho se destacam em determinadas situações, mas deixam a desejam em outras. Por exemplo, a Filtragem Colaborativa, como será descrito na Seção \ref{sec:desafios}, apresenta dificuldades em gerar recomendações adequadas para um novo usuário e para um novo item \cite{Ricci2010}. 
    
    Como forma de contornar as limitações das abordagens surgem os sistemas de recomendação híbridos, onde se aplicada em um mesmo SR, diversas abordagens, para gerar recomendações, permitindo que os pontos fortes de cada uma sejam aproveitados enquanto os pontos fracos atenuados. Para tanto, os SR's fazem uso de diversos dados de entrada, geralmente utilizados por cada abordagem, sendo eles, avaliações de itens, perfis de usuários, modelos de conhecimento, características dos itens, entre outros. As abordagens chamadas ``puras'' utilizam algumas dessas entradas, enquanto a abordagem híbrida pode fazer uso de múltiplas fontes de informação \cite{Jannach2010}.
    
    O processo de hibridização de um sistema de recomendação, ou seja, a mesclagem de diversas abordagens, deve seguir algum critério específico que indique como e quando os itens de recomendação de cada abordagem serão utilizados. \citeonline{Jannach2010} expõe três designs para tornar SRs híbridos: monolítico, paralelo e sequencial, cada um com suas respectivas subdivisões.
    
    O design monolítico apresenta um único componente que integra múltiplas abordagens por preprocessamento e combinação de múltiplas fontes de conhecimento \cite{Jannach2010}, ou seja, apenas um componente abriga todas as abordagens aplicadas no sistema, onde se obter o comportamento híbrido a partir da mudança de comportamento explorando os diferentes tipos de dados de entrada. Sistemas híbridos com design monolítico podem ser sub-categorizados em dois outros designs sendo eles por combinação de características e aumento de características.
    %onde o primeiro faz uso de diversas entradas distintas e, o segundo, aplica mais passos de transformação além da utilização das entradas distintas.
    
    O design paralelo aplica mais componentes, ao contrário do monolítico, onde vários componentes de recomendação são utilizados lado a lado. As respectivas saídas são combinadas a partir de mecanismos de agregação, sendo o mesclado, ponderado e chaveado. 
    %O primeiro aplica uma operação de união aos conjuntos de itens recomendados por cada componente. Por outro lado, o ponderado consiste em atribuir um peso ao grau relevância que cada abordagem atribui a cada item, gerando uma saída que é uma soma ponderada dos componentes. Por fim, o chaveado alterna a abordagem que é aplicada a cada momento no sistema.
    
    O design sequencial aplica as diversas abordagens sequencialmente, ou seja, a saída da abordagem atual é aplicada como entrada na seguinte e, assim, sucessivamente até a última abordagem gerar o conjunto de itens para recomendação ao usuário. Esse tipo de design pode ser dividido ainda em duas categorias: cascata e meta-nível.
    
    
    \cite{Burke2002}    

    

\section{Outras Abordagens}
    abc 
    
    \subsection{Baseado em Contexto}
    abc
    
    \subsection{Geográfica}
    abc
    
    \subsection{Demográfica}
    abc
    
    \subsection{Baseada em Utilidade}
    
    % dissertação thales
    abc
    
    
% Estudar também a questão geográfica (recomendação de acordo com o lugar) porque no caso de comida, as comidas mais apropriadas para cada indivíduo estão imersas num contexto cultural.
\section{Desafios} \label{sec:desafios}

     % Problemas: cold-start, esparçabilidade, escalabilidade, shilling attacks ... Soluções propostas por outros autores.
    
    Os sistemas de recomendação apresentam algumas limitações e obstáculos que dificultam que as recomendações sejam calculadas e realizadas. 
    
    Quando o sistema recebe novos usuários para recomendação não há informações suficientes sobre suas preferências que possibilite a geração de  recomendações para ele. Por outro lado, em relação aos itens, não existe nenhuma interação com usuários que permita ser recomendado a alguém. Os sistemas de recomendação ``puros'', ou seja baseados em filtragem colaborativa, baseados em conteúdo entre outros, não conseguem lidar com esses problemas, também chamados de \textit{cold start}, isto é partida fria, no entanto, sistemas híbridos o fazem \cite{Miranda2010}.
    
    Além do problema em recomendar novos itens, há o efeito gerado pela adição deste item, ou seja, o conjunto de recomendação prévio não leva em conta esse novo item e, portanto, pode não ser preciso. Assim, é necessário que o sistema ajuste as bases de recomendação para que o novo item seja recomendado. No entanto, tais bases contêm entre centenas à milhares de itens e usuários e, por isso, efetuar a atualização de uma base de grande porte pode ser custoso e inviável para cada novo item acrescentado. Esse problema se refere à \textit{escalabilidade} de sistemas de recomendação \cite{Lue2012}. 
    
    Outro desafio se refere a \textit{esparsabilidade}, ou seja, a proporção dos itens avaliados por usuários em relação ao número total de itens é consideravelmente pequena e mal distribuída. Além disso, a intersecção do conjunto de itens avaliados por pares de usuários tende a ser modesta. Isso acontece em função da baixa taxa de avaliação de itens, seja explícita ou implicitamente \cite{Lue2012}.

        % pode ser feito em parte offline

\section{Aplicações}

% Netflix Prize
% Spotify
% Financeiro
% IoT
% Turístico
% Google, a forma como é feita a popularidade de uma página.

% Analog devices
%There are now many examples of fielded case-based recommenders. The electronics component manufacturer Analog Devices, for example, continues to use a case-based recommender to provide customers with more intuitive and flexible access to its catalog of operational amplifiers (OpAmps)
	\chapter{Sistema proposto}
\label{cap:sistema_proposto}
	\chapter{Aplicação do Modelo e Análise dos Resultados}
\label{cap:avaliacao_sistema}

Nesse capítulo será realizada a apresentação dos resultados obtidos a partir da aplicação do modelo proposto materializado em um sistema computacional a um cenário de estudo.

%%%%%%%%%%%%%%%%%%%%%%%%%%%%%%%%%%%%%%%%%%%%%%%%%%%%%%%%%%%%%%%%%%%%%%%%%
\section{Introdução}

As discussões nesse capítulo estão decompostas em três partes, sendo elas:

\begin{itemize}[noitemsep,topsep=5pt]
    \item \textbf{Fluxo de execução do sistema:} Apresenta a dinâmica do sistema a começar pela leitura dos produtos contidos na geladeira e finalizando na apresentação de recomendações e receitas, além de algumas funcionalidades, dentre as descritas no Capítulo 4.
    \item \textbf{Cenário de Aplicação:} Apresenta o cenário em que o sistema será avaliado. Assim, detalha-se o ambiente da simulação, os dados utilizados, sua obtenção, organização, além das ações executadas pelo sistema.
    \item \textbf{Avaliação do Protótipo:} Nesta seção, a partir do cenário proposto, serão demonstradas as interações com o protótipo e o resultado das interações, seja na forma de recomendações de produtos e receitas ou através de informações de estado da geladeira e de produtos contidos nesta.
\end{itemize}

%%%%%%%%%%%%%%%%%%%%%%%%%%%%%%%%%%%%%%%%%%%%%%%%%%%%%%%%%%%%%%%%%%%%%%%%%
\section{Fluxos de Execução do Sistema} \label{sec:fluxos-de-execucao}

No sistema diversas operações são executadas, às quais incluem os diversos componentes que o compõem. A seguir, são apresentados e explanados alguns fluxos de execução.

% Demonstrar fluxo de execução da leitura de conteúdo
\subsection{Leitura de Conteúdo}

No momento em que a geladeira for ligada, conforme a Figura \ref{fig:cap5_diagr_leitura}, o sistema será inicializado e as portas (ou terminais) serão configurados a fim de permitir a comunicação com os leitores e com o mecanismo de fechamento. Ademais, os eventos, disparados pela mudança do estado da porta, também são configurados. Após a conclusão das ações mencionadas, o sistema entra em modo de espera.

A partir desse ponto, o sistema dependerá da ação do usuário para operar. Assim, quando o usuário abrir ou fechar a porta, o sensor de fechamento irá propagar um novo sinal para o sistema. Quando o evento de mudança ocorre verifica-se qual o tipo de evento ocorreu, ou seja, abertura ou fechamento. Caso aberto, o sistema entra em modo de espera por alguns segundos até prosseguir com a operação, garantindo que a porta esteja realmente aberta e reduzindo as chances de ruídos interferirem no funcionamento do leitor. Além disso, garante-se que o usuário tenha tempo suficiente para fechar a porta de maneira voluntária. 

Após o período transcorrer, uma nova leitura é realizada e caso ainda aberta, um registro de porta aberta é enviado ao servidor. Caso for verificado, após a interação do usuário, que a porta está fechada, espera-se também alguns segundos. Após esse tempo, um comando de leitura é enviado para os leitores de etiquetas. Estes, por sua vez, realizam a captura das informações das \textit{tags} ao alcance e as enviam ao sistema da geladeira. Por fim, um registro de interação é enviado no servidor e gravado na base de interações.

\begin{figure}[H]
    \caption{Fluxo de leitura do conteúdo da geladeira}
    \label{fig:cap5_diagr_leitura}
    \includegraphics[height=0.9\textwidth, angle=90]{diagramas/diagr_leitura.png}
    
    \footnotesize{Fonte: Elaborado pelo Autor}
\end{figure}

%%%%%%%%%%%%%%%%%%%%%%%%%%%%%%%%%%%%%%%%%%%%%%%%%%%%%%%%%%%%%%%%%%%%%%%%%
\subsection{Listagem do Conteúdo da Geladeira}

Como  continuação do processo anterior, o processo de listagem de produtos disponibiliza na interface de usuário o conjunto de produtos disponíveis. A Figura \ref{fig:cap5_diagr_lista_prod} demonstra o fluxo de atividades para este processo.

O gatilho para tal ação é provido na interface de usuário e esta enviará uma requisição da lista de produtos referentes à uma geladeira específica. Ao receber a solicitação, o servidor realiza uma busca na base de interações pelo último registro gravado ao qual contém os códigos EPC lidos das etiquetas. Após isso, tendo o conjunto de códigos EPC, fará uma busca pelo produto correspondente a cada um. Assim, se terá uma lista de produtos e suas respectivas quantidades. Por fim, a lista de produtos, em formato JSON, é retornada à interface e apresentada ao usuário.

% Demonstrar fluxo de execução de listagem de produtos
\begin{figure}[htb]
    \caption{Fluxo para listagem de conteúdo da geladeira}
    \label{fig:cap5_diagr_lista_prod}
    \includegraphics[width=\textwidth]{diagramas/diagr_lista_prod.png}
    
    \footnotesize{Fonte: Elaborado pelo Autor}
\end{figure}

\subsection{Recomendação de Produtos Novos} \label{ssec:geracao_rec_novo}

O processo de geração de recomendações, como já dito, é automaticamente inicializado pelo servidor, conforme Figura \ref{fig:cap5_diagr_geracao_rec}. Inicialmente, a matriz de frequência de interações de usuários com produtos é criada. Vale destacar, que um usuário está relacionado à uma geladeira. Em seguida, busca-se na base de interações todas as interações armazenadas. A partir das interações, na base de metainformação é efetuada a combinação entre códigos EPC e código de barras. A matriz de frequências é, então, preenchida com os dados obtidos no processo anterior.

Como parte do cálculo da correlação de Pearson, a média de interações de cada usuário deve ser calculada. Logo após, é computada a similaridade com a Equação \ref{eq:correlacao-pearson} entre os pares de usuários, indicados na Figura \ref{fig:cap5_diagr_geracao_rec} como $U1$ e $U2$.

Utilizando as similaridades calculadas inicia-se o processo de recomendação para cada usuário ($U1$) cadastrado no sistema. Inicialmente, o conjunto de similaridades do usuário $U1$ com os demais é ordenado em ordem decrescente, ou seja, do usuário com maior similaridade ao menor. Em seguida, para cada usuário $U2$ na lista de similaridades é realizada uma subtração de conjuntos de produtos aos quais $U2$ interagiu pelos itens que $U1$ interagiu. Assim, tem-se um conjunto de produtos que $U1$ não conhece e que poderão ser sugeridos como novas opções de compra. 

O processo citado é repetido até que todos os indivíduos na lista de usuários similares forneçam recomendações ou quando o número máximo de produtos recomendados for atingido. E quando uma das duas possibilidades ocorrer, o conjunto de recomendações será salvo na base de recomendações e o processo de recomendação se repetirá para o próximo usuário.

% Demonstrar fluxo de execução de geração de recomendação
\begin{figure}[H]
    \caption{Fluxo para recomendação de produtos novos} 
    \label{fig:cap5_diagr_geracao_rec}
    \includegraphics[width=\textwidth]{diagramas/diagr_geracao_rec.png}
    
    \footnotesize{Fonte: Elaborado pelo Autor}
\end{figure}

%%%%%%%%%%%%%%%%%%%%%%%%%%%%%%%%%%%%%%%%%%%%%%%%%%%%%%%%%%%%%%%%%%%%%%%%%
\subsection{Recomendação de Produtos Faltantes} \label{ssec:cap5_rec_prod_falt}

Da mesma forma que a seção anterior, o processo de recomendação é iniciado automaticamente pelo servidor, conforme Figura \ref{fig:cap5_rec_produto_faltante}. A partir disso, para que se recomende produtos é necessário ter as listas de produtos disponíveis atualmente e de produtos requisitados. Para tanto, inicialmente, a base de interações é consultada para o resgate da última interação, tendo esta a indicação dos itens mais recentes. A partir da lista de códigos EPC extraídos das interações é realizada uma consulta à base de metainformação para que sejam obtidos os dados dos produtos.

O próximo passo consiste na obtenção, na base de estruturas auxiliares, das informações de configuração do usuário, às quais indicam os produtos essenciais. A partir da extração dessas informações é efetuada uma comparação de quantidade entre produtos disponíveis e essenciais. Caso o produto essencial esteja disponível na quantidade necessária, nada acontece. No entanto, caso o produto esteja disponível e em quantidade insuficiente, este é inserido na lista de recomendações tendo como quantidade sugerida a diferença entre o valor necessário e o existente. Caso o produto necessário não exista, sugere-se a quantidade total indicada nas configurações.

A seguir, para o conjunto de produtos recomendados, é executada uma verificação no mercado sobre a disponibilidade dos mesmos conforme a quantidade especificada no passo anterior. Assim, caso o item esteja acessível, uma indicação positiva será retornada. Caso contrário, o serviço do mercado fará uma busca por um produto similar e retornará tal item como alternativa. Por fim, podem haver casos em que nenhum item similar exista. Assim, apenas uma indicação negativa é retornada.

Com o conjunto de produtos indicados pelo mercado como disponíveis para compra, a lista final de recomendações é criada e inserida na base de recomendações.

\begin{figure}[H]
    \caption{Fluxo para recomendação de produtos faltantes} 
    \label{fig:cap5_rec_produto_faltante}
    \includegraphics[width=\textwidth]{cap5_rec_produto_faltante}
    
    \footnotesize{Fonte: Elaborado pelo Autor}
\end{figure}

%%%%%%%%%%%%%%%%%%%%%%%%%%%%%%%%%%%%%%%%%%%%%%%%%%%%%%%%%%%%%%%%%%%%%%%%%
\subsection{Recomendação de Receitas por Conteúdo}

O processo de recomendação, conforme descrito na Seção \ref{sssec:proc_ger_rec}, decorre a partir da recomendação de receitas que contenham alguns dos produtos disponíveis na geladeira. 

O processo é iniciado automaticamente no servidor e o primeiro passo ocorre por meio na seleção da última interação, contendo os códigos EPC, na respectiva base de dados.

A partir dos códigos EPC, obtém-se o conjunto de produtos correspondentes a ele considerando a base de metainformação. Então, seleciona-se na base de metainformação as receitas que contenham pelo menos um dos produtos do conjunto. A partir do número de correspondências entre produtos da lista selecionada e da receita o conjunto de receitas é ordenado em ordem descendente, ou seja, receitas com maior número de correspondências primeiro. Por fim, o conjunto gerado é registrado na base de recomendações.

\begin{figure}[H]
    \caption{Fluxo para recomendação de receitas por conteúdo} 
    \label{fig:cap5_rec_receita_conteudo}
    \includegraphics[width=\textwidth]{cap5_rec_receita_conteudo}
    
    \footnotesize{Fonte: Elaborado pelo Autor}
\end{figure}

%%%%%%%%%%%%%%%%%%%%%%%%%%%%%%%%%%%%%%%%%%%%%%%%%%%%%%%%%%%%%%%%%%%%%%%%%
\subsection{Recomendação de Receitas por Perfil}

A sugestão de receitas por perfil, conforme Figura \ref{fig:cap_diagr_receita_perfil}, considera os produtos com os quais o usuário mais interagiu durante a escolha de quais receitas recomendar. 

E, como nos demais processos, inicia-se no servidor automaticamente. 
O primeiro passo consiste na seleção de todas as interações do usuário na base de interações, seguida pela extração dos produtos de cada interação na base de metainformação. A partir disso, a matriz de frequências é criada e preenchida com base nos obtidos e da quantidade de vezes em que aparecem nas interações. O próximo passo, portanto, é a ordenação descendente através da frequência. Os produtos com maior frequência serão, então, utilizados como base na seleção de receitas. Nessa implementação, limitou-se em cinco (5) o número de produtos utilizados.

Com a lista de produtos, obtém-se o conjunto de receitas que contêm pelo menos um desses itens. O próximo passo consiste em ordenar a lista em disposição descendente a partir do número de produtos na receita correspondidos na lista de produtos.

Por fim, o conjunto de receitas sugeridas é registrado na base de recomendações.

\begin{figure}[H]
    \caption{Fluxo para recomendação de receitas por perfil} 
    \label{fig:cap_diagr_receita_perfil}
    \includegraphics[width=\textwidth]{cap_diagr_receita_perfil}
    
    \footnotesize{Fonte: Elaborado pelo Autor}
\end{figure}

%%%%%%%%%%%%%%%%%%%%%%%%%%%%%%%%%%%%%%%%%%%%%%%%%%%%%%%%%%%%%%%%%%%%%%%%%
\subsection{Fluxo para Detecção e Registro de Porta Aberta}

Considerando-se, inicialmente, que a geladeira se encontra fechada. Em uma dado momento, o usuário abre-a e insere e/ou retira produtos, podendo, ao final, não fechar a porta. No momento da abertura, o sistema da geladeira detecta a ação e inicia um período de espera. Ao final desse tempo, um registro correspondente ao estado da porta é inserido na base de estruturas auxiliares.

Com relação à consulta de estado da porta, apesar de seu respectivo diagrama não ser apresentado neste trabalho, o fluxo é muito semelhante aos relacionados à consulta de listagem de produtos. Assim, quando a interface de usuário solicita o estado atual no servidor receberá o último registro realizado. A partir dele, caso indique estado aberto, uma notificação é emitida na interface.

\begin{figure}[H]
    \caption{Detecção de porta aberta}
    \label{fig:cap5_diagr_porta_aberta}
    \includegraphics[width=\textwidth]{cap5_diagr_porta_aberta}
    
    \footnotesize{Fonte: Elaborado pelo Autor}
\end{figure}

\subsection{Listagem de Recomendações} \label{ssec:listagem_rec}

O processo de listagem de recomendações é disparado através da interface de usuário. Desse modo, a interface envia uma requisição ao servidor solicitando um conjunto de recomendações para um determinado usuário, conforme Figura \ref{fig:cap5_diagr_lista_rec}.  Ao receber a solicitação, o servidor faz uma busca na base de recomendações pelos produtos recomendados. Em seguida, os dados são organizados e o conjunto de recomendação é enviado à interface.

% Demonstrar fluxo de execução de listagem de recomendação
\begin{figure}[H]
    \caption{Fluxo para listagem de recomendações}
    \label{fig:cap5_diagr_lista_rec}
    \includegraphics[width=\textwidth]{diagramas/diagr_list_rec.png}
    
    \footnotesize{Fonte: Elaborado pelo Autor}
\end{figure}

% Demonstrar fluxo de execução de listagem de produtos

\section{Cenário de Aplicação}

A avaliação do modelo dar-se-á a partir do ambiente ilustrado na Figura \ref{fig:cap5_ambiente-cenario}. O ambiente consiste na demonstração de uma sequência de ações, começando pela interação do usuário, com o objetivo de observar o comportamento e as funcionalidades do protótipo implementado.

\begin{figure}[htb]
    \caption{Ambiente do cenário}
    \label{fig:cap5_ambiente-cenario}
    \includegraphics[width=\textwidth]{figuras/cap5_cenario.png}
    
    \footnotesize{Fonte: Elaborado pelo Autor}
\end{figure}

Em relação, as bases de dados, essas foram preenchidas antes da avaliação do sistema. 
% Descrição dos dados de produtos
Primeiramente, para a base de metainformação foram alocadas informações de quarenta (40) produtos, extraídas de produtos originais a partir de um supermercado na cidade de Araranguá. Por outro lado, a informação referente à URL da imagem do produto foi obtida em \textit{websites} de supermercados.
%Excetua-se apenas o endereço URL da imagem de cada produto ao qual foi obtido \textit{websites} de supermercados. 
% Descrição de dados de receitas
Já em relação às receitas, utilizou-se dez (10) receitas obtidas a partir de sites de culinária. Tanto as informações de produtos quanto as receitas foram formatadas conforme a Seção \ref{sssec:metainfo}.

% Descrição da quantidade de usuários
Quanto à identificação das geladeiras e, por consequência, dos usuários, não criou-se uma base específica para eles, apenas atribuiu-se um identificador utilizado em cada registro específico a um determinado usuário. Como foram definidas dez (10) geladeiras, foram atribuídos os valores de um (1) a dez (10) para diferenciação de cada uma.

% Descrição da quantidade de interações por usuários.
Para cada geladeira, foram criadas, de maneira aleatória, interações, cada qual com um conjunto de códigos EPCs representando os produtos. Assim, definiu-se que o número de interações seria maior ou igual a cinco (5) e menor ou igual a 25. Além disso, para cada interação o número mínimo de itens registrados seria de três (3) e, no máximo, dez (10).

% Configurações de produtos
Em relação às configurações, descritas na Seção \ref{sssec:base_est-aux}, todas as geladeiras apresentaram parâmetros idênticos excetuando-se as listas de produtos essenciais e suas respectivas quantidades mínimas. Assim, tais produtos e suas respectivas quantidades foram definidos randomicamente para cada usuário.
%Assim, quais produtos e quais quantidades foram definidas randomicamente. 
Quanto à quantidade de produtos requisitados, estabeleceu-se o número de cinco (5) produtos e, em relação à quantidade, definiu-se um valor randômico entre um (1) e vinte (20).

% Quais os produtos que têm RFID que serão usados na geladeira
Como forma de teste do protótipo referente à estrutura física da geladeira, dois produtos receberam etiquetas com o respectivo código EPC, sendo eles, a ``Margarina com Sal Qualy\textsuperscript{\textregistered}\footnote{http://qualy.com.br/}'' e ``Creme de Leite Tirol\textsuperscript{\textregistered}\footnote{http://www.tirol.com.br/pt/}''.

%%%%%%%%%%%%%%%%%%%%%%%%%%%%%%%%%%%%%%%%%%%%%%%%%%%%%%%%%%%%%%%%%%%%%%%%%%%%%%%%%%%%%%%%%%%%%%%%%%%%%%%%%%%%%%%%%%%%%%%%%%%%%%%%%%%%%%%%%%%%%%%%%%%%%%%%%%%%%%%%%%%%%%%%%%%%%%%%%%%%%%%%%%%%%%%%%%%%%%%%%%%%%%%%%%%%%%%%%%%%%%%%%%%%%%%%%%%%%%%%%%%%%%%%%%%%%%%%%%%%%%%
\section{Avaliação do Protótipo}

Nessa seção é descrita a avaliação do protótipo criado. Para tanto, este será aplicados aos fluxos desenvolvidos na Seção \ref{sec:fluxos-de-execucao}. Assim, para cada fluxo comparar-se-á os resultados obtidos com os esperados.

%<<Descrever aqui uma introdução para esta seção>>

\subsection{Leitura do Conteúdo}
 Inicialmente a geladeira está vazia.
 Ao adicionar dois produtos, sendo eles, a ``Margarina com Sal Qualy\textsuperscript{\textregistered}'' e ``Creme de Leite Tirol\textsuperscript{\textregistered}'', respectivamente, depois de alguns minutos os seguintes códigos EPC são lidos.
 
\begin{itemize}[noitemsep,topsep=5pt]
     \item 8665580279609348107299713701
     \item 8665580277303506451141610373
 \end{itemize}
 
 Os dados são, então, enviados ao serviço de registro de interação e o registro é gravado na base de interações, conforme apresentado no Quadro \ref{fig:cap5_registro_interacao}. 
 
 %%%%%%%% JSON DO REGISTRO  %%%%%%%%
 \begin{quadro}[htb]
     \caption{Registro de interação dos produtos}
     \label{fig:cap5_registro_interacao}
     
     \frame{\includegraphics[width=0.5\textwidth]{cap5_registro_interacao}}
     
     \footnotesize{Fonte: Elaborado pelo Autor}
 \end{quadro}
 
\subsection{Listagem de Conteúdo da Geladeira}
 
Continuando o fluxo da seção anterior, ao tentar ler o conteúdo da geladeira logo após o fechamento da porta, a listagem da interface ainda se mantém vazia. Alguns segundos depois, a listagem é atualizada e o conteúdo exibido é mostrado na Figura \ref{fig:cap5_listagem_atual}.


 %%%%%%%%%%%%%%%%%%%%%   FIGURA DA LISTAGEM APÓS A INTERAÇÃO %%%%%%%%%%%%%%%%%%%%%%%%%%%%%%%%%%%
\begin{figure}[htb]
    \caption{Listagem de produtos na Interface}
    \label{fig:cap5_listagem_atual}
    \includegraphics[width=0.8\textwidth]{cap5_listagem_atual}
    
    \footnotesize{Fonte: Elaborado pelo Autor}
\end{figure}
 
Dessa forma, afirma-se que a listagem de produtos produz resultados de acordo com o esperado, já que a lista é alterada após a interação. No entanto, o fator temporal entre a interação e a exibição é compreensível. Assim, o intervalo de tempo decorre do aguardo, na geladeira, até a realização da leitura. Isso é necessário para garantir que exista tempo hábil para coletar ou inserir todos os produtos que se deseja.


%%%%%%%%%%%%%%%%%%%%%%%%%%%%%%%%%%%%%%%%%%%%%%%%%%%%%%%%%%%%%%%%%%%%%%%%%%%%%%%%%%%%%%%%%%%%%%
\subsection{Recomendações de Produtos Novos}

Como exemplo, deseja-se gerar recomendações para o Usuário 1. Para tanto, seguiu-se o fluxo da Seção \ref{ssec:geracao_rec_novo}. Os passos foram executados até o cálculo de similaridade entre usuários sobre a matriz de frequências de interações e obteve-se o usuário mais semelhante. Tal indivíduo é identificado como Usuário 10 e o grau de similaridade entre este e o Usuário 1 foi de 0,15. A média de interações com produtos para o Usuário 1 foi de 2,05, já para o Usuário 10 foi de 2,12.

%inicia-se com a criação da matriz de frequências seguida pelo resgate das interações e, a partir dessas, dos códigos de barras e, por fim o preenchimento da matriz, conforme Figura \ref{fig:cap5_diagr_geracao_rec}.

Como forma de verificação, conforme Tabela \ref{tab:cap5_matriz_sim}, tem-se as linhas da matriz de frequência referentes aos usuários.


%%%%%%%%%%%%%%%%%%  MATRIZ  %%%%%%%%%%%%%%%%%%%%%%
\begin{table}[htb]
\caption{Matriz de frequências de interações}
\label{tab:cap5_matriz_sim}
%\begin{tabular}{C{0.72cm}c}
\toprule
\textbf{U1} \hspace{0.23cm} 0 2  2  6  2  3  6  0  2  4  2  5  2  1  1  2  1  8  5  6  5  1  2  3  3  5  5  3  1  0  0  0  0  0  0 \vspace{0.1cm} \midrule 
\textbf{U10} \hspace{0.02cm} 2  2  7  3  3  3  4  3  0  0  2  4  0  0  4  0  2  9  6  0  2  0  0  5  4  0  2  3  1  2  4  7  5  2  3 \vspace{0.1cm}  \bottomrule \vspace{0.1cm} 
%\end{tabular}

\footnotesize{Fonte: Elaborado pelo Autor}
\end{table}        

\newpage
Executando o cálculo da correlação de Pearson, descrita na Equação \ref{eq:correlacao-pearson}.

%%%%%%%%%%%%%%%%%%  EQUAÇÃO COM VALORES SUBSTITUÍDOS  %%%%%%%%%%%%%%%%%%%%%%%%%%%%%%
\begin{equation}
sim=\dfrac{\left[
\splitdfrac{
(0-2,05)(2-2,12)+
(2-2,05)(2-2,12)+
}
{\splitdfrac{
(2-2,05)(7-2,12)+
(6-2,05)(3-2,12)+
}{
(2-2,05)(3-2,12)+...}}\right]}
{
\left[
\splitdfrac{
\sqrt{\splitdfrac{(0-2,05)^2+
(2-2,05)^2+
(2-2,05)^2+
}{
(6-2,05)^2+
(2-2,05)^2+...}}
}{
\times\sqrt{\splitdfrac{(2-2,12)^2+
(2-2,12)^2+}{\splitdfrac{
(7-2,12)^2+
(3-2,12)^2+
}{
(3-2,12)^2+...}}}
}\right]
}
\nonumber
\end{equation}

\begin{equation}
sim = 0,15 \nonumber
\end{equation}

Assim, demonstra-se que o cálculo de similaridade está correto. Percebe-se que o número de produtos com os quais o Usuário 1 não apresentou nenhuma interação, ou seja, as posições na linha do usuário mencionado preenchidas com zero (0) é igual a oito (8). Assim, espera-se que pelos menos esses produtos sejam recomendados.

No entanto, um número maior de itens podem ser sugeridos. Isso ocorrerá quando a quantidade obtida de recomendações através do usuário com maior similaridade for insuficiente. Assim, recomendações de outros usuários com graus de semelhança menores serão consideradas.

Obteve-se, a partir da consulta à base de recomendações, a seguinte lista de itens:

\begin{itemize}[noitemsep,topsep=5pt]
%    \item 7892840800000 - Refrigerante Pepsi
    \item 789034630442 - Creme de Leite Parmalat
    \item 7894900093056 - Iogurte Danone
    \item 7896648699453 - Leite Integral Langaru
    \item 7894904326044 - Pizza Calabresa Seara
%    \item 7891991010481 - Cerveja Budweiser
    \item 7896256603422 - Leite Desnatado Tirol
    \item 7896256602050 - Iogurte de Morango Tirol
    \item 7891025101376 - Iogurte Danone
    \item 7896034680010 - Leite Condensado Parmalat
    \item 7891515490430 - Lasanha Calabresa Perdigão
\end{itemize}

 %%%%%%%%%%%%%%%%%%%%%   FIGURA DA LISTAGEM DE RECOMENDAÇÕES DE PRODUTOS NOVOS %%%%%%%%%%%%%%%%%%%%%%%%%%%%%%%%%%%

Os primeiros oito produtos da lista são aqueles que o Usuário 1 não interagiu, mas que o Usuário 10 o fez, na mesma sequência mostrada na Tabela \ref{tab:cap5_matriz_sim}. Já os demais itens foram sugeridos com base em outros usuários com menor similaridade. Vale ressaltar que o JSON gerado pelo sistema não foi inserido diretamente neste trabalho visto a extensão do mesmo.

\subsection{Recomendação de Produtos Faltantes}

Nesta seção, busca-se avaliar a recomendação da reposição de produtos aos quais o Usuário 2 julga serem essenciais. Para tanto, considera-se o fluxo de execução demonstrado na Seção \ref{ssec:cap5_rec_prod_falt} e o conjunto de produtos contidos atualmente e suas respectivas quantidades:

%%%%%%%%%%%%%%% Lista de um conjunto de produtos atuais %%%%%%%%%%%%%%%
\begin{itemize}[noitemsep,topsep=5pt]
    \item Mortadela Perdigão\textsuperscript{\textregistered} \footnote{http://www.perdigao.com.br/} 5 UN
    \item Pizza Calabresa Sadia\textsuperscript{\textregistered} \footnote{http://www.sadia.com.br/}, 4 UN
    \item Iogurte de Morango Activia\textsuperscript{\textregistered}\footnote{https://www.activiadanone.com.br/}, 5 UN
\end{itemize}

E o conjunto de produtos que o Usuário 2 estipulou que sempre devem estar à disposição:

%%%%%%%%%%%%%%% Lista de um conjunto de produtos essenciais %%%%%%%%%%%%%%%
\begin{itemize}[noitemsep,topsep=5pt]
    \item Mortadela, 8 UN
    \item Pizza Calabresa, 6 UN
    \item Iogurte de Morango, 5 UN
    \item Linguiça de Pernil, 15 UN
    \item Leite integral, 8 UN
\end{itemize}


Percebe-se que alguns produtos estão ausentes e, outros, em falta como, por exemplo, o produto Pizza está em quantidade insuficiente e o produto Leite está ausente. Assim, a recomendação deve conter tais produtos além dos outros não citados.

Seguindo o fluxo da Seção \ref{ssec:listagem_rec}, tem-se a listagem de sugestões na interface de usuário como é mostrado na Figura \ref{fig:cap5_rec_faltante}.

\begin{figure}[htb]
    \caption{Listagem de Recomendações por falta}
    \label{fig:cap5_rec_faltante}
    \includegraphics[width=0.8\textwidth]{cap5_rec_faltante}
    
    \footnotesize{Fonte: Elaborado pelo Autor}
\end{figure}

%%%%%%%%%%%%%%% Lista de um conjunto de produtos recomendados %%%%%%%%%%%%%%%

Pode-se verificar que há diferenciação entre os produtos, sendo de ``Reposição'' e ``Sem Alternativas''. A reposição trata da recolocação de produtos essenciais. O ``Sem alternativas'', indica que não foi possível encontrar no mercado, o produto esperado nem um similar a ele. Há, além disso, outras duas formas de recomendação de compras: ``Novo'', referente à recomendação de produtos novos, como descrito anteriormente, além de ``Similar'' que indica a reposição de produtos essenciais que não estavam disponíveis no mercado, mas que foram substituídos por produtos similares.

\subsection{Recomendação de Receitas a partir de Conteúdo}

A recomendação de receitas por conteúdo, como descrito na seção respectiva no Capítulo 4, busca disponibilizar um conjunto de receitas ao usuário de acordo com os itens que este possui em sua geladeira. 

Considera-se que, inicialmente, os seguintes produtos estejam disponíveis com suas respectivas quantidades:

\begin{itemize}[noitemsep,topsep=5pt]
    \item Leite Integral, com 2 UN
    \item Leite Condensado Parmalat\textsuperscript{\textregistered}\footnote{http://www.parmalat.com.br/}, com 3 UN 
    \item Queijo Mussarela Sadia\textsuperscript{\textregistered}, com 5 UN.
\end{itemize}

O processo de recomendação avaliará os produtos do ponto de vista de suas características, ou seja, terá um enfoque nos tipos de produtos e não em produtos de determinadas marcas.

O processo de recomendação analisa quais receitas satisfazem o requisito especificado e retornará um conjunto de itens como sugestão.

Os itens sugeridos como recomendações são mostrados na Figura \ref{fig:cap5_rec_recipe_content}.

%%%%%%%%%%%%%%%%%%    FIGURA X (Lista de receitas de recomendação)    %%%%%%%%%%%%%%%%%%%%%%%%%%%%%
\begin{figure}[htb]
\caption{Resultado da recomendação de receitas}
\label{fig:cap5_rec_recipe_content}
\includegraphics[width=0.8\textwidth]{cap5_rec_recipe_content}

\footnotesize{Fonte: Elaborado pelo Autor}
\end{figure}

Como resultado da recomendação, tem-se três receitas sugeridas dentre as cinco utilizadas nesse cenário, ou seja, duas receitas não incluíam nenhum dos produtos contidos na geladeira. A primeira receita na lista conta com o Leite, a segunda com o Queijo Mussarela e a terceira com Leite e Leite Condensado.

\subsection{Recomendação de Receitas por Perfil}

Como descrito na Seção \ref{sssec:proc_ger_rec}, a sugestão de receitas por perfil buscará receitas que contenham alguns dos itens com os quais o usuário mais interagiu. Para tanto, uma matriz de frequência de interações é elaborada. Com base nas frequências de interações do usuário em questão, uma lista é criada a partir das classes de produto.

Considerando que o total de produtos para geração de recomendação foi limitado em cinco (5) tem-se os seguintes produtos:

\begin{itemize}[noitemsep,topsep=5pt]
    \item Margarina com Sal Qualy, com 9 interações
    \item Refrigerante de Guaraná Fanta, com 9 interações
    \item Linguiça de Pernil Gold Meat, com 8 interações
    \item Iogurte Danone, com 7 interações
    \item Queijo Mussarela Sulfrios, com 7 interações
\end{itemize}

Com base neste conjunto, o processo de recomendação realiza uma busca na base de interações pela receitas que possuem tais categorias de produtos. A partir disso, o conjunto de receitas da Figura \ref{fig:cap5_rec_recipe_profile} foi sugerido.


%%%%%%%%%%%%%%%%%%%%%% FIGURA RECEITA SUGERIDA %%%%%%%%%%%%%%%%%%%%

\begin{figure}[htb]
    \caption{Lista de Receitas Sugeridas por Perfil}  
    \label{fig:cap5_rec_recipe_profile}
    \includegraphics[width=0.8\textwidth]{cap5_rec_recipe_profile}
   
    \footnotesize{Fonte: Elaborado pelo Autor}
\end{figure}

Apesar de apenas uma receita ter sido sugerida, devido ao total de receitas muito reduzido do cenário, é possível validar a recomendação, já que esta conta com o produto Queijo Mussarela.

\subsection{Alerta de Porta Aberta}

%% OBJETIVO
Os alertas enviados aos usuários que esquecem a porta da geladeira aberta, ou não a fecham corretamente, é importante para evitar gastos desnecessários com energia e, possivelmente, com alimentos estragados. Como descrito na Seção \ref{ssec:c5_camada_servicos}, o processo de verificação de porta aberta, dispara um registro no servidor. Para o exemplo, considera-se que inicialmente a porta estava fechada e, em um dado momento, foi deixada aberta.

Após alguns segundos o registro do Quadro \ref{fig:cap5_open_door_reg} foi efetuado na base de estruturas auxiliares, informando a situação.


%%%%%%%%%%%%%%%%% REGISTRO DE PORTA ABERTA  %%%%%%%%%%%%%%%
\begin{quadro}[H]
    \caption{Registro de porta aberta}
    \label{fig:cap5_open_door_reg}
    \frame{\includegraphics[width=0.5\textwidth]{cap5_open_door_reg}}
    
    \footnotesize{Fonte: Elaborado pelo Autor}
\end{quadro}

Algum tempo depois, a interface de usuário realiza uma requisição ao serviço de estado de porta. Neste caso, a aplicação da interface verificou que o estado era aberto e notificou ao usuário, conforme a Figura \ref{fig:cap5_open_door_alert}.


%%%%%%%%%%%%%%% IMG NOTIFICAÇÃO DE PORTA ABERTA  %%%%%%%%%%%%
\begin{figure}[H]
    \caption{Alerta de porta aberta}
    \label{fig:cap5_open_door_alert}
    \includegraphics[width=\textwidth]{cap5_open_door_alert}
    
    \footnotesize{Fonte: Elaborado pelo Autor}
\end{figure}





	\chapter{Considerações Finais}
\label{cap:consideracoes_finais}

%%%%%%%%%%%%%%%%%%  RETOMAR O OBJETIVO GERAL %%%%%%%%%%%%%%%%%
Este trabalho teve como objetivo o desenvolvimento de um modelo de geladeira inteligente que facilitasse o dia a dia dos usuários a partir da análises das interações destes, baseando-se nos conceitos de Internet das Coisas e Sistemas de Recomendação.

%%%%%%%%%%%%%%%%%% COMENTÁRIOS SOBRE OS OBJ. ESPECÍFICOS ATINGIDOS OU NÃO %%%%%%%%%%%%%%%%%
Para atingir o objetivo geral, inicialmente, o modelo foi projetado. Assim, definiu-se o escopo do projeto bem como todos os seus componentes, sendo eles, o sistema da geladeira, o servidor de serviços, as bases de dados e de recomendações, a interface de usuário e, por fim, os serviços disponibilizados pelo mercado.

% Elaborar e desenvolver um projeto de leitura e monitoramento dos produtos contidos na geladeira.
Com o modelo criado, um projeto de leitura e monitoramento dos produtos contidos na geladeira foi elaborado e desenvolvido. Para tanto fez-se uso da tecnologia RFID e do conceito de códigos EPC. Assim, à cada produto era agregada uma etiqueta RFID contendo o código único o que possibilitou a identificação de maneira simples no momento em que fosse inserido na geladeira. Além disso, montou-se uma estrutura que armazenasse os equipamentos de leitura e processamento. Ao final, todos as informações de produtos disponíveis estavam sendo enviadas ao servidor principal.

% Implementar um sistema de análise das interações e recomendação de produtos.
A partir do projeto de leitura e monitoramento, ou seja, a fonte de dados sobre os hábitos do usuários, tornou-se viável a implementação do sistema de análise das interações e recomendação de produtos e receitas. Para tanto, utilizou-se de algumas abordagem de sistemas de recomendação. Ao final, o sistema foi capaz de recomendar produtos novos ao usuários a partir do conceito de filtragem colaborativa. Além disso, passou-se a sugerir produtos considerados essenciais, a partir de recomendações baseadas em conteúdoa, bem como sugestão de receitas, a partir, tanto dos produtos disponíveis na geladeira, como do perfil do usuário, ou seja, nos produtos que o usuário mais apresentou interesse.

% Elaborar um cenário que permita a avaliação do modelo proposto. 
Como forma de avaliar o sistema implementado elaborou-se um cenário de testes, onde definiu-se parâmetros como número de usuários, de produtos, interações etc. Ademais, diagramas de fluxos de execução foram desenvolvidos a fim de fornecerem uma base para a avaliação do sistema.

% Avaliar e discutir os resultados obtidos a partir do sistema proposto.
Ao final, o sistema demonstrou, em todos os fluxos montados para o cenário, os respectivos comportamentos esperados. Portanto, conclui-se que o objetivo geral do trabalho foi atingido, já que a análise de interações implementada, em conjunto com o monitoramento de produtos, proveem funcionalidades que eliminam algumas atividades das mãos dos usuários e que, por consequência, tornam seu dia-a-dia mais facilitado. 

% Limitações e dificuldades
Durante a implementação do trabalho algumas dificuldades foram encontradas. A principal delas se refere ao modo como a leitura de etiquetas RFID dos produtos, ou seja, com apenas dois leitores. Claramente, é improvável que uma geladeira contenha até dois produtos, no entanto, não encontrou-se leitores capazes de efetuar a leitura de um maior número de etiquetas de forma simultânea, mas que tivesse um custo acessível ao projeto.

% Considerações sobre o trabalho


\section{Trabalhos Futuros}

Durante o desenvolvimento desse trabalho foram elencadas outras possibilidades como trabalhos futuros. As principais se relacionam ao sistema de leitura e monitoramento, além dos algoritmos de recomendação. No que se refere ao sistema de leitura, percebeu-se que é factível o aprimoramento do mesmo a partir da modificação dos dispositivos de leitura RFID. Isso seria possível a partir do uso de RFID UHF, no entanto, seria necessário encontrar um meio de baratear os custos desta tecnologia. Outra maneira, seria a criação de uma plataforma única de leitura por RFID que se estendesse por todo o espaço de um ``andar'' da geladeira. Outra de melhora no monitoramento poderia estar no uso de visão computacional como forma de reconhecer os produtos. Por fim, um avanço no projeto está na substituição da placa Raspberry\textsuperscript{\textregistered} por um sistema computacional projetado de acordo com as necessidades do sistema, o que permitiria uma redução em custos.

Outra possibilidade de trabalhos futuros, como dito está na melhoria dos algoritmos de recomendação. O primeiro ponto a ser revisto está nos dados de entrada, já que informações adicionais poderiam conferir maior precisão nas sugestões. Entre elas está a questão geográfica, ou seja, os costumes de uma determinada região tendem a determinar quais produtos e receitas os usuários apresentariam maior interesse. Além disso, informações específicas do usuário, como restrições alimentares como intolerância à lactose, glúten entre outros, bem como veganismo. Como última melhora relacionada às entradas, tem-se a possibilidade de utilização de mais detalhes a cerca dos produtos recomendados, como informações nutricionais, por exemplo. 

Em relação à recomendação em si, um aspecto que pode ser melhorado é a questão temporal nas preferências de usuários, ou seja, os gostos do usuários variam com o tempo. Além disso, é possível utilizar algoritmos mais eficientes.

Por fim, possibilidades de integração com outras tecnologias e produtos existentes foram levantadas. A primeira delas é a integração com a assistente pessoal da Amazon\textsuperscript{\textregistered}, a Alexa\textsuperscript{\textregistered}, à qual permitiria que os usuários comandassem a geladeira por comandos de voz. Além disso, o conceito de monitoramento de produtos e interações com usuário poderia ser expandidos para outros itens do dia a dia, como dispensa de alimentos poderiam ser utilizadas. 

% Alexa

%Evolução no protótipo
% Evolução nos alg de rec

% Estudar também a questão geográfica (recomendação de acordo com o lugar) porque no caso de comida, as comidas mais apropriadas para cada indivíduo estão imersas num contexto cultural.

%Visão computacional: integrar computação visual na identificação de produtos e das interações que ocorrem, ou seja, usar computação visual para 

%Integração com a assistente pessoal Alexa, onde o Amazon Echo se torna um 'gateway' e a geladeira se comunica apenas com o Echo.

%Uso de uma placa de leitura única, por toda a prateleira

%Possibilidade de ser aplicado em qualquer lugar

% Substituição do Raspberry por um sistema embarcado propriamente dimensionado para o projeto e que integre com o resto do sistema e possa gerenciar melhor o consumo de energia.

% Comunicação com outros dispositivos da casa

% Recomendação mais inteligente que consiga recomendar produtos para perfis específicos como intolerantes a lactose, vegetarianos a partir da configuração do usuário.

% Consideração do aspecto temporal em recomendações, ou seja, que as preferências de usuário variam ao longo do tempo.

% Diferenciação de quem está coletando o produto
% - Reconhecimento facial da pessoa
	
	%--------------------------------------------------
	% Elementos pós-textuais
	\bibliographystyle{ufscThesis/ufsc-alf}
	\bibliography{postextual/referencias}


	%\include{postextual/apendice}
	\anexo
\chapter{Matriz de Frequências Referentes aos Usuário 1 e 10}
Texto do anexo aqui.



\end{document}
