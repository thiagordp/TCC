\chapter{Internet das Coisas}

%%%%%%%%   PARTE 1   %%%%%%%%
\section{Conceito}


% TODO: INTRODUÇÃO

% Contextualização

% REVER
A tecnologia, com o passar dos anos, está cada vez mais presente nas indústrias, lares, comércios 
etc. ao mesmo tempo tornando-se indispensável para todas essas entidades. No entanto, nos últimos 
anos um novo paradigma está emergindo: a Internet das Coisas. A partir dela, a Internet vai deixar 
de existir como é vista hoje tornando, assim, onipresente.

% O que é 

O conceito de Internet das Coisas (IoT) está relacionado à interconexão de objetos distintos 
através de uma rede, sendo esta, muitas vezes, a Internet. Desse modo, elementos do mundo real, que 
antes funcionavam de maneira independente ao meio aos quais estavam inseridos, são capazes de 
interagir com outros objetos à sua volta e, assim, trocar informações que possam ser relevantes 
permitindo a agregação de novas funcionalidades.  Além disso, a IoT abre espaço para interação 
entre o mundo físico e o digital a partir de dispositivos capazes de capturar dados físicos no meio 
em que estão tais como, temperatura, distância etc., representá-los digitalmente e trasmití-los 
para outros dispositivos.

	
O termo ``Internet das Coisas'' foi citado pela primeira vez por Kevin Ashton, diretor executivo da 
AutoIDCentre do MIT, em 1999 enquanto realizava uma apresentação para promover a ideia do uso de 
Identificadores de Radio Frequência (RFID) na etiquetagem de produtos. O uso da tecnologia 
beneficiaria a logística da cadeia de produção \cite{kevin-ashton}. Apesar de o termo IoT ter sido 
usado apenas em 1999, aplicações práticas da ideia já existiam anos antes. Um exemplo disso, é a 
torradeira que podia ser ligada e desligada via internet criada em 1990 \cite{survey-suresh}.



% TODO: Projeções de grandes companhias (número de dispositivos, expansão)

A Internet das Coisas está em grande expansão. Estima-se que em 2020 cerca de 24 bilhões de 
dispositivos IoT estejam conectados, implicando em cerca de quatro dispositivos por pessoa. Para 
tanto, em torno de 6 trilhões de dólares serão investidos em desenvolvimento de tecnologias de 
hardware e software, como aplicações, segurança e dispositivos de hardware. Apesar da grande 
quantia investida, o setor é visto como promissor. Estima-se será gerado em torno de 13 trilhões de 
dólares 
em 2025 \cite{andrewmeola2016}. 

Para conectar uma grande quantidade de objetos são necessárias tecnologias, muitas delas sem fio, 
que permitam que os dispostivos interajam entre si trocando informações de maneira eficiente. A 
próxima seção tratará dessas tecnologias e a maneira com são organizadas para formar uma 
arquitetura. A seção seguinte apresenta alguns novos ``conceitos(sobre os smart tudo)'' que a IoT 
trouxe e algumas aplicações específicas para algumas das mais variadas áreas já existentes.

% TODO: GANCHO PARA AS TECNOLOGIAS

% TODO: Empresas, associações que abraçaram a ideia

% TODO: Benefícios, facilidade comodidade

%%%%%%%%   PARTE 2   %%%%%%%%
\section{Tecnologias}









% ==================================================================================================
\subsection{Bluetooth}

% O que é
% MELHORAR e referenciar
O Bluetooth é uma especificação de rede WPAN, ou seja, rede sem-fio pessoal, sendo descrito e 
especificado pela IEEE 802.15.1. O Bluetooth foi criado na década de 90 com o objetivo de unir
tecnologias distintas, tais como computadores, celulares entre outros a partir de uma padronização 
de comunicação sem fio entre os dispositivos \cite{jimkardach2008}. 

%\begin{figure}[H]
%	\centering
%	\caption[ABC]{Interconexão entre diversas classes de dispositivos}
%	\label{key}
%	\includegraphics[width=0.7\textwidth]{img/bluetooth-ecosystem}
%    \newline Fonte: Repositório de imagens Pixabay %(link: 
%    %https://pixabay.com/en/bluetooth-connectivity-wireless-1690677/)
%\end{figure}

Uma das principais características da tecnologia \textit{wireless} é o curto alcance de 
transmissão variando de centímetros até alguns metros \cite{huang2007bluetooth}. 

A tecnologia vem sendo usada ao longo dos últimos anos em diversas aplicações como transferência de 
arquivos entre dispositivos, transmissão de áudio entre smartphones e fones sem fio, dispositivos 
capazes de determinar contexto, como os beacons, entre outros.

% Topologias
No IEEE 802.15.1 há suporte para criação de redes \textit{ad-hoc}, aos quais, é desnecessário uma 
infraestrutura de rede para conexão dos dispositivos. A partir disso é possível criar redes 
chamadas \textit{picorredes}, nas quais os dispositivos são organizados em até oito associados, 
sendo um deles um mestre, ao qual coordena as operações, e os demais escravos 
\cite{bluetoothSIG2017}.

% Características
A tecnologia Bluetooth opera na faixa ISM de 2.4 GHz de uso livre em modo TDM com 
um delta de $625\mu$s, proporcionando uma taxa de transmissão máxima em torno de 2 Mb/s, podendo 
variar de acordo com o dispositivo e a categoria de tecnologia de Bluetooth utilizada. 
\cite{bluetoothSIG2017}.

% Forma de conexão.

\subsubsection{Categorias}

Segundo \cite{sig2017}, o Bluetooth pode ser categorizado em:

BR/EDR
%2.0 a 2.1

% REVER
Esta é a subdivisão mais popularizada do Bluetooth presente nas versões $2.0$ e $2.1$ do Bluetooth, 
onde as principais características são alta velocidade de transmissão alta em relação à outra 
categoria, baixo alcance e necessidade de conexão através de pareamento, onde os dispositivos 
confirmam a conexão. A partir disso, há um transmissão contínua de dados. 
Uma desvantagem é o consumo de energia considerável para o funcionamento do Bluetooth, já que há 
uma conexão contínua e uma taxa de transmissão que mantêm o dispositivo ativo por um longo 
período ininterrupto.
% taxa de dados
A taxa de transmissão gira em torno de 2Mb/s.
 
BLE (Smart)
% 4.0, 4.1, 4.2

% O que é
O \textit{Bluetooth Low Energy} (BLE) é a mais recente categoria do Bluetooth incorporada 
na versão 4.0, em 2011, além de ser a menos comum \cite{linklabs2015}.
% Foco
BLE está centrado no baixo consumo de energia para permitir que certos 
dispositivos não precisem recarregar ou trocar suas fontes de carga, muitas vezes uma bateria, por 
longos períodos, que podem chegar a anos. 
% Pareamento
Para uma conexão para transmissão de dados, ao contrário do BR/EPR, não é necessário um pareamento 
para realizá-la, além disso esta tem curta duração, na ordem de milissegundos.
%Taxa e alcance
Ademais, a taxa de dados é baixa e o alcance alto. A baixa taxa de dados decorre do modo de 
funcionamento dos dispositivos BLE, aos quais, enviam dados em rajadas, ou seja, de tempos em 
tempos dados são transmitidos em forma de \textit{broadcast} e os dispositivos que estiverem 
conectados receberão esses dados. Nos intervalos de tempo em que o dispositivo não transmite, ele 
``dorme'', isto é, entra em modo de consumo mínimo a fim de poupar energia.

%Aplicação
A aplicação prática dessas características está na IoT através de \textit{beacons} e  
\textit{wearables}, aos quais incorporam o BLE. Os beacons foram introduzidos pela \textit{Apple} 
em conjunto com o iOS 7, com o nome de \textit{iBeacon}, que permitia aos aplicativos 
possuíssem senso de localização \cite{apple2014}. 
Com esses dispositivos é possível 
aprimorar a experiência do usuário em estabelecimentos como museus, supermercados, shoppings, 
estádios, através da identificação de contexto, na qual, a partir da 
detecção de um beacon e da aproximação ou afastamento deste, uma aplicação móvel em um smartphone 
de um usuário pode exibir conteúdos, indicar promoções entre outros relacionados aquele dispositivo 
BLE.

%\begin{figure}[H]
%	\centering
%	\caption[ABC]{{\noindent Alguns exemplares de beacons}}
%	\label{fig:beacons}
%	\includegraphics[width=0.5\textwidth]{img/beacons.jpg}
%	\newline Fonte: Shine Solutions %(link: 
%%https://shinesolutions.com/2014/02/17/the-beacon-experiments-low-energy-bluetooth-devices-in-action/
%\end{figure}

Dual-mode
%
Esta categoria se refere a dispositivos, como \textit{smartphones} que precisam 
se conectar tanto com dispositivos BR/EDR, como fones de ouvido, e BLE, como 
\textit{beacons} \cite{sig2017}.

\subsubsection{Bluetooth 5.0}

A versão 5.0 do Bluetooth foi lançada em dezembro de 2016 (adopted-specfi) e trás consigo 
aprimoramentos em desempenho e segurança, garantindo duas vezes mais velocidade, quatro vezes mais 
alcance, oito vezes mais taxa de dados e, por fim, maior coexistência \cite{bluetoothsig2016}. 

Com a nova versão, veio a flexibilidade para construção de soluções baseadas em necessidade. 
Parâmetros como alcance, velocidade e segurança podem ser regulados para diversos objetivos a 
depender das aplicações \cite{bluetoothsig2016}.

Algumas atualizações contribuem para a redução de interferência com outras tecnologias sem fio, 
dessa forma, proporciona melhor coexistência entre dispositivos Bluetooth e de outras tecnologias, 
dentro do cenário emergente da IoT \cite{bluetoothsig2016}.






% ==================================================================================================
\subsection{RFID}

O RFID (Indentificação por Rádio Frequência) é uma tecnologia de identificação automática, entre 
diversas outras como código de barras, cartão inteligente e procedimentos biométricos, no entanto 
se distingue pelo modo de funcionamento, ou seja, por ondas eletromagnéticas. 
% INFLUÊNCIA EXTERNAS NO FUNCIONAMENTOS
Por outro lado, o RFID se destaca em relação às outras tecnologias em relação às 
influências externas no seu funcionamento, como sujeira, posição de leitura. Desse modo, não é 
necessário nem limpar ou reposicionar o dispositivo RFID para efetuar a leitura \cite{klaus2011}. 

% LEITOR E TRANSPONDER
No RFID, os dados são transmitidos através de ondas de rádio entre dois dispositivos: 
\textit{transponder} ou \textit{tag} e \textit{leitor}. O transponder é localizado no objeto 
identificado, um 
produto, equipamento etc., e nele são mantidos os dados de identificação. Já o leitor é responsável 
pela leitura e escrita dos dados presentes no transponder.

% FUNCIONAMENTO
Para a transmissão dos dados entre os dois dispositivos o leitor emite ondas de rádio na tag. Ao 
receber o estímulo, a tag responde com os dados contidos nela. Além disso, existem tags que 
utilizam a energia do campo eletromagnético gerado pelo leitor para seu funcionamento, sendos estas 
chamadas de \textit{passivas}. Existem, também, aquelas que possuem uma fonte própria de energia e 
por isso são denominadas \textit{ativas}.

Um exemplo de tag ativa é mostrada na Figura \ref{fig:active-rfid}.

%\begin{figure}[H]
%	\centering
%	\caption[ABC]{{\noindent Exemplo tag RFID ativa}}
%	\label{fig:active-rfid}
%	\includegraphics[width=0.3\textwidth]{img/active-tag.jpg}
%	\newline Fonte: Vnsky %(link: 
%	%http://www.vnsky.com/parts/2673180/ACTIVE-TAG-REFERENCE-DESIGN-KIT.html
%\end{figure}

Uma tag passiva é mostrada na Figura \ref{fig:passive-rfid}.

%\begin{figure}[H]
%	\centering
%	\caption[ABC]{{\noindent Exemplo tag RFID passiva}}
%	\label{fig:passive-rfid}
%	\includegraphics[width=0.3\textwidth]{img/passive.png}
%	\newline Fonte: Eletrosome %(link: 
%	%https://electrosome.com/rfid-radio-frequency-identification/
%\end{figure}

% TIPOS (alimentação)

% ATIVO
% PASSIVO

% FREQ. DE OPERAÇÃO
Uma das características mais importantes dos dispositivos RFID é a frequência de operação já que 
ela influi no distância máxima de operação. Tal fator é determinado pelo leitor. 
Os dispositivos são classificados, de acordo com a frequência de operação, em três grupos:

\begin{itemize} \parskip -4pt
	\item \textbf{LF (Baixa Frequência):} Entre 30kHz à 300kHz
	\item \textbf{HF (Alta Frequência):} Entre 3MHz à 30MHz
	\item \textbf{UHF (Ultra Alta Frequência)}: Entre 300MHz a 3GHz.
\end{itemize}

É possível distinguir pelo alcance:

\begin{itemize} \parskip -4pt
	\item \textbf{\textit{Long-range} ou longo alcance:} maior que um metro.
	\item \textbf{\textit{Remote-coupling} ou ligação remota:} até um metro
	\item \textbf{\textit{Close-coupling} ou ligação próxima:} até um centímetro.
\end{itemize}

Geralmente, a frequência de operação é diretamente proporcional ao alcance. Por exemplo, 
dispositivos de longo alcance operam na faixa UHF. 






% ==================================================================================================
\subsection{NFC}
% O_QUE É
O NFC é um sistema de comunicação sem fio derivado do RFID. Ele permite transações simples e 
seguras entre dois dispositivos a partir da curta distância de operação, em torno de 
4cm, e do funcionamento baseado em aproximação dos objetos em questão \cite{nfcforumabout2017}. 
Assim, é 
possível realizar leituras de tags e obter conteúdos de acordo com a aplicação, transferir dados 
entre smartphones entre outras funcionalidades.

% COMPATIBILIDADE
Outra vantagem do NFC é a compatibilidade com a infraestrutura de cartões sem contato 
existentes permitindo usar um único dispositivo em tecnologias diferentes. Desse modo, é possível 
interagir com tags RFID, por exemplo.


% FUNCIONAMENTO
Como o RFID, o NFC funciona através de ondas eletromagnéticas com uma taxa de transmissão máxima de 
424kbps \cite{nfcforumabout2017}. Além disso, pode operar em dois modos de comunicação 
\cite{brianjepsondoncolemantomigoe2014}: ativo e passivo. Assim como no RFID, é possível que os 
dispositivos NFC que contenham os dados usem a energia do leitor para transmitir seus dados, no 
modo passivo, ou usem uma fonte própria para tal procedimento, no modo ativo.


% MODOS DE OPERAÇÃO
Outra característica importante no NFC são os modos de operação. De acordo com 
\cite{nfcforumabout2017} existem três modos:

\begin{itemize} \parskip -4pt
	\item \textit{Leitor/Escritor de tag}: Tem por objetivo ligar o mundo físico ao digital através 
	de aplicações que leem e/ou escrevem em tags para obter dados e, assim, fornecer conteúdo ao 
	usuário relacionado à tag lida. Um exemplo é um smartphone ao ler uma tag NFC de um cartaz na 
	rua.
	\item \textit{Peer to Peer}: Visa conectar dispositivos por aproximação física e permite troca 
	de arquivos. Um exemplo é o Android Beam que permite troca de arquivos entre smartphones com o 
	sistema operacional da Google.
	\item \textit{Emulação de cartão}: Conecta o dispositivo do usuário em uma infraestrutura 
	possibilitando a simulação de um cartão, além da realização de transações financeiras e 
	identificação no sistema de transporte a partir da aproximação do dispositivo a um leitor 
	específico.
\end{itemize}

% CATEGORIAS
Há quatros tipos de tags definidas \cite{nfcforumtypespec2017}, sendo que todos operam no modo 
Leitor/Escritor descrito anteriorente : 

\begin{itemize} \parskip -4pt
	\item \textbf{Tipo 1}: 96 bytes de memória disponível e expansível para 2kiB. Usuário pode 
	configurá-la para somente leitura.
	\item \textbf{Tipo 2}: 48 bytes de memória disponível e expansível para 2kiB. Usuário pode 
	configurá-la para somente leitura.
	\item \textbf{Tipo 3}: Baseado no padrão industrial japonês e conhecido como FeliCa. Pode ser 
	configuradas para leitura/escrita ou somente leitura na fabricação. A memória disponível varia, 
	mas com um limite teórico de 1MiB por serviço.
	\item \textbf{Tipo 4}: A memória disponível varia estando acima de 35 kiB por serviço. É 
	possível ser configurada para leitura/escrita ou somente leitura.
\end{itemize}

O NFC possui um padrão com o qual dispositivos devem estar formatados, o NDEF (\textit{NFC Data 
Exchange Format}) um formato comum de comunicação \cite{brianjepsondoncolemantomigoe2014}. Desse 
modo, os dados armazenados em tags devem estar gravados nesse formato. A partir do NDEF é possível 
armazenar e trocar documentos binários como MIME, que incluem imagens, arquivos PDF entre outros, 
URL, texto simples entre outros.







% ==================================================================================================
\subsection{Zigbee}

% WHAT IS IT
O Zigbee é um protocolo padrão de comunicação de baixa-potência para redes sem-fio \textit{mesh}, 
ao qual permite que diversos dispositivo trabalharem em conjunto \cite{FALUDI2010}.

O Zigbee é descrito como um conjunto de camadas implementadas sobre o IEEE 802.15.4 \cite{FALUDI2010}, ao qual 
especifica a camada física(PHY) e o controle de acesso ao meio (MAC) para 
redes sem-fio de baixa potência \cite{IEEE802154_2011}.

As camadas do Zigbee, de acordo com \cite{FALUDI2010}, fazem:

\begin{itemize} \parskip -4pt
	\item \underline{Roteamento:} Tabelas de roteamento que definem como um nó envia dados até um 
	destino
	\item \underline{Rede Adhoc:} Criação automática de rede
	\item \underline{\textit{Self healing mesh}:} Descobe se nós se perderam da rede e a 
	reconfigura para garantir uma rota para os dispositivos conectados ao nó faltantes
\end{itemize}

O Zigbee opera na faixa não licenciada ISM, de 2,4GHz, o que permite sua expansão global e, assim, ser capaz de operar em qualquer local do mundo.

O Zigbee especifica que os nós das redes criadas possam assumir papeis específicos. Cada nó deve 
assumir uma das categorias a seguir \cite{FALUDI2010}:

\begin{itemize} \parskip -4pt
	\item \textit{Coordenador}: Responsável por criar a rede, distribuir endereços, manter a rede segura, 
	mantê-la em funcionamento entre outras funções que caracterizam a rede. Cada rede tem um e 
	apenas um coordenador.
	\item \textit{Roteador}: Tem capacidade de unir redes existentes, enviar e receber informações e rotear 
	informações, atuando como um intermediário entre dispositivos que, por estarem muito distantes 
	entre si, não podem se comunicar diretamente. É permitido às redes terem múltiplos roteadores, 
	podendo também não possuírem nenhum e, caso exista, cada roteador deve estar conectado a um 
	coordenador ou outro roteador.
	\item \textit{Dispositivo final}: É um tipo de nó capaz de se unir a redes além de enviar e receber 
	informações da rede. Além disso, podem se desligar de tempos em tempos para poupar energia. 
	Caso mensagens para um dispositivo final desligado sejam detectadas, o nó responsável por ele, 
	podendo ser um coordenador ou roteador, armazena as mensagens até que o nó desperte.
\end{itemize}

Há diversas topologias suportadas, nas quais, englobam os três tipos de nós e suas possíveis 
maneiras de organização \cite{FALUDI2010}:

\begin{itemize} \parskip -4pt
	\item \textit{Par a par}: Uma rede formada apenas por dois nós, sendo um deles, obrigatoriamente, um 
	coordenador e nó restante podendo ser um roteador ou dispositivo final.
	\item \textit{Estrela}: Nessa topologia, o coordenador se situa no centro da rede e os demais nós, 
	roteadores ou dispositivos finais, conectados apenas a ele, formando uma rede no formato de 
	estrela.
	\item \textit{Mesh}: Os dispositivos finais circundam os demais nós roteadores e coordenador. O 
	coordenador e roteadores atuam como intermediários, roteando mensagens para dispositivos 
	finais, outros roteadores ou para o coordenador. Apesar da nova função do coordenador, este 
	permanece no controle e gerenciamento da rede.
	\item \textit{Cluster tree}: Nessa topologia, cada roteador é responsável por um conjunto de 
	dispositivos finais. As mensagens vindas desses dispositivos devem ser encaminhadas 
	primeiramente para seu roteador responsável para então ser encaminhada ao destino na rede.
\end{itemize}

% TODO: Colocar ilustração das topologias

O Zigbee define três maneiras de identificação de nodos, que podem utilizadas em uma aplicação para 
diferenciar os nós.

\begin{itemize} \parskip -4pt
	\item 64 bits: Único e permanente para cada rádio fabricado.
	\item 16 bits: Dinamicamente configurado pelo coordenador ao entrar em uma rede. É único apenas 
	dentro do contexto da rede.
	\item Node Id: Pequena cadeia de texto. Não é possível garantir sua unicidade em nenhum 
	contexto, apesar disso, é mais amigável aos olhos humanos.
\end{itemize}

\subsubsection{Criação de uma rede}

Cada rede de sensores deve possuir um identificador chamado endereço PAN (Personal Area Network). 
Além disso, cada nó deve ter o mesmo PAN configurado e o mesmo canal de comunicação, que é 
escolhido de acordo com a disponibilidade pelo coordenador. 

Para que uma mensagem chegue a um destino, é necessário que o nó emissor tenha conhecimento do 
endereço do nó destinatário do pacote.


\subsubsection{XBee}

O XBee é um dispositivo fabricado pela Digi. Existem cerca de 
30 combinações de hardware, protocolos de \textit{firmware}, potência de transmissão e antenas.

Apesar das diversas combinações, há duas versões básicas do XBee: Série 1 e Série 2.

Os nós XBee Série 1 proveem comunicações ponto a ponto, bem como uma implementação proprietária de 
rede \textit{mesh}. Já os nós Série 2 permitem diversas derivações de padrões de redes mesh Zigbee.

% IMAGEM XBEE S1 e XBEE S2
% ==================================================================================================
\section{Aplicações}

\subsection{Smart home}

Uma \textit{Smart home} ou casa inteligente é formada por um conjunto de sensores e atuadores 
conectados em rede, aos quais, podem se comunicar entre si e que permitem ao morador o controle de 
maneira remota de diversos dispositivos tais como lâmpadas \cite{mandula2015}, 
condicionamento de ar, segurança entre outros.

A empresa \textit{Amazon}, empresa de tecnologia dos Estados Unidos, oferece o \textit{Amazon 
Echo}, um dispositivo que oferece diversas funcionalidades multimídia, como reprodução de músicas 
através 
de controle por voz, inclusive se algo estiver tocando, além de oferecer informações como previsão 
do tempo, notícias, tráfego entre outros através do \textit{Alexa Voice Service}. É capaz de 
controlar a luz, tomadas e termostatos além de ser compatível com produtos de empresas, como 
Samsung, Philips entre outras, com foco em \textit{smart homes} \cite{amazon2017}.

\subsection{Smart grid}

\textit{Smart grid} é uma rede elétrica inteligente, na qual, através da tecnologia coleta dados de 
consumo e dos distribuidores toma ações a partir deles, tornando o sistema mais eficiente, seguro 
e sustentável \cite{Cecilia2016}. As smart grids têm três componentes importantes: rede de 
transmissão inteligente, às quais contêm sensores, troca de informações, controle e tecnologias de 
comunicação que proporcionam uma transmissão eficiente, tecnologia da informação e comunicação de 
smart grid  e tecnologia de medição inteligente, no qual, além de realizar medições como os 
equipamentos tradicionais, é capaz de trocar informações com a rede inteligente. 

Empresas como a \textit{Texas Instruments} (TI) têm investido em soluções para smart grid que 
proporcionam segurança, eficiência e inteligência. A TI oferece soluções para monitoramento da rede 
através de medidores de eletricidade, gás e calor, além de tecnologias para comunicação como o 
\textit{Power Line Communications} entre outras \cite{texasinstruments2017}.

%\subsection{Smart city}
% Smart city 
% Tudo conectado
% Eficiência

\subsection{Wearables}

\textit{Wearables} são dispositivos computacionais vestíveis, ou seja, são itens que uma pessoa pode usar no dia a dia como, roupas, relógios, óculos, sapatos entre outros e, ainda obter novas funcionalidades, graças à presença da tecnologia nesses dispositivos. Isso é possível devido a sensores aos quais medem sinais vitais do corpo humano e dados do ambiente, a depender da aplicação, além de pequenos equipamentos de hardware responsáveis por ler os dados dos sensores e comandos do usuário e assim tomar as ações necessárias de acordo com a situação.

Os \textit{wearables} têm aplicações nas mais diversas áreas, partindo das áreas da saúde e esportes até lazer e trabalho. 

Algumas empresas têm investido na área de dispositivos vestíveis. Um exemplo é a Microsoft que vem desenvolvendo o \textit{Holo Lens}, um dispositivo que se assemelha a um óculos, com foco em realidade aumentada. A tecnologia pode ser usada na concepção e design de produtos, educação, astronomia entre outros.

Outra tecnologia em ascensão é o \textit{smart watch}, relógio digital conectado à internet, dotado de aplicativos entre outras funções que vão além de mostrar as horas.

Na área de esportes e saúde, existem as \textit{smart bands}, pulseiras capazes de monitorar sinais, como batimentos além das calorias eliminadas durante um exercício, distância percorrida entre outros.


\subsection{Indústrias}
% Quarta revolução industrial,

% Apoio a logística

\subsection{Transportes}


A empresa \textit{General Eletric} (GE), a maior empresa digital industrial do mundo, atua em diversos setores como saúde, aviação, transporte, energias renovável, entre outros \cite{generalelectric2017} . A empresa fabrica uma locomotiva, modelo \textit{Evolution}, com cerca de 250 sensores, resultando em 150 mil leituras por minuto. Com a leitura dos sensores, dados como tempo, pressão de óleo, temperatura, velocidade entre outros, podem ser utilizado para determinar a performance da máquina em um dado momento. Além disso, é possível prever quando surgirá algum problema, devido a algum componente que indica falha, a partir do software de análise em escala industrial \textit{Predix} \cite{danielterdiman2014}.


\subsection{Turismo}
% Exemplo de turismo do artigo
Uma aplicação proposta consiste em um sistema de informação turístico com intuito de expandir a experiência dos visitantes nos diversos pontos turísticos no Japão. Isso é possível graças ao uso de beacons, sensores sem fio descritos na seção anterior, e realidade aumentada, viabilizada por uma aplicação móvel \cite{SHIBATA2016}. 
