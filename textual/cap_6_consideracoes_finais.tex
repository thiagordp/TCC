\chapter{Considerações Finais}
\label{cap:consideracoes_finais}

%%%%%%%%%%%%%%%%%%  RETOMAR O OBJETIVO GERAL %%%%%%%%%%%%%%%%%
Este trabalho teve como objetivo o desenvolvimento de um modelo de geladeira inteligente que facilitasse o dia a dia dos usuários a partir da análises das interações destes, baseando-se nos conceitos de IoT e Sistemas de Recomendação.

%%%%%%%%%%%%%%%%%% COMENTÁRIOS SOBRE OS OBJ. ESPECÍFICOS ATINGIDOS OU NÃO %%%%%%%%%%%%%%%%%
Para atingir o objetivo geral, inicialmente, o modelo foi projetado. Assim, definiu-se o escopo do projeto, bem como, todos os seus componentes, sendo eles, o sistema da geladeira, o servidor de serviços, as bases de dados e de recomendações, a interface de usuário e, por fim, os serviços disponibilizados pelo mercado.

% Elaborar e desenvolver um projeto de leitura e monitoramento dos produtos contidos na geladeira.
Com o modelo criado, um projeto de leitura e monitoramento dos produtos contidos na geladeira foi elaborado e desenvolvido. Para tanto, fez-se uso da tecnologia RFID e do conceito de códigos EPC. Assim, à cada produto foi agregada uma etiqueta RFID contendo o código único possibilitando a identificação de maneira simples no momento em que este fosse inserido na geladeira. Além disso, elaborou-se uma estrutura para conter os equipamentos de leitura e processamento. Ao final, todas as informações de produtos disponíveis são enviadas ao servidor principal.

% Implementar um sistema de análise das interações e recomendação de produtos.
A partir do projeto de leitura e monitoramento, ou seja, a fonte de dados sobre os hábitos do usuários, tornou-se viável a implementação do sistema de análise das interações e recomendação de produtos e receitas. Para tanto, utilizou-se de algumas abordagem de sistemas de recomendação. Ao final, o sistema foi capaz de recomendar produtos novos ao usuários a partir do conceito de filtragem colaborativa. Além disso, foi possível sugerir produtos essenciais considerando recomendações baseadas em conteúdo, bem como, sugestões de receitas, levando-se em conta tanto os produtos disponíveis na geladeira quanto o perfil do usuário, ou seja, os produtos em que o usuário teve mais interesse.

% Elaborar um cenário que permita a avaliação do modelo proposto. 
Como forma de avaliar o sistema implementado elaborou-se um cenário de testes, onde definiu-se parâmetros como o número de usuários, de produtos, de interações etc. Ademais, diagramas de fluxos de execução foram desenvolvidos a fim de fornecerem uma base para a avaliação do sistema.

% Avaliar e discutir os resultados obtidos a partir do sistema proposto.
Ao final, o sistema demonstrou, em todos os fluxos elaborados para o cenário, os respectivos comportamentos esperados. Portanto, conclui-se que o objetivo geral do trabalho foi atingido, já que a análise de interações implementada, em conjunto com o monitoramento de produtos, proveem funcionalidades que eliminam ou facilitam algumas atividades diárias dos usuários e que, por consequência, objetiva facilitar o seu dia-a-dia. 

% Limitações e dificuldades
Durante a implementação do trabalho algumas dificuldades foram encontradas.
%A principal delas se refere ao modo como a leitura de etiquetas RFID dos produtos, ou seja, com apenas dois leitores. 
A principal delas se refere à limitação do número de produtos na geladeira, já que apenas dois leitores foram utilizados.
Claramente, é improvável que uma geladeira contenha somente dois produtos, no entanto, não encontrou-se leitores capazes de efetuar a leitura de um maior número de etiquetas de forma simultânea que tivesse um custo acessível ao projeto.

% Considerações sobre o trabalho


\section{Trabalhos Futuros}

Durante o desenvolvimento desse trabalho foram elencadas outras possibilidades como trabalhos futuros. As principais se relacionam ao sistema de leitura e monitoramento, além dos algoritmos de recomendação e na sincronização da metainformação. No que se refere ao sistema de leitura, percebeu-se que é factível o aprimoramento do mesmo a partir da modificação dos dispositivos de leitura RFID. Isso seria possível a partir do uso de RFID UHF, no entanto, seria necessário encontrar um meio de baratear os custos desta tecnologia. Outra maneira, seria a criação de uma plataforma única de leitura por RFID que se estendesse por todo o espaço de um ``andar'' da geladeira. Outra possibilidade de melhoria no monitoramento se refere ao uso de visão computacional como forma de reconhecer os produtos. Por fim, um avanço no projeto está na substituição da placa Raspberry\textsuperscript{\textregistered} por um sistema computacional projetado de acordo com as necessidades do sistema, o que permitiria uma redução em custos.

Outra possibilidade de trabalho futuro, como mencionado, está na melhoria dos algoritmos de recomendação. O primeiro ponto a ser revisto refere-se aos dados de entrada, já que informações adicionais poderiam conferir maior precisão nas sugestões. Entre elas está a questão geográfica, ou seja, os costumes de uma determinada região tendem a determinar quais produtos e receitas os usuários apresentariam maior interesse. Além disso, informações específicas do usuário voltadas às restrições alimentares, tais como, intolerância à lactose, glúten entre outros, bem como veganismo. Como última melhora relacionada a captura de dados, tem-se a possibilidade de utilização de mais detalhes acerca dos produtos recomendados, como informações nutricionais, por exemplo. 


Em relação à recomendação em si, um aspecto que pode ser melhorado é a questão temporal nas preferências de usuários, ou seja, os gostos do usuários variam com o tempo. Além disso, é possível utilizar algoritmos mais eficientes.

Há também implementações que ficaram pendentes neste trabalho. A primeira delas se refere à sincronização da base de metainformação a partir do mercado. Além disso, não foi implementada a possibilidade de o usuário personalizar suas configurações através do respectivo menu na interface.


Um ponto de alta importância que deve ser considerado é a segurança. Como descrito no Capítulo 2, em IoT, este é um dos desafios que devem ser superados. No contexto da aplicação, é imprescindível que se utilize de ferramentas e protocolos de segurança pra garantir transações de compras sem fraudes, envio preciso de informações sobre produtos contidos, entre outros.

Outro ponto relevante a ser considerado, no futuro, é a diferenciação dos usuários, ou seja, uma geladeira é utilizada por diversas pessoas em uma casa. Assim, um meio pode ser utilizado para diferenciar os usuários em meio às interações que ocorrem com a geladeira.

Por fim, possibilidades de integração com outras tecnologias e produtos existentes foram levantadas. A primeira delas é a integração com a assistente pessoal da Amazon\textsuperscript{\textregistered}, a Alexa\textsuperscript{\textregistered}, à qual permitiria que  usuários comandassem a geladeira por comandos de voz. Além disso, o conceito de monitoramento de produtos e interações com o usuário poderiam ser expandidos à outros compartimentos, por exemplo, uma dispensa ou algum tipo de recipiente para alimentos não perecíveis. 

% Alexa

%Evolução no protótipo
% Evolução nos alg de rec

% Estudar também a questão geográfica (recomendação de acordo com o lugar) porque no caso de comida, as comidas mais apropriadas para cada indivíduo estão imersas num contexto cultural.

%Visão computacional: integrar computação visual na identificação de produtos e das interações que ocorrem, ou seja, usar computação visual para 

%Integração com a assistente pessoal Alexa, onde o Amazon Echo se torna um 'gateway' e a geladeira se comunica apenas com o Echo.

%Uso de uma placa de leitura única, por toda a prateleira

%Possibilidade de ser aplicado em qualquer lugar

% Substituição do Raspberry por um sistema embarcado propriamente dimensionado para o projeto e que integre com o resto do sistema e possa gerenciar melhor o consumo de energia.

% Comunicação com outros dispositivos da casa

% Recomendação mais inteligente que consiga recomendar produtos para perfis específicos como intolerantes a lactose, vegetarianos a partir da configuração do usuário.

% Consideração do aspecto temporal em recomendações, ou seja, que as preferências de usuário variam ao longo do tempo.

% Diferenciação de quem está coletando o produto
% - Reconhecimento facial da pessoa