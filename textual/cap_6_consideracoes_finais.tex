\chapter{Considerações Finais}
\label{cap:consideracoes_finais}

%%%%%%%%%%%%%%%%%%  RETOMAR O OBJETIVO GERAL %%%%%%%%%%%%%%%%%
Este trabalho teve como objetivo o desenvolvimento de um modelo de geladeira inteligente que facilitasse o dia a dia dos usuários a partir da análises das interações destes, baseando-se nos conceitos de Internet das Coisas e Sistemas de Recomendação.

%%%%%%%%%%%%%%%%%% COMENTÁRIOS SOBRE OS OBJ. ESPECÍFICOS ATINGIDOS OU NÃO %%%%%%%%%%%%%%%%%
Para atingir o objetivo geral, inicialmente, o modelo foi projetado. Assim, definiu-se o escopo do projeto bem como todos os seus componentes, sendo eles, o sistema da geladeira, o servidor de serviços, as bases de dados e de recomendações, a interface de usuário e, por fim, os serviços disponibilizados pelo mercado.

% Elaborar e desenvolver um projeto de leitura e monitoramento dos produtos contidos na geladeira.
Com o modelo criado, um projeto de leitura e monitoramento dos produtos contidos na geladeira foi elaborado e desenvolvido. Para tanto fez-se uso da tecnologia RFID e do conceito de códigos EPC. Assim, à cada produto era agregada uma etiqueta RFID contendo o código único o que possibilitou a identificação de maneira simples no momento em que fosse inserido na geladeira. Além disso, montou-se uma estrutura que armazenasse os equipamentos de leitura e processamento. Ao final, todos as informações de produtos disponíveis estavam sendo enviadas ao servidor principal.

% Implementar um sistema de análise das interações e recomendação de produtos.
A partir do projeto de leitura e monitoramento, ou seja, a fonte de dados sobre os hábitos do usuários, tornou-se viável a implementação do sistema de análise das interações e recomendação de produtos e receitas. Para tanto, utilizou-se de algumas abordagem de sistemas de recomendação. Ao final, o sistema foi capaz de recomendar produtos novos ao usuários a partir do conceito de filtragem colaborativa. Além disso, passou-se a sugerir produtos considerados essenciais, a partir de recomendações baseadas em conteúdoa, bem como sugestão de receitas, a partir, tanto dos produtos disponíveis na geladeira, como do perfil do usuário, ou seja, nos produtos que o usuário mais apresentou interesse.

% Elaborar um cenário que permita a avaliação do modelo proposto. 
Como forma de avaliar o sistema implementado elaborou-se um cenário de testes, onde definiu-se parâmetros como número de usuários, de produtos, interações etc. Ademais, diagramas de fluxos de execução foram desenvolvidos a fim de fornecerem uma base para a avaliação do sistema.

% Avaliar e discutir os resultados obtidos a partir do sistema proposto.
Ao final, o sistema demonstrou, em todos os fluxos montados para o cenário, os respectivos comportamentos esperados. Portanto, conclui-se que o objetivo geral do trabalho foi atingido, já que a análise de interações implementada, em conjunto com o monitoramento de produtos, proveem funcionalidades que eliminam algumas atividades das mãos dos usuários e que, por consequência, tornam seu dia-a-dia mais facilitado. 

% Limitações e dificuldades
Durante a implementação do trabalho algumas dificuldades foram encontradas. A principal delas se refere ao modo como a leitura de etiquetas RFID dos produtos, ou seja, com apenas dois leitores. Claramente, é improvável que uma geladeira contenha até dois produtos, no entanto, não encontrou-se leitores capazes de efetuar a leitura de um maior número de etiquetas de forma simultânea, mas que tivesse um custo acessível ao projeto.

% Considerações sobre o trabalho


\section{Trabalhos Futuros}

Durante o desenvolvimento desse trabalho foram elencadas outras possibilidades como trabalhos futuros. As principais se relacionam ao sistema de leitura e monitoramento, além dos algoritmos de recomendação. No que se refere ao sistema de leitura, percebeu-se que é factível o aprimoramento do mesmo a partir da modificação dos dispositivos de leitura RFID. Isso seria possível a partir do uso de RFID UHF, no entanto, seria necessário encontrar um meio de baratear os custos desta tecnologia. Outra maneira, seria a criação de uma plataforma única de leitura por RFID que se estendesse por todo o espaço de um ``andar'' da geladeira. Outra de melhora no monitoramento poderia estar no uso de visão computacional como forma de reconhecer os produtos. Por fim, um avanço no projeto está na substituição da placa Raspberry\textsuperscript{\textregistered} por um sistema computacional projetado de acordo com as necessidades do sistema, o que permitiria uma redução em custos.

Outra possibilidade de trabalhos futuros, como dito está na melhoria dos algoritmos de recomendação. O primeiro ponto a ser revisto está nos dados de entrada, já que informações adicionais poderiam conferir maior precisão nas sugestões. Entre elas está a questão geográfica, ou seja, os costumes de uma determinada região tendem a determinar quais produtos e receitas os usuários apresentariam maior interesse. Além disso, informações específicas do usuário, como restrições alimentares como intolerância à lactose, glúten entre outros, bem como veganismo. Como última melhora relacionada às entradas, tem-se a possibilidade de utilização de mais detalhes a cerca dos produtos recomendados, como informações nutricionais, por exemplo. 

Em relação à recomendação em si, um aspecto que pode ser melhorado é a questão temporal nas preferências de usuários, ou seja, os gostos do usuários variam com o tempo. Além disso, é possível utilizar algoritmos mais eficientes.

Por fim, possibilidades de integração com outras tecnologias e produtos existentes foram levantadas. A primeira delas é a integração com a assistente pessoal da Amazon\textsuperscript{\textregistered}, a Alexa\textsuperscript{\textregistered}, à qual permitiria que os usuários comandassem a geladeira por comandos de voz. Além disso, o conceito de monitoramento de produtos e interações com usuário poderia ser expandidos para outros itens do dia a dia, como dispensa de alimentos poderiam ser utilizadas. 

% Alexa

%Evolução no protótipo
% Evolução nos alg de rec

% Estudar também a questão geográfica (recomendação de acordo com o lugar) porque no caso de comida, as comidas mais apropriadas para cada indivíduo estão imersas num contexto cultural.

%Visão computacional: integrar computação visual na identificação de produtos e das interações que ocorrem, ou seja, usar computação visual para 

%Integração com a assistente pessoal Alexa, onde o Amazon Echo se torna um 'gateway' e a geladeira se comunica apenas com o Echo.

%Uso de uma placa de leitura única, por toda a prateleira

%Possibilidade de ser aplicado em qualquer lugar

% Substituição do Raspberry por um sistema embarcado propriamente dimensionado para o projeto e que integre com o resto do sistema e possa gerenciar melhor o consumo de energia.

% Comunicação com outros dispositivos da casa

% Recomendação mais inteligente que consiga recomendar produtos para perfis específicos como intolerantes a lactose, vegetarianos a partir da configuração do usuário.

% Consideração do aspecto temporal em recomendações, ou seja, que as preferências de usuário variam ao longo do tempo.

% Diferenciação de quem está coletando o produto
% - Reconhecimento facial da pessoa