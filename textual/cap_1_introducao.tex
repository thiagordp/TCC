\chapter{Introdução}

%TODO: Introdução: Contextualização dos dias atuais... (Era uma vez)
% Apresentar o tema e o que significa

%TODO: Escrita sobre Internet das Coisas

	%TODO: Tecnologias que proporcionam a IoT

%TODO: Escrita sobre Objetos inteligentes

%TODO: Escrita s\part{title}obre Sistemas de Recomendação


% Contextualização de todo o trabalho

% Apresentar o cenário.

Nas últimas décadas, presencia-se os acelerados avanços em ciência e tecnologia impulsionados por empresas dos mais diversos ramos e pelas constantes pesquisas nas universidades. 

Um dos avanços mais significativos é a Internet, com um grande impacto no desenvolvimento da economia global e na sociedade atual. Em duas décadas tem decorrido um grande crescimento na disponibilidade do acesso à rede. Em setembro de 2016, o número de usuário da rede mundial de computadores era de, aproximadamente, 3,75 bilhões, cerca de metade da população mundial e, aproximadamente, 92 vezes maior em relação ao ano 2000 \cite{Stats2017}.  Outro grande avanço se tem nos celulares, aos quais evoluíram tanto nos últimos anos que passaram de simples e grandes telefones sem fio à dispositivos menores, no entanto, com acesso a Internet, recursos avançados de áudio e vídeo e poder de processamento equiparável ao de computadores de mesa (\textit{desktops}) e/ou notebooks.

Em meio ao domínio da Internet, um novo paradigma surgiu no meio acadêmico e aos poucos ganha terreno nas grandes empresas. A sua proposta é levar a tecnologia a objetos do dia a dia, como condicionadores de ar, lâmpadas, fogões etc., e, assim, criar novas formas de interação além de funcionalidades inéditas, seguindo o exemplo dos \textit{smartphones}. 

Mencionada pela primeira vez, por Kevin Ashton, em 1999 \cite{Ashton2009}, a Internet das Coisas ou IoT (em inglês, \textit{Internet of Things}) está cada vez mais próxima da realidade. Abrigará um variado ecossistema de dispositivos com capacidade de processamento, sensoriamento, conexão com demais dispositivos e, em muito casos, com a Internet, entre outros avanços. Estima-se que, em 2020, cerca de 24 bilhões de dispositivos IoT estejam conectados, implicando em cerca de quatro dispositivos por pessoa \cite{Meola2016}.   
Contudo, com o crescimento do número de itens computacionais, a quantidade de dados gerado por eles também cresce, mas de maneira exponencial \cite{Chiang2016}. A partir disso, o fluxo de dados na rede de internet se intensifica a ponto de comprometer o desempenho desta. Isso ocorre devido ao modelo atual de funcionamento da rede, ou seja, centralizado. Portanto, uma nova arquitetura se faz necessária para incorporar os dispositivos IoT à Internet tradicional. % já citado
Nesse contexto, a \textit{fog computing} ou computação de neblima, surge como potencial solução, com base na proposta de uma nova forma de organização da rede para complementar a atual. Isso é possível com base na aproximação de algumas funcionalidades centralizadas em servidores aos dispositivos que as utilizam \cite{Chiang2016}. Para tanto, um dispositivo de rede seria responsável por uma \textit{nuvem local}. Assim, os dispositivos IoT se comunicariam com esse equipamento e obteriam funcionalidades necessárias de maneira mais eficiente. A nuvem local se comunicaria diretamente com a nuvem convencional, transferindo apenas informações necessárias \cite{Syed2016}.

Os objetos inteligentes, ou \textit{smart objects}, com funcionalidades expandidas como comunicação, sensoriamento, processamento e atuação sobre o ambiente, promovem a interação entre o mundo físico (analógico) e o mundo digital\cite{Stojkoska2017}. Isso ocorre graças a sensores capazes de capturar grandezas como temperatura e luminosidade e, a partir disso, permite que aplicações tenham conhecimento do contexto do ambiente. Baseando-se nesses conceitos, algumas companhias vêm colocando no mercado novos produtos com características citadas. Como exemplo, é citável o Amazon Echo\textsuperscript{\textregistered}\footnote{https://www.amazon.com/Amazon-Echo-Bluetooth-Speaker-with-WiFi-Alexa/dp/B00X4WHP5E}, um dispositivo que opera com o serviço de assistente pessoal Alexa, e interage com pessoas em uma casa a partir de comando de voz. Outro produto destacável, é a \textit{smart lock} da empresa Nuki\textsuperscript{\textregistered}\footnote{https://nuki.io/en/shop/nuki-smart-lock/}, pelo qual é possível abrir e fechar a porta apenas com um toque no aplicativo móvel pelo \textit{smartphone} ou através de um \textit{smart watch}. Por outro lado, áreas como esportes também recebem atenção. Por fim, o último exemplo é o CARV\textsuperscript{\textregistered}\footnote{https://www.kickstarter.com/projects/333155164/carv-the-worlds-first-wearable-that-helps-you-ski}, um dispositivo vestível ou \textit{wearable}, ao qual propõe um calçado para praticantes de ski capaz de analisar em tempo real o modo de esquiar e fornecer informações detalhadas sobre.

Os \textit{smart objects} poderão, a partir da IoT, operar em conjunto e comporem os chamados \textit{smart environments}, ambientes nos quais a integração dos dispositivos agrega novas funcionalidades e formas de interação para aquele ambiente \cite{Asano2016}. Entre os ambientes inteligentes emergentes estão as \textit{smart grids}, às quais propõem a atualização do sistema elétrico atual a partir do uso da tecnologia. Uma das principais mudanças será o direcionamento do fluxo de energia e informações em dois sentidos. Como consequência, será possível consumir e fornecer energia para o sistema elétrico, bem como trocar informações sobre o estado da rede de eletricidade, o consumo entre outros avanços. Tudo isso será viável em virtude da capacidade de sensoriamento, troca de informações, controle e de tecnologia da informação e comunicação \cite{Cecilia2016}.  

Além das \textit{smart grids}, outro ambiente em expansão é a \textit{smart home}. Através dela, os moradores de uma casa podem interagir com um ambiente inteligente capaz de responder ao seus comportamentos e prover diversas funcionalidades \cite{Silva2012}. Isso se deve  à presença de dispositivos dotados com tecnologias de sensoriamento, controle e comunicação. Além disso, é possível subdividir \textit{smart homes} em ambientes menores. Um desses ambientes é a cozinha inteligente ou \textit{smart kitchen}, na qual, é capaz de prover ao usuário novas maneiras de interagir com os utensílios na preparação de alimentos, escolha de produtos entre outros. A partir disso, surgem diversas oportunidades em termos de criação de produtos.

%TODO: Conectar melhor.
%TODO: Reescrever aplicações.

As interações das pessoas com os ambientes citados gerará uma grande quantidade de dados. Um aproveitamento eficiente desses dados pode ampliar as aplicações da IoT.
% Como amplia? Escrever mais sobre as características, 
Uma das diversas formas para colocar essa ideia em prática são os sistemas de recomendação. Com base nas preferências indicadas pelo usuário ou no seu comportamento, esses sistemas buscam selecionar e fornecer informações relevantes \cite{Filho2008}.
Sistemas de recomendação são divididos em duas classes: filtragem colaborativa em que as recomendações são feitas com base na similaridade das preferências de usuários com os demais, isto é, em gostos de outros usuários em produtos, serviços etc., que o usuário não conhece, mas tem alta probabilidade de interesse. Outra categoria é a baseada em conteúdo, na qual os conteúdos apresentados ao usuário são baseados nas suas próprias preferências, ou seja, em itens semelhantes aos de interesse. Por fim, há a abordagem mista, em que ambas as categorias citadas são mescladas, aproveitamento, desse modo, as melhores características de cada uma \cite{Thomas2016}.
Além disso, as aplicações de sistemas de recomendação são aplicáveis nos mais diversos campos, entre eles, aplicações de streaming de filmes e séries, sites de vendas online além de outras aplicações como sistemas capazes de propor pontos de carga para condutores de carros elétricos \cite{Ferreira2011} e notícias personalizadas \cite{Yeung2010}.

\section{Problemática}
% Explicação:
% A problemática deve enfatizar os desafios identificados na literatura, ou seja, durante a leitura dos artigos alguns desafios foram identificados nas áreas de pesquisa do trabalho (Internet das Coisas, Smart Things ou correlatos, Sistemas de Recomendação) e devem ser apresentados. A partir da dissertação sobre tais desafios se finaliza a seção com uma pergunta de pesquisa, geralmente iniciando pela palavra "Como".

% Fog
Com o avanço da Internet das Coisas, a rede de internet necessitará ser adaptada para suportar o grande número de dispositivos e alto volume de dados que serão transmitidos a todo momento.

Além disso, como cada dispositivo estará conectado à rede, ele estará, portanto, ao alcance de ciber criminosos. No entanto, a segurança em IoT não está tão avançada para garantir segurança desses dispositivos.

Um dos problemas no monitoramento e registro de itens em uma geladeira é o modo com o qual é feito. Muitas propostas fazem uso dispositivos nos quais operam por ondas eletromagnéticas. No entanto, os alimentos que contém água (tem mais) interferem no desempenho das leituras.

Citar só problemas em aberto ou que sejam diretamente relacionados ao trabalho?

%TODO: Escrever sobre o problema em questão

%TODO: Concluir com a pergunta de pesquisa
Desse modo, tem-se como objetivo de pesquisa: ``Como [...] ?''.

\section{Objetivos}
Esta seção apresenta o objetivo geral e os objetivos específicos do trabalho.

\subsection{Geral}

\textit{Rascunho}

Desenvolver uma geladeira capaz de monitorar os produtos contidos nela e prover recomendações de receitas com base nos padrões de consumo dos produtos.

% TODO: Detalhar especificamente o que dá pra fazer ou ser mais geral?
% TODO: Propor um sistema diferente para geladeiras ou uma geladeira com recursos novos?

\subsection{Específicos}
\begin{enumerate} \parskip -4pt
	\item Levantar o estado da arte com relação a Internet das Coisas e Sistemas de Recomendação
	\item Propor um sistema de monitoramento de produtos 
	\item Propor um projeto de leitura e monitoramento dos produtos contidos na geladeira.
\end{enumerate}

\section{Justificativa}


\section{Organização do trabalho}

O capítulo 2 é sobre Internet das Coisas

O capítulo 3 é sobre Sistemas de Recomendação

O capítulo 4 é sobre o Sistema Proposto

O capítulo 5 é sobre a Avaliação do Sistema Proposto
	- Descrever um cenário
	- Avaliar e discutir o cenário a partir do sistema proposto
	
O capítulo 6 é sobre Considerações Finais




